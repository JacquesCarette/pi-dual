\documentclass[preprint]{sigplanconf}

\usepackage{graphicx}
\usepackage{comment}
\usepackage{amsmath}
\usepackage{xspace}
\usepackage{amssymb}
\usepackage{stmaryrd}
\usepackage{proof}
\usepackage{multicol}
\usepackage[nodayofweek]{datetime}
\usepackage{etex}
\usepackage[all, cmtip]{xy}
\usepackage{xcolor}
\usepackage{listings}
\usepackage{multicol}
\newcommand\hmmax{0} % default \newcommand\bmmax{0} % default 4
\usepackage{bm}
\usepackage{cmll}

\newcommand{\fname}[1]{\ulcorner #1 \urcorner}
\newcommand{\fconame}[1]{\llcorner #1 \lrcorner}

\newcommand{\xcomment}[2]{\textbf{#1:~\textsl{#2}}}
\newcommand{\amr}[1]{\xcomment{Amr}{#1}}
\newcommand{\roshan}[1]{\xcomment{Roshan}{#1}}

\newcommand{\asterix}[0]{*}

\newcommand{\ie}{\textit{i.e.}\xspace}
\newcommand{\eg}{\textit{e.g.}\xspace}

\newcommand{\lcal}{\ensuremath{\lambda}-calculus\xspace}
\newcommand{\G}{\ensuremath{\mathcal{G}}\xspace}

\newcommand{\code}[1]{\lstinline[basicstyle=\small]{#1}\xspace}
\newcommand{\name}[1]{\code{#1}}

\def\newblock{}

\newenvironment{floatrule}
    {\hrule width \hsize height .33pt \vspace{.5pc}}
    {\par\addvspace{.5pc}}

\newtheorem{theorem}{Theorem}[section]
\newtheorem{lemma}[theorem]{Lemma}
\newtheorem{definition}[theorem]{Definition}
\newtheorem{proposition}[theorem]{Proposition}
\newenvironment{proof}[1][Proof.]{\begin{trivlist}\item[\hskip \labelsep {\bfseries #1}]}{\end{trivlist}}

\newcommand{\arrow}[1]{\mathtt{#1}}

\newcommand{\dgm}[2][0.95]{
\begin{center}
\scalebox{#1}{
\includegraphics{diagrams/#2.pdf}
}
\end{center}
}

%subcode-inline{bnf-inline} name langRev
%! swap+ = \mathit{swap}^+
%! swap* = \mathit{swap}^*
%! dagger =  ^{\dagger}
%! assocl+ = \mathit{assocl}^+
%! assocr+ = \mathit{assocr}^+
%! assocl* = \mathit{assocl}^*
%! assocr* = \mathit{assocr}^*
%! identr* = \mathit{uniti}
%! identl* = \mathit{unite}
%! dist = \mathit{distrib}
%! factor = \mathit{factor}
%! eta = \eta
%! eps = \epsilon
%! eta+ = \eta^+
%! eps+ = \epsilon^+
%! eta* = \eta^{\times}
%! eps* = \epsilon^{\times}
%! trace+ = trace^+
%! trace* = trace^{\times}
%! ^^^ = ^{-1}
%! (o) = \circ
%! (;) = \fatsemi
%! (*) = \times
%! (+) = +
%! LeftP = L^+
%! RightP = R^+
%! LeftT = L^{\times}
%! RightT = R^{\times}
%! alpha = \alpha
%! bool = \textit{bool}
%! color = \textit{color}
%! Gr = G

%subcode-inline{bnf-inline} regex \{\{(((\}[^\}])|[^\}])*)\}\} name main include langRev
%! Gx = \Gamma^{\times}
%! G = \Gamma
%! [] = \Box
%! |-->* = \mapsto^{\asterix}
%! |-->> = \mapsto_{\ggg}
%! |--> = \mapsto
%! <--| = \mapsfrom
%! |- = \vdash
%! <><> = \approx
%! ==> = \Longrightarrow
%! <== = \Longleftarrow
%! <=> = \Longleftrightarrow
%! <-> = \leftrightarrow
%! ~> = \leadsto
%! -o+ = \multimap^{+}
%! -o* = \multimap^{\times}
%! -o = \multimap
%! ::= = &::=&
%! /= = \neq
%! @@ = \mu
%! [^ = \lceil
%! ^] = \rceil
%! forall = \forall
%! exists = \exists
%! empty = \epsilon
%! Pi = \Pi
%! Pi0 = \Pi^{o}
%! PiEE* = \Pi^{\eta\epsilon}_{*}
%! PiEE+ = \Pi^{\eta\epsilon}_{+}
%! PiEE = \Pi^{\eta\epsilon}
%! CatSet = \textbf{Set}
%! theseus = Theseus
%! sqrt(x) = \sqrt{#x}
%! surd(p,x) = \sqrt[#p]{#x}
%! inv(x) = \frac{1}{#x}
%! frac(x,y) = \frac{#x}{#y}
%! * = \times

%%%%%%%%%%%%%%%%%%%%%%%%%%%%%%%%%%%%%%%%%%%%%%%%%%%%%%%%%%%%%%%%%%%%%%%%%%%%%%%%
\begin{document}

\conferenceinfo{POPL'13}{}
\CopyrightYear{}
\copyrightdata{}
\titlebanner{}
\preprintfooter{}

\title{Fractional Types} 

\authorinfo{*}{*}{*}
\maketitle

\begin{abstract}

\end{abstract}

\category{D.3.1}{Formal Definitions and Theory}{}
\category{F.3.2}{Semantics of Programming Languages}{}
\category{F.3.3}{Studies of Program Constructs}{Type structure}

\terms
Languages, Theory

\keywords continuations, information flow, linear logic, logic programming,
quantum computing, reversible logic, symmetric monoidal categories, compact
closed categories.

%%%%%%%%%%%%%%%%%%%%%%%%%%%%%%%%%%%%%%%%%%%%%%%%%%%%%%%%%%%%%%%%%%%%%%%%%%%%%%%%
\section{The Game}

We start with a type {{A}} and build terms of the types {{A}}, {{Id_A}},
{{Id_{Id_a} }}, etc.

%%%%%%%%%%%%%%%%%%%%%%%%%%%%%%%%%%%%%%%%%%%%%%%%%%%%%%%%%%%%%%%%%%%%%%%%%%%%%%%%
\section{Level of Atoms}

We start with the kind {{type}} and terms of kind {{type}}:

%subcode{bnf} include main
% kinds, k ::= type
% value types, b ::= 0 | 1 | b + b | b * b 

%subcode{proof} include main
%@  ~
%@@ 0 : type
%
%@  ~
%@@ 1 : type
%
%@  b1 : type
%@  b2 : type
%@@ b1 + b2 : type
%
%@  b1 : type
%@  b2 : type
%@@ b1 * b2 : type

%%%%%%%%%%%%%%%%%%%%%%%%%%%%%%%%%%%%%%%%%%%%%%%%%%%%%%%%%%%%%%%%%%%%%%%%%%%%%%%%
\section{ {{Pi}} Combinators}

Now we introduce isomorphisms between terms of kind {{type}}. This is where
the {{Pi}} combinators appear.

%subcode{bnf} include main
% kinds, k ::= type
% value types, b ::= 0 | 1 | b + b | b * b 
% 
% identity types, t0 ::= Id_k(b1,b2)
% iso ::= zeroe | zeroi 
%     &|& swap+ | assocl+ | assocr+ 
%     &|& identl* | identr* 
%     &|& swap* | assocl* | assocr* 
%     &|& dist | factor 
% comb., c ::= iso | id | sym c | c (;) c | c (+) c | c (*) c 

%subcode{proof} include main
%@  b : k
%@@ zeroe : Id_k(0+b,b)
%
%@  b : k
%@@ zeroi : Id_k(b,0+b)
%---
%@  b1 : k
%@  b2 : k
%@@ swap+ : Id_k(b1+b2,b2+b1)
%---
%@  b1 : k
%@  b2 : k
%@  b3 : k
%@@ assocl+ : Id_k(b1+(b2+b3),(b1+b2)+b3)
%---
%@  b1 : k
%@  b2 : k
%@  b3 : k
%@@ assocr+ : Id_k((b1+b2)+b3,b1+(b2+b3))
%---
%@  b : k
%@@ unite : Id_k(1 * b, b)
%
%@  b : k
%@@ uniti : Id_k(b, 1 * b)
%---
%@  b1 : k
%@  b2 : k
%@@ swap* : Id_k(b1 * b2,b2 * b1)
%---
%@  b1 : k
%@  b2 : k
%@  b3 : k
%@@ assocl* : Id_k(b1*(b2*b3),(b1*b2)*b3)
%---
%@  b1 : k
%@  b2 : k
%@  b3 : k
%@@ assocr* : Id_k((b1*b2)*b3,b1*(b2*b3))
%---
%@  b1 : k
%@  b2 : k
%@  b3 : k
%@@ distrib : Id_k((b1+b2)*b3,(b1*b3)+(b2*b3))
%---
%@  b1 : k
%@  b2 : k
%@  b3 : k
%@@ factor : Id_k((b1*b3)+(b2*b3),(b1+b2)*b3)
%---
%@ b : k
%@@ id : Id_k(b,b)
%
%@ c : Id_k(b1,b2)
%@@ sym c : Id_k(b2,b1)
%---
%@ c1 : Id_k(b1,b2)
%@ c2 : Id_k(b2,b3)
%@@ c1(;)c2 : Id_k(b1,b3)
%---
%@ c1 : Id_k(b1,b3)
%@ c2 : Id_k(b2,b4)
%@@ c1(+)c2 : Id_k(b1+b2,b3+b4)
%---
%@ c1 : Id_k(b1,b3)
%@ c2 : Id_k(b2,b4)
%@@ c1(*)c2 : Id_k(b1*b2,b3*b4)

We will give the semantics in an usual style to simplify the presentation of
next section. First we define values and clauses:

%subcode{bnf} include main
% values, v ::= () | left v | right v | (v,v)
% clauses, m ::= (1/v,v) 

%subcode{proof} include main
%@  ~
%@@ () : 1
%
%@  v : b1
%@@ left v : b1 + b2
%
%@  v : b2
%@@ right v : b1 + b2
%---
%@  v1 : b1
%@  v2 : b2
%@@ (v1,v2) : b1 * b2

The intuition behind clauses is the following: if applying 
{{c : Id_k(b1,b2)}} to {{v1 : b1}} yields {{v2 : b2}}, we say that 
the clause {{ (1/v1, v2) }} has type {{c}}. 

%subcode{proof} include main
%@ zeroe : Id_k(0+b,b)
%@ v : b
%@@ (1/right v, v) : zeroe
% 
%@ zeroi : Id_k(b,0+b)
%@ v : b
%@@ (1/v, right v) : zeroi
%---
%@ swap+ : Id_k(b1+b2,b2+b1)
%@ v : b1
%@@ (1/left v, right v) : swap+
%---
%@ swap+ : Id_k(b1+b2,b2+b1)
%@ v : b2
%@@ (1/right v, left v) : swap+
%---
%@ assocl+ : Id_k(b1+(b2+b3),(b1+b2)+b3)
%@ v : b1
%@@ (1/ left v, left (left v)) : assocl+
%---
%@ assocl+ : Id_k(b1+(b2+b3),(b1+b2)+b3)
%@ v : b2
%@@ (1/ right (left v), left (right v)) : assocl+
%---
%@ assocl+ : Id_k(b1+(b2+b3),(b1+b2)+b3)
%@ v : b3
%@@ (1/ right (right v), right v) : assocl+
%---
%@ assocr+ : Id_k((b1+b2)+b3,b1+(b2+b3))
%@ v : b1
%@@ (1/ left (left v), left v) : assocr+
%---
%@ assocr+ : Id_k((b1+b2)+b3,b1+(b2+b3))
%@ v : b2
%@@ (1/ left (right v), right (left v)) : assocr+
%---
%@ assocr+ : Id_k((b1+b2)+b3,b1+(b2+b3))
%@ v : b3
%@@ (1/ right v, right (right v)) : assocr+
%---
%@ unite : Id_k(1 * b, b)
%@ v : b
%@@ (1/((),v), v) : unite
% 
%@ uniti : Id_k(b, 1 * b)
%@ v : b
%@@ (1/v, ((),v)) : uniti
%---
%@ swap* : Id_k(b1 * b2,b2 * b1)
%@ v1 : b1
%@ v2 : b2
%@@ (1/(v1,v2), (v2,v1)) : swap*
%---
%@ assocl* : Id_k(b1*(b2*b3),(b1*b2)*b3)
%@ v1 : b1
%@ v2 : b2
%@ v3 : b3
%@@ (1/(v1,(v2,v3)), ((v1,v2),v3)) : assocl*
%---
%@ assocr* : Id_k((b1*b2)*b3,b1*(b2*b3))
%@ v1 : b1
%@ v2 : b2
%@ v3 : b3
%@@ (1/((v1,v2),v3), (v1,(v2,v3))) : assocr*
%---
%@ distrib : Id_k((b1+b2)*b3,(b1*b3)+(b2*b3))
%@ v1 : b1
%@ v3 : b3
%@@ (1/(left v1,v3), left (v1,v3)) : distrib
%---
%@ distrib : Id_k((b1+b2)*b3,(b1*b3)+(b2*b3))
%@ v2 : b2
%@ v3 : b3
%@@ (1/(right v2,v3), right (v2,v3)) : distrib
%---
%@ factor : Id_k((b1*b3)+(b2*b3),(b1+b2)*b3)
%@ v1 : b1
%@ v3 : b3
%@@ (1/(left (v1,v3)), (left v1,v3)) : factor
%---
%@ factor : Id_k((b1*b3)+(b2*b3),(b1+b2)*b3)
%@ v2 : b2
%@ v3 : b3
%@@ (1/(right (v2,v3)), (right v2,v3)) : factor
%---
%@ id : Id_k(b,b)
%@ v : b
%@@ (1/v,v) : id
%
%@ c : Id_k(b2,b1)
%@ (1/v2, v1) : c
%@@ (1/v1, v2) : sym c
%---
%@ c1 : Id_k(b1,b2)
%@ c2 : Id_k(b2,b3)
%@ (1/v1, v2) : c1
%@ (1/v2, v3) : c2
%@@ (1/v1, v3) : c1 (;) c2
%---
%@ c1 : Id_k(b1,b3)
%@ (1/v1, v3) :: c1
%@@ (1/left v1, left v3) : c1 (+) c2
%
%@ c2 : Id_k(b2,b4)
%@ (1/v2, v4) :: c2
%@@ (1/right v2, right v4) : c1 (+) c2
%---
%@ c1 : Id_k(b1,b3)
%@ c2 : Id_k(b2,b4)
%@ (1/v1, v3) :: c1
%@ (1/v2, v4) :: c2
%@@ (1/(v1,v2), (v3,v4)) : c1 (*) c2

%%%%%%%%%%%%%%%%%%%%%%%%%%%%%%%%%%%%%%%%%%%%%%%%%%%%%%%%%%%%%%%%%%%%%%%%%%%%%%%%
\section{ {{Pi}} semantics I}

Now we introduce isomoprhisms between terms of type {{Id_k(b1,b2)}}. This is
where we start reasoning about equivalence of the {{Pi}}-combinators
themselves. For {{c1 : Id_k(b1,b2)}} and {{c2 : Id_k(b1,b2)}}, terms of type
{{Id_{Id_k(b1,b2)}(c1,c2)}} would witness that {{c1}} and {{c2}} are
isomorphic (up to equivalence at the next level). 

For example, given the combinator {{uniti}} at type {{bool = 1+1}}:

%subcode{proof} include main
%@ uniti : Id_k(1+1, 1 * (1+1))
%@ v : 1+1
%@@ (1/v, ((),v)) : uniti

when viewed as a type, the elements of {{uniti}} are:
{{ (1/left (), ((), left ())) }}
and 
{{ (1/right (), ((), right ())) }}

Using our isomorphisms we can apply {{ id (*) unite }} to the above to
produce
{{ (1/left (), left ()) }}
and 
{{ (1/right (), right ()) }}.

In other words, there is an isomorphism {{alpha = id (*) unite}} beween
{{uniti}} and {{id}} at type {{bool}}. We now need to generalize this idea to
the full language. We possibly (probably) need to add the meadow rules so
that we can convert a clause to a regular value, use the isomorphisms on it,
and then translate it back to a clause.

%subcode{bnf} include main
% kinds, k ::= type
% value types, b ::= 0 | 1 | b + b | b * b 
%
% identity types, t0 ::= Id_k(b1,b2) 
% iso ::= zeroe | zeroi 
%     &|& swap+ | assocl+ | assocr+ 
%     &|& identl* | identr* 
%     &|& swap* | assocl* | assocr* 
%     &|& dist | factor 
% comb., c ::= iso | id | sym c | c (;) c | c (+) c | c (*) c 
%
% identity types, t1 ::= Id_{t0}(c1,c2)
% alpha ::= ...

%%%%%%%%%%%%%%%%%%%%%%%%%%%%%%%%%%%%%%%%%%%%%%%%%%%%%%%%%%%%%%%%%%%%%%%%

\end{document}

%%%%%%%%%%%%%%%%%%%%%%%%%%%%%%%%%%%%%%%%%%%%%%%%%%%%%%%%%%%%%%%%%%%%%%%%
