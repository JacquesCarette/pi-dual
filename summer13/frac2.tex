\documentclass{sigplanconf}
\usepackage{url}
\usepackage{proof}
\usepackage{amssymb}
\usepackage{stmaryrd}
\usepackage{listings}
\usepackage{graphicx}
\usepackage{comment}

\newcommand{\dgm}[2][1.5]{
\begin{center}
\scalebox{#1}{
\includegraphics{diagrams/#2.pdf}
}
\end{center}
}
\newcommand{\todo}[1]{\textbf{TODO:} #1}
\newcommand{\jacques}[1]{\textsc{Jacques says:} #1}
\newcommand{\amr}[1]{\textsc{Amr says:} #1}

\newtheorem{theorem}{Theorem}[section]
\newtheorem{lemma}[theorem]{Lemma}
\newtheorem{definition}[theorem]{Definition}
\newtheorem{proposition}[theorem]{Proposition}

%subcode-inline{bnf-inline} name Pi
%! swap+ = \mathit{swap}^+
%! swap* = \mathit{swap}^*
%! dagger =  ^{\dagger}
%! assocl+ = \mathit{assocl}^+
%! assocr+ = \mathit{assocr}^+
%! assocl* = \mathit{assocl}^*
%! assocr* = \mathit{assocr}^*
%! identr* = \mathit{uniti}
%! identl* = \mathit{unite}
%! dist = \mathit{distrib}
%! factor = \mathit{factor}
%! (o) = \fatsemi
%! (;) = \fatsemi
%! (*) = \times
%! (+) = +
%! foldB = fold_B
%! unfoldB = unfold_B
%! foldN = fold_N
%! unfoldN = unfold_N
%! trace+ = \mathit{trace}^{+}
%! trace* = \mathit{trace}^{\times}
%! :-* = \multimap
%! :-+ = \multimap^{+}
%! emptyset = \emptyset

%subcode-inline{bnf-inline} regex \{\{(((\}[^\}])|[^\}])*)\}\} name main include Pi
%! [^ = \ulcorner
%! ^] = \urcorner
%! [v = \llcorner
%! v] = \lrcorner
%! [[ = \llbracket
%! ]] = \rrbracket
%! ^^^ = ^{\dagger}
%! eta* = \eta
%! eps* = \epsilon
%! Union = \bigcup
%! in = \in
%! |-->* = \mapsto^{*}
%! |-->> = \mapsto_{\ggg}
%! |-->let = \mapsto_{let}
%! |--> = \mapsto
%! <--| = \mapsfrom
%! |- = \vdash
%! <=> = \Longleftrightarrow
%! <-> = \leftrightarrow
%! -> = \rightarrow
%! ~> = \leadsto
%! ::= = ::=
%! amp = \&
%! /= = \neq
%! vi = v_i
%! di = d_i
%! si = s_i
%! sj = s_j
%! F = \texttt{F}
%! T = \texttt{T}
%! forall = \forall
%! exists = \exists
%! empty = \emptyset
%! Sigma = \Sigma
%! eta = \eta
%! where = \textbf{where}
%! epsilon = \varepsilon
%! least = \phi
%! loop+ = loop_{+}
%! loop* = loop_{\times}
%! CatC = {\mathcal C}
%! CatA = {\mathcal A}
%! gamma = \gamma
%! {[ = \{
%! ]} = \}
%! elem = \in
%! dagger = ^\dagger
%! alpha = \alpha
%! beta = \beta
%! rho = \rho
%! @@ = \mu
%! @ = \,@\,
%! Pow = \mathcal{P}
%! Pi = \Pi
%! PiT = \Pi^{o}
%! PiEE = \Pi^{\eta\epsilon}_{+}
%! PiT = \Pi^{o}
%! PiTF = \Pi^{/}
%! bullet = \bullet
%! * = \times

\begin{document}
\conferenceinfo{POPL '14}{date, City.} 
\copyrightyear{2014}
\copyrightdata{[to be supplied]} 

\title{\ldots}
\authorinfo{\ldots}
           {\ldots}
           {\ldots}

\maketitle

\begin{abstract}
\ldots
\end{abstract}

%%%%%%%%%%%%%%%%%%%%%%%%%%%%%%%%%%%%%%%%%%%%%%%%%%%%%%%%%%%%%%%%%%%%%%%%%%%%%
\section{Background: {{Pi}} }
\label{sec:pi}

We review our language {{Pi}} providing the necessary background and
context for our higher-order extension.\footnote{The presentation in this
  section focuses on the simplest version of {{Pi}}. Other versions
  include the empty type, recursive types, and trace operators but these
  extensions are orthogonal to the higher-order extension emphasized in this
  paper.} The terms of {{Pi}} are not classical values and functions;
rather, the terms are isomorphism witnesses.  In other words, the terms of
{{Pi}} are proofs that certain ``shapes of values'' are isomorphic.
And, in classical Curry-Howard fashion, our operational semantics shows how
these proofs can be directly interpreted as actions on ordinary values which
effect this shape transformation. Of course, ``shapes of values'' are very
familiar already: they are usually called \emph{types}.  But frequently one
designs a type system as a method of classifying terms, with the eventual
purpose to show that certain properties of well-typed terms hold, such as
safety.  Our approach is different: we start from a type system, and then
present a term language which naturally inhabits these types, along with an
appropriate operational semantics.

\paragraph*{Data.}
We view {{Pi}} as having two levels:  it has traditional values, given by:
%subcode{bnf} include main
% values, v ::= () | left v | right v | (v, v)

\noindent and these are classified by ordinary types:
%subcode{bnf} include main
% value types, b ::= 1 | b + b | b * b 

\noindent 
Types include the unit type {{1}}, sum types {{b1+b2}}, and product types
{{b1*b2}}.  Values include {{()}} which is the only value of type {{1}},
{{left v}} and {{right v}} which inject~{{v}} into a sum type, and
{{(v1,v2)}} which builds a value of product type.  These values should be
regarded as largely ancillary: we do not treat them as first-class citizens,
and they only occur when observing the effect of an isomorphism.

\paragraph*{Isomorphisms.} The important terms of {{Pi}} are witnesses to
(value) type isomorphisms.  They are also typed, by the shape of the (value)
type isomorphism they witness {{b <-> b}}.  Specifically, they are witnesses
to the following isomorphisms:
%subcode{bnf} include main
%! columnStyle = r@{\hspace{-0.5pt}}r c l@{\hspace{-0.5pt}}l
%swap+ :&  b1 + b2 & <-> & b2 + b1 &: swap+
%assocl+ :&  b1 + (b2 + b3) & <-> & (b1 + b2) + b3 &: assocr+
%identl* :&  1 * b & <-> & b &: identr*
%swap* :&  b1 * b2 & <-> & b2 * b1 &: swap*
%assocl* :&  b1 * (b2 * b3) & <-> & (b1 * b2) * b3 &: assocr*
%dist :&~ (b1 + b2) * b3 & <-> & (b1 * b3) + (b2 * b3)~ &: factor

\noindent Each line of the above table introduces a pair of dual
constants\footnote{where {{swap*}} and {{swap+}} are self-dual.} that witness
the type isomorphism in the middle.  These are the base (non-reducible) terms
of the second, principal level of {{Pi}}. Note how the above has two
readings: first as a set of typing relations for a set of constants. Second,
if these axioms are seen as universally quantified, orientable statements,
they also induce transformations of the (traditional) values. The
(categorical) intuition here is that these axioms have computational content
because they witness isomorphisms rather than merely stating an extensional
equality.

The isomorphisms are extended to form a congruence relation by adding the
following constructors that witness equivalence and compatible closure:

%subcode{proof} include main
%@  ~
%@@ id : b <-> b 
%
%@ c : b1 <-> b2
%@@ sym c : b2 <-> b1
%
%@ c1 : b1 <-> b2
%@ c2 : b2 <-> b3
%@@ c1(;)c2 : b1 <-> b3
%---
%@ c1 : b1 <-> b3
%@ c2 : b2 <-> b4
%@@ c1 (+) c2 : b1 + b2 <-> b3 + b4
%
%@ c1 : b1 <-> b3
%@ c2 : b2 <-> b4
%@@ c1 (*) c2 : b1 * b2 <-> b3 * b4
\noindent The syntax is overloaded: we use the same symbol at the value-type level
and at the isomorphism-type level for denoting sums and products.  Hopefully
this will not cause undue confusion.

It is important to note that ``values'' and ``isomorphisms'' are completely
separate syntactic categories which do not intermix. The semantics of the
language come when these are made to interact at the ``top level'' via
\emph{application}: 
%subcode{bnf} include main
% top level term, l ::= c v

\noindent
To summarize, the syntax of {{Pi}} is given as follows. 

\begin{definition}{(Syntax of {{Pi}})}
\label{def:Pi}
%subcode{bnf} include main
% value types, b ::= 1 | b+b | b*b 
% values, v ::= () | left v | right v | (v,v) 
%
% iso.~types, t ::= b <-> b
% base iso ::= swap+ | assocl+ | assocr+ 
%     &|& unite | uniti | swap* | assocl* | assocr* 
%     &|& dist | factor 
% iso comb., c ::= iso | id | sym c | c (;) c | c (+) c | c (*) c 
% top level term, l ::= c v
\end{definition}

The language presented above, at the type level, models a commutative ringoid
where the multiplicative structure forms a commutative monoid, but the
additive structure is just a commutative semigroup.  Note that the version of
{{Pi}} that includes the empty type with its usual laws exactly captures, at
the type level, the notion of a \emph{semiring} (occasionally called a
\emph{rig}) where we replace equality by isomorphism.  Semantically, {{Pi}}
models a \emph{bimonoidal category} whose simplest example is the category of
finite sets and bijections. In that interpretation, each value type denotes a
finite set of a size calculated by viewing the types as natural numbers and
each combinator {{c : b1 <-> b2}} denotes a bijection between the sets
denoted by~{{b1}} and~{{b2}}. 

Operationally, the semantics of {{Pi}} is given using two mutually recursive
interpreters: one going forward and one going backwards. The use of {{sym}}
switches control from one evaluator to the other. We will present the
operational semantics in the next section along with the extension with
fractional types and values. For now, we state without proof that the
evaluation of well-typed combinators always terminates and that {{Pi}} is
logically reversible, i.e., that for all combinators {{c : b1 <-> b2}} and
values {{v1 : b1}} and {{v2 : b2}} we have the forward evaluation of {{c v1}}
produces {{v2}} iff the backwards evaluation of {{c v2}} produces {{v1}}.

As is usual with type theories, the coherence conditions (pentagonal and
triangle identity) are not explicit, but would correspond to certain identity
types being trivial.

\section{ {{Pi}} with trace}

In order to take a step toward being able to write the Int construction in
{{Pi}} to give it fractional types, we want to add a trace combinator to it,
with the following type:

%subcode{proof} include main
%@  c : b1 * b <-> b2 * b
%@@ trace c : b1 <-> b2

This kind of combinator allows us to cancel out a common factor. When the type
{{0}} is removed from the language, it is always possible to do this, but adding
computational content becomes tricky. For example, consider permutations on the
finite set of size {{6}}. If we wish to trace one of these to get a permutation
on the set of size {{3}}, we are able to, but which set of size {{3}}? There
are several different ways to do this, but there is no obvious canonical answer.

More concretely, suppose we have a combinator {{c : 3 * 2 <-> 3 * 2}}. The
combinator {{trace c : 3 <-> 3}} can be constructed in two obvious ways: let
{{(trace c) v}} run {{c}} on {{(v, true)}} or on {{(v, false)}}. It is possible
to choose {{c}} so that depending on which default value of type {{2}} is used,
a different combinator results.

To fix this, we parameterize {{trace}} with a value {{v : b}}, as follows:

%subcode{proof} include main
%@  c : b1 * b <-> b2 * b
%@  v : b
%@@ trace_v c : b1 <-> b2

It remains to show that this is ``well-behaved''; that is, we want it to obey
some (ideally all) of the trace properties, or at the very least some subset of
them that is sufficient to prove some necessary properties for {{Pi}} extended
with fractional types, as presented in the next few sections.


%%%%%%%%%%%%%%%%%%%%%%%%%%%%%%%%%%%%%%%%%%%%%%%%%%%%%%%%%%%%%%%%%%%%%%%%%%%%%
\section{Syntax of Fractionals}

The types of {{Pi}} correspond to the natural numbers. Our goal is to have a
type system that corresponds to the (positive) rational numbers. 

\begin{definition}{(Syntax of {{PiTF}})}
\label{def:PiTF}
%subcode{bnf} include main
% value types, r ::= b1 / b2
% values, f ::= (v1 |--> v2, c)
%
% iso.~types, t ::= r <=> r
% base iso ::= swap+ | assocl+ | assocr+ 
%     &|& unite | uniti | swap* | assocl* | assocr* 
%     &|& dist | factor 
%     &|& eta | epsilon
% iso comb., c ::= iso | id | sym c | c (;) c | c (+) c | c (*) c 
% top level term, l ::= c v
\end{definition}

If we restrict our attention to types {{b / 1}} and values {{ () |--> v amp id}},
we recover {{Pi}}. 

In algebra, the rational number {{a/b}} denotes 
$\{ (a','b) ~|~ a * b' = a' * b \}$. We essentially use the same denotation but
add computational content, i.e., the type {{a/b}} denotes
$\{ (a',b',c) ~|~ c : a * b' \leftrightarrow a' * b \}$ where $c$ is a {{Pi}}
combinator witnessing the equivalence of {{a * b'}} and {{a' * b}}. 

%%%%%%%%%%%%%%%%%%%%%%%%%%%%%%%%%%%%%%%%%%%%%%%%%%%%%%%%%%%%%%%%%%%%%%%%%%%%%
\section{Higher-Order Reversible Functions}

To get from {{Pi}}~\cite{James:2012:IE:2103656.2103667} types and combinators
to a higher order setting, we hope to add \emph{fractional types} to the
language. Recall that to construct the (positive) rational numbers {{Q}}, we
can take pairs of positive integers {{(m, n)}} and quotient by the
equivalence relation that says {{(m1, n1)}} is equal to {{(m2, n2)}} if 
{{m1 * n2 = m2 * n1}}. We can write this as an inference rule as follows:

%subcode{proof} include main
%@ m1 * n2 = m2 * n1
%@@ (m1, n1) = (m2, n2)

\noindent
Adding in computational content, pi-style, suggests that we want something like:

%subcode{proof} include main
%@ c : A1 * B2 <-> B1 * A2
%@@ {[c]} : (B1 |--> A1) <=> (B2 |--> A2)

\noindent
However, this does not quite work. To see why, recall that we want to treat the
pair {{(m, n)}} (written here as {{A |--> B}}) as an internalization of the
reversible function space. Thus, the type {{(B1 |--> A1) <=> (B2 |--> A2)}}
should be a transformation of a {{Pi}} combinator of type {{B1 <-> A1}} into a
{{Pi}} combinator of type {{B2 <-> A2}}. In order to do this, we need a trace
combinator (TODO: draw a diagram that demonstrates this) with the type

%subcode{proof} include main
%@ c : A * C <-> B * C
%@@ trace c : A <-> B

\noindent
This is an issue, since prior attempts at defining an adequate trace operator
failed: we got a language that could express all relations, but was no longer
reversible in a meaningful sense.

%%%%%%%%%%%%%%%%%%%
\subsection{Defining trace}

The trace operator had two problems: it allowed arbitrary function definition,
and it would not work in a language that contained the empty type {{0}}. The
former problem arose due to the logic programming-style interpretation we
assigned to trace: it would create *all* possible values of type {{C}}, then run
the given combinator on them and throw away the branches that did not return the
same {{C}} value.

We can solve this problem by parameterizing the trace operator with a value {{v
: C}}. This results in the following new typing rule for trace:

%subcode{proof} include main
%@ c : A * C <-> B * C
%@ v : C
%@@ trace_v c : A <-> B

\noindent
This ends up solving both problems: we give the trace a default value, and
prevent it from running on the empty type (since {{C}} now must be inhabited).
However, we still must assign dynamic semantics to trace. The na\"ive solution
of running {{c}} on the pair of the input value and {{v}} does not work, as this
is trivially irreversible---consider, for example, {{trace_v swap*}}, which
takes any value of the same type as {{v}} and returns {{v}}. Thus, the semantics
for {{trace}} must be slightly more complicated.

Recall that in {{Pi}}, combinators can be viewed as permutations on finite sets.
A combinator of type {{A * C <-> B * C}} can be seen as two separate
permutations: one on {{A}} (or equivalently, B, since they must have the same
size and thus the same denotation as a finite set) and one on {{C}}. If we
apply the fragment of the combinator that acts on {{C}} enough times, we will
eventually get out the same value that we started with\footnote{I want to prove
that this is actually true; having the extra {{A}} there seems like it might
complicate things.}. Thus, the semantics that we want is for {{(trace_v c) v'}}
to run {{c}} on the pair of {{v'}} and the last {{C}} value (starting with
{{v}}) until the {{C}} value matches {{v}}, at which point it returns the {{B}}
value returned by the combinator. In pseudocode:

\begin{verbatim}
TODO: figure out how to do the code 
environment in subcode

trace v c vin =
  v0 = v
  <vout, v'> = c <vin, v>

  while (v' != v)
    <vout, v'> := c <vin, v'>

  return vout
\end{verbatim}

A more concise formulation as an inference rule is as follows:

%subcode{proof} include main
%@ exists v0 : C. c <v_{in}, v> = <v_{out}, v>
%@@ (trace_v c) v_{in} = v_{out}

This is easily reversible: {{(trace_v c)dagger = trace_v (c dagger)}}
\footnote{Also need to prove this.}.

%%%%%%%%%%%%%%%%%%%
\subsection{Higher-Order Combinators}

Now that we have the trace combinator defined, we can proceed with defining
higher-order combinators. Note that this parameterizes our definition of
$\{c\}$ by {{v}}, the value of type {{A}} (or equivalently, type {{B}}) used
in the trace:

%subcode{proof} include main
%@ c : A1 * B2 <-> B1 * A2
%@@ {[c]}_v : (B1 |--> A1) <=> (B2 |--> A2)

\noindent
We want this to be a transformation on combinators, taking a combinator of type
{{B1 <-> A1}} to a combinator of type {{B2 <-> A2}}. This can be achieved by
letting the combinator {{{[c]}_v c'}} be {{trace_v (c ; c' * id)}}, where
{{c : B1 <-> A1}} and {{v : A1}}.

From here, we want to define bifunctors {{/}}, {{*}}, and {{+}}, to make sure
that our definition of functions is truly higher-order.

\begin{itemize}

\item {{(B |--> A) / (D |--> C) = (A * D) |--> (B * C)}}, with

{{{[c1]}_{v1} : ((B |--> A) <=> (D |--> C)) / {[c2]}_{v_2} : ((B' |--> A') <=> (D' |--> C')) = {[shuffle ; c1 * c2dagger ; shuffle]}_{<v1, c2 v2>}}}
%%: B1 * A2 |--> A1 * B2 <=> D1 * C2 |--> C1 * D2}},
where
{{shuffle : (A * B) * (C * D) <-> (A * C) * (B * D)}}.

\item {{(B |--> A) * (D |--> C) = (B * D) |--> (A * C)}}, with

\end{itemize}


%%%%%%%%%%%%%%%%%%%%%%%%%%%%%%%%%%%%%%%%%%%%%%%%%%%%%%%%%%%%%%%%%%%%%%%%%%%%%

\bibliographystyle{abbrvnat}
\bibliography{cites}
\end{document}
