\documentclass{llncs}
\usepackage{url}
\usepackage{proof}
\usepackage{amssymb}
\usepackage{stmaryrd}
\usepackage{listings}
\usepackage{graphicx}
\usepackage{comment}

\newcommand{\dgm}[2][1.5]{
\begin{center}
\scalebox{#1}{
\includegraphics{diagrams/#2.pdf}
}
\end{center}
}

\newcommand{\todo}[1]{\textbf{TODO:} #1}
\newcommand{\jacques}[1]{\textsc{Jacques says:} #1}
\newcommand{\amr}[1]{\textsc{Amr says:} #1}


\hyphenation{a-reas}

%subcode-inline{bnf-inline} name Pi
%! swap+ = \mathit{swap}^+
%! swap* = \mathit{swap}^*
%! dagger =  ^{\dagger}
%! assocl+ = \mathit{assocl}^+
%! assocr+ = \mathit{assocr}^+
%! assocl* = \mathit{assocl}^*
%! assocr* = \mathit{assocr}^*
%! identr* = \mathit{uniti}
%! identl* = \mathit{unite}
%! dist = \mathit{distrib}
%! factor = \mathit{factor}
%! (o) = \fatsemi
%! (;) = \fatsemi
%! (*) = \times
%! (+) = +
%! foldB = fold_B
%! unfoldB = unfold_B
%! foldN = fold_N
%! unfoldN = unfold_N
%! trace+ = \mathit{trace}^{+}
%! trace* = \mathit{trace}^{\times}
%! :-* = \multimap
%! :-+ = \multimap^{+}
%! emptyset = \emptyset

%subcode-inline{bnf-inline} regex \{\{(((\}[^\}])|[^\}])*)\}\} name main include Pi
%! [^ = \ulcorner
%! ^] = \urcorner
%! [v = \llcorner
%! v] = \lrcorner
%! [[ = \llbracket
%! ]] = \rrbracket
%! ^^^ = ^{\dagger}
%! eta* = \eta
%! eps* = \epsilon
%! Union = \bigcup
%! in = \in
%! |-->* = \mapsto^{*}
%! |-->> = \mapsto_{\ggg}
%! |-->let = \mapsto_{let}
%! |--> = \mapsto
%! <--| = \mapsfrom
%! |- = \vdash
%! <=> = \Longleftrightarrow
%! <-> = \leftrightarrow
%! -> = \rightarrow
%! ~> = \leadsto
%! ::= = ::=
%! /= = \neq
%! vi = v_i
%! di = d_i
%! si = s_i
%! sj = s_j
%! F = \texttt{F}
%! T = \texttt{T}
%! forall = \forall
%! exists = \exists
%! empty = \emptyset
%! Sigma = \Sigma
%! eta = \eta
%! where = \textbf{where}
%! epsilon = \varepsilon
%! least = \phi
%! loop+ = loop_{+}
%! loop* = loop_{\times}
%! CatC = {\mathcal C}
%! CatA = {\mathcal A}
%! gamma = \gamma
%! {[ = \{
%! ]} = \}
%! elem = \in
%! dagger = ^\dagger
%! alpha = \alpha
%! beta = \beta
%! rho = \rho
%! @@ = \mu
%! @ = \,@\,
%! Pow = \mathcal{P}
%! Pi = \Pi
%! PiT = \Pi^{o}
%! PiEE = \Pi^{\eta\epsilon}_{+}
%! PiT = \Pi^{o}
%! PiTF = \Pi^{/}
%! bullet = \bullet
%! * = \times

%%%%%%%%%%%%%%%%%%%%%%%%%%%%%%%%%%%%%%%%%%%%%%%%%%%%%%%%%%%%%%%%%%%%%%%%%%%%%

\begin{document}

\section{Higher-Order Reversible Functions}

To get from {{Pi}} types and combinators to a higher order setting, we hope to
add \emph{fractional types} to the language. Recall that to construct the
(positive) rational numbers {{Q}}, we can take pairs of positive integers
{{(m, n)}} and quotient by the equivalence relation that says {{(m1, n1)}} is
equal to {{(m2, n2)}} if {{m1 * n2 = m2 * n1}}. We can write this as an
inference rule as follows:

%subcode{proof} include main
%@ m1 * n2 = m2 * n1
%@@ (m1, n1) = (m2, n2)

\noindent
Adding in computational content, pi-style, suggests that we want something like:

%subcode{proof} include main
%@ c : A1 * B2 <-> B1 * A2
%@@ \hat{c} : (B1 |--> A1) <=> (B2 |--> A2)

\noindent
However, this does not quite work. To see why, recall that we want to treat the
pair {{(m, n)}} (written here as {{A |--> B}}) as an internalization of the
reversible function space. Thus, the type {{(B1 |--> A1) <=> (B2 |--> A2)}}
should be a transformation of a {{Pi}} combinator of type {{B1 <-> A1}} into a
{{Pi}} combinator of type {{B2 <-> A2}}. In order to do this, we need a trace
combinator (TODO: draw a diagram that demonstrates this) with the type

%subcode{proof} include main
%@ c : A * C <-> B * C
%@@ trace c : A <-> B

\noindent
This is an issue, since prior attempts at defining an adequate trace operator
failed: we got a language that could express all relations, but was no longer
reversible in a meaningful sense.

\subsection{Defining trace}

The trace operator had two problems: it allowed arbitrary function definition,
and it would not work in a language that contained the empty type {{0}}. The
former problem arose due to the logic programming-style interpretation we
assigned to trace: it would create *all* possible values of type {{C}}, then run
the given combinator on them and throw away the branches that did not return the
same {{C}} value.

We can solve this problem by parameterizing the trace operator with a value {{v
: C}}. This results in the following new typing rule for trace:

%subcode{proof} include main
%@ c : A * C <-> B * C
%@ v : C
%@@ trace_v c : A <-> B

\noindent
This ends up solving both problems: we give the trace a default value, and
prevent it from running on the empty type (since {{C}} now must be inhabited).
However, we still must assign dynamic semantics to trace. The na\"ive solution
of running {{c}} on the pair of the input value and {{v}} does not work, as this
is trivially irreversible---consider, for example, {{trace_v swap*}}, which
takes any value of the same type as {{v}} and returns {{v}}. Thus, the semantics
for {{trace}} must be slightly more complicated.

Recall that in {{Pi}}, combinators can be viewed as permutations on finite sets.
A combinator of type {{A * C <-> B * C}} can be seen as two separate
permutations: one on {{A}} (or equivalently, B, since they must have the same
size and thus the same denotation as a finite set) and one on {{C}}. If we
apply the fragment of the combinator that acts on {{C}} enough times, we will
eventually get out the same value that we started with\footnote{I want to prove
that this is actually true; having the extra {{A}} there seems like it might
complicate things.}. Thus, the semantics that we want is for {{(trace_v c) v'}}
to run {{c}} on the pair of {{v'}} and the last {{C}} value (starting with
{{v}}) until the {{C}} value matches {{v}}, at which point it returns the {{B}}
value returned by the combinator. In pseudocode:

\begin{verbatim}
TODO: figure out how to do the code environment in subcode

trace v c vin =
  v0 = v
  <vout, v'> = c <vin, v>

  while (v' != v)
    <vout, v'> := c <vin, v'>

  return vout
\end{verbatim}

A more concise formulation as an inference rule is as follows:

%subcode{proof} include main
%@ \exists v0 : C. c <v_{in}, v> = <v_{out}, v>
%@@ (trace_v c) v_{in} = v_{out}

This is easily reversible: {{(trace_v c)dagger = trace_v c
dagger}}\footnote{Also need to prove this.}.

\subsection{Higher-Order Combinators}

Now that we have the trace combinator defined, we can proceed with defining
higher-order combinators. Note that this parameterizes our definition of
{{\hat{c}}} by {{v}}, the value of type {{A}} (or equivalently, type {{B}}) used
in the trace:

%subcode{proof} include main
%@ c : A1 * B2 <-> B1 * A2
%@@ \hat{c}_v : (B1 |--> A1) <=> (B2 |--> A2)

\noindent
We want this to be a transformation on combinators, taking a combinator of type
{{B1 <-> A1}} to a combinator of type {{B2 <-> A2}}. This can be achieved by
letting the combinator {{\hat{c}_v c'}} be {{trace_v (c ; c' * id)}}, where {{c
: B1 <-> A1}} and {{v : A1}}.

From here, we want to define bifunctors {{/}}, {{*}}, and {{+}}, to make sure
that our definition of functions is truly higher-order.

\begin{itemize}

\item {{(B |--> A) / (D |--> C) = (A * D) |--> (B * C)}}, with {{{[c1]}_{v1} :
((B |--> A) <=> (D |--> C)) / {[c2]}_{v_2} : ((B' |--> A') <=> (D' |--> C')) =
{[shuffle ; c1 * c2 dagger ; shuffle]}_{<v1, c2 v2>} : B1 * A2 |--> A1 * B2 <=>
D1 * C2 |--> C1 * D2}}, where {{shuffle : (A * B) * (C * D) <-> (A * C) * (B *
D)}}.

\item {{(B |--> A) * (D |--> C) = (B * D) |--> (A * C)}}, with

\end{itemize}



\end{document}
