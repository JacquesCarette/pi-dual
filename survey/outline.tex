\documentclass[12pt]{article}
\usepackage{fullpage}
\title{Embracing the Laws of Physics: \\ Reversible Models of Computation}
\author{Jacques Carette, Roshan James, Amr Sabry}
\begin{document}
\maketitle 

%%%%%%%%%%%%%%%%%%%%%%%%%%%%%%%%%%%%%%%%%%%%%%%%%%%%%%%%%%%%%%%%%%%%%%%%%%%%%%
\section{Introduction}

Abstract models of computation: Turing machine, RAM, Von Neumann
architecture, $\lambda$-calculus, logic, discrete mathematics

Cyber-physical systems increasingly common. Are computing foundations
adequate? 

Toffoli 1980: Mathematical models of computation are abstract
constructions, by their nature unfettered by physical laws. However,
if these models are to give indications that are relevant to concrete
computing, they must somehow capture, albeit in a selective and
stylized way, certain general physical restrictions to which all
concrete computing processes are subjected.

%%%%%%%%%%%%%%%%%%%%%%%%%%%%%%%%%%%%%%%%%%%%%%%%%%%%%%%%%%%%%%%%%%%%%%%%%%%%%%
\section{Models of Computation}

Computation in the ``real world'' vs. idealized computation in
abstract models. 
\begin{itemize}
\item Mathematical abstraction of sorting is well-understood; sorting
  petabytes of data violates many of the assumptions: hardware
  component failure frequent, global synchronization impossible;
  consistency must be eventual; etc. 
\item Mathematical abstraction of sortedness property: easy; in the
  real world with petabytes of evolving data; it is not clear that it
  is even possible to check sortedness; easy to get inconsistent
  results
\item Noise does not exist in our mathematical abstractions of
  computation; essential use in differential privacy
\item Mathematical models of secure computing typically ignore power
  consumption, electromagnetic signatures; actual attacks based on DPA
  and DEMA however
\end{itemize}

%%%%%%%%%%%%%%%%%%%%%%%%%%%%%%%%%%%%%%%%%%%%%%%%%%%%%%%%%%%%%%%%%%%%%%%%%%%%%%
\section{Physics, Computation, and Logic}

Conflict between Church-Turing thesis, textbook quantum mechanics, and
RSA: Either Shor’s algorithm is not ``natural''. (Textbook quantum
mechanics is wrong); or, Shor’s algorithm is ``natural'' and there is
no classical counterpart. (There are ``natural'' computing models that
are exponentially faster than the Turing Machine); or, there is an
efficient classical factoring algorithm. (Possible but surprising); At
least one of these wild claims is true!!!

Conflict between quantum mechanics and logic. Possibilities:

\paragraph*{Revise quantum mechanics}.

The mathematician's vision of an unlimited sequence of totally
reliable operations is unlikely to be implementable in this real
universe.  But the real world is unlikely to supply us with unlimited
memory or unlimited Turing machine tapes. Therefore, continuum
mathematics is not executable, and physical laws which invoke that can
not really be satisfactory. They are references to illusionary
procedures.  (Landauer 1996 and 1999)

I want to talk about the possibility that there is to be an exact
simulation, that the computer will do exactly the same as nature. If
this is to be proved and the type of computer is as I've already
explained, then it's going to be necessary that everything that
happens in a finite volume of space and time would have to be exactly
analyzable with a finite number of logical operations. The present
theory of physics is not that way, apparently. It allows space to go
down into infinitesimal distances, wavelengths to get infinitely
great, terms to be summed in infinite order, and so forth; and
therefore, if this proposition is right, physical law is wrong.
(Feynman 1981)

\paragraph*{Revise Computer Science.}

Another thing that had been suggested early was that natural laws are
reversible, but that computer rules are not. But this turned out to be
false; the computer rules can be reversible, and it has been a very,
very useful thing to notice and to discover that. This is a place
where the relationship of physics and computation has turned itself
the other way and told us something about the possibilities of
computation. So this is an interesting subject because it tells us
something about computer rules…  (Feynman 1981)

Turing hoped that his abstracted-paper-tape model was so simple, so
transparent and well defined, that it would not depend on any
assumptions about physics that could conceivably be falsified, and
therefore that it could become the basis of an abstract theory of
computation that was independent of the underlying physics. ‘He
thought,’ as Feynman once put it, ‘that he understood paper.’ But he
was mistaken. Real, quantum-mechanical paper is wildly different from
the abstract stuff that the Turing machine uses. The Turing machine is
entirely classical, and does not allow for the possibility the paper
might have different symbols written on it in different universes, and
that those might interfere with one another.  (Deutsch 1985)

Ed Fredkin pursued the idea that information must be finite in
density. One day, he announced that things must be even more simple
than that. He said that he was going to assume that information itself
is conserved. ``You’re out of you mind, Ed.'' I pronounced. ``That’s
completely ridiculous. Nothing could happen in such a world. There
couldn’t even be logical gates. No decisions could ever be made.'' But
when Fredkin gets one of his ideas, he’s quite immune to objections
like that; indeed, they fuel him with energy. Soon he went on to
assume that information processing must also be reversible — and
invented what’s now called the Fredkin gate. (Minsky 1999)

Even revise the laws of thought themselves: In other terms, what is so
good in logic that quantum physics should obey? Can't we imagine that
our conceptions about logic are wrong, so wrong that they are unable
to cope with the quantum miracle? […] Instead of teaching logic to
nature, it is more reasonable to learn from her. Instead of
interpreting quantum into logic, we shall interpret logic into
quantum. (Girard 2007)

%%%%%%%%%%%%%%%%%%%%%%%%%%%%%%%%%%%%%%%%%%%%%%%%%%%%%%%%%%%%%%%%%%%%%%%%%%%%%%
\section{Conservation of Information}

A physical principle of computation

No creation of information
No duplication of information

All programs, proofs, deductions are equivalences, isomorphisms,
reversible

Several pages on reversible logic; reversible family of level 1 and
level 2 languages (Pi);

%%%%%%%%%%%%%%%%%%%%%%%%%%%%%%%%%%%%%%%%%%%%%%%%%%%%%%%%%%%%%%%%%%%%%%%%%%%%%%
\section{Applications}

Perhaps Information Flow Security

%%%%%%%%%%%%%%%%%%%%%%%%%%%%%%%%%%%%%%%%%%%%%%%%%%%%%%%%%%%%%%%%%%%%%%%%%%%%%%
\section{Conclusions}

%%%%%%%%%%%%%%%%%%%%%%%%%%%%%%%%%%%%%%%%%%%%%%%%%%%%%%%%%%%%%%%%%%%%%%%%%%%%%%
\end{document}