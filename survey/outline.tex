\documentclass{article}
\usepackage{fullpage}
\usepackage{url}
\title{Embracing the Laws of Physics: \\ Reversible Models of Computation}
\author{Jacques Carette, Roshan P. James, Amr Sabry}

\newcommand{\amr}[1]{\fbox{Amr says:} \textbf{#1}}
\newcommand{\jc}[1]{\fbox{Jacques says:} \textbf{#1}}
\newcommand{\roshan}[1]{\fbox{Roshan says:} \textbf{#1}}

\newcommand{\lcal}{\ensuremath{\lambda}-calculus}

\begin{document}
\maketitle

%%%%%%%%%%%%%%%%%%%%%%%%%%%%%%%%%%%%%%%%%%%%%%%%%%%%%%%%%%%%%%%%%%%%%%%%%%%%%%
\section{Introduction : Reversibility, the Missing Principle}


To fully appreciate the missing principle of conventional computing,
one must go back to an old thought experiment by J. C. Maxwell,
codified in a letter that Maxwell wrote to P. G Tait in 1867 -- this
is the letter whose ideas are now well known as the `Maxwell's
Demon'. Maxwell's Demon was a thought experiment that seemed to
indicate that intelligent beings can somehow violate the second law of
thermodynamics, thereby violating physics itself.

Many resolutions were offered for this conundrum (for a compilation,
see Maxwell's Demon by Leff and Rex), but none withstood careful
scrutiny until the establishment of the so called `Landauer's
Principle' in 1961 -- a principle whose experimental validation
happened recently in 2012 by Eric Lutz and others.

Maxwell's demon appears to violate the second law of thermodynamics by
having a tiny `intelligence' observing the movement of individual
particles of a gas and separating fast moving particles from slow
moving ones, thereby reducing the total entropy of the
system. Landauer's resolution of the demon relied on two ideas that
took root only a few decades earlier -- the formal notion of
computation (through the work of Turing and Church, 1936) and the
formal notion of information (through the work of Claude Shannon,
1948). Landauer reasoned that the computation done by the finite brain
of the demon, involves getting information about the movement of
molecules, acting on that information and then overwriting it to make
room for the next computation.  Landauer reasoned, and this is the
important bit, that the computation that is manipulating information
in the demon's brain \textit{must be thermodynamic work}, thereby
bringing the demon back into the fold of physics.

In its essence this is a strange and wonderful idea. Information,
physics and computation are inextricably linked and that ideas in each
field have implications for the other. (Cite Bennett and the various
`thermodynamics of computation' papers here.)

When the early models computation, the Turing machine, and the
$\lambda$-calculus, were developed there was no real reason to the
take the information content of computations into serious
consideration -- in fact, at that time there was no quantifiable
notion of information. These models followed in the footsteps of logic
where conventional logic gates such as AND and OR happily erase their
inputs.

\begin{quote}
Toffoli 1980: Mathematical models of computation are abstract
constructions, by their nature unfettered by physical laws. However,
if these models are to give indications that are relevant to concrete
computing, they must somehow capture, albeit in a selective and
stylized way, certain general physical restrictions to which all
concrete computing processes are subjected.
\end{quote}

Now however, through the establishment of the Landauer's principle and
through recent investigations into computational models that are well
suited to quantum mechanics it has become increasingly evident that
our computation models need to be revisited. Information, not just
seems to have a physical manifestation, but also that it, like other
physical entities, is conserved.

Going back to the drawing board, can we have computation that closely
matches our underlying physics?  i.e. can we have computation that in
its essence preserves information? What is the missing principle that
we can use as a basis of a new model of computation to achieve this?
It turns out that we need to look at physics again for inspiration --
in physics all of the primitive rules, nature itself at its most
fundamental level, is reversible. Conventional computation is not. Can
we embrace this simple principle as the building block of a model of
computation? What would computation look like when viewed from this
vantage point? And, does it have applications?

The objective of this survey is to touch upon the many related areas
of research in this space.

%% Abstract models of computation: Turing machine, RAM, Von Neumann
%% architecture, $\lambda$-calculus, logic, discrete mathematics

%% \jc{why are logic and DM listed alongside MoC?}

%% Cyber-physical systems increasingly common. Are computing foundations
%% adequate?
%% \jc{your questioning of MoC below is not about CPS! Should either
%% foreshadow the next section 'properly' here, or add a leading
%% example based on CPS to MoC section}


%%%%%%%%%%%%%%%%%%%%%%%%%%%%%%%%%%%%%%%%%%%%%%%%%%%%%%%%%%%%%%%%%%%%%%%%%%%%%%
\section{Models of Computation}

Explain classical models of computation and their limitations in
real-world ``physical'' applications.

Computation in the ``real world'' vs. idealized computation in
abstract models.
\begin{itemize}
\item Mathematical abstraction of sorting is well-understood; sorting
  petabytes of data violates many of the assumptions: hardware
  component failure frequent, global synchronization impossible;
  consistency must be eventual; etc.
\item Mathematical abstraction of sortedness property: easy; in the
  real world with petabytes of evolving data; it is not clear that it
  is even possible to check sortedness; easy to get inconsistent
  results
\item Noise does not exist in our mathematical abstractions of
  computation; essential use in differential privacy
\item Mathematical models of secure computing typically ignore power
  consumption, electromagnetic signatures; actual attacks based on DPA
  and DEMA however
\end{itemize}

% \jc{While all the above are true (and not exhaustive), what I also read
% is that the first two are actually ``the same'' point mostly related
% to scale and distributivity, the 3rd is about inadequate models (but not
% related to physics), while the 4th is about the physics of devices.
% They are not about CPS. And not about Quantum either. So how are these
% motivating a deeper look at how quantum affects our MoC? It feels to me
% that we ought to have a quantum ``example'' here already.}

% \jc{What this seems to motivate is to look at our models with respect
% to what we're trying to achieve.  Models are, by definition, abstractions
% that ``abstract away'' irrelevant details. So the point would be that
% some details that we thought were irrelevant really are not.
% So I could see this section motivating a whole programme of
% targeted de-abstraction. This could lead into the focus of this paper
% following one strand, namely one aspect of quantum physics.}

\begin{small}
\begin{quote}
``Turing hoped that his abstracted-paper-tape model was so simple, so
  transparent and well defined, that it would not depend on any
  assumptions about physics that could conceivably be falsified, and
  therefore that it could become the basis of an abstract theory of
  computation that was independent of the underlying physics.  `He
  thought,' as Feynman once put it, `that he understood paper.' But he
  was mistaken.''
\end{quote}
\hfill --- \emph{The Fabric of Reality}~\cite{deutsch}
\end{small}

%%%
\subsection{Software: Foundations and Applications}

We begin with a bold thesis: \textit{common computational models
  embody various physical assumptions that are plainly
  \emph{wrong}}. As we discuss in the remainder of this chapter, we
are not the first to make such a statement. We do, however, make two
novel contributions:
\begin{itemize}
\item we argue on pragmatic terms that our physically incorrect
  computational abstractions contribute significantly to the
  ``software crisis'' and that they inhibit advances in new emerging
  application domains; 
\item we identify the physical principle of \textbf{conservation of
  information} as a simple twist to our computational models that has,
  as we demonstrate in this book, far reaching theoretical and
  practical positive consequences to the science of computing.
\end{itemize}

To appreciate the first point, we examine various textbook abstractions and
contrast them with the actual (i.e., physical) computational situations to
which they correspond. Generally speaking, we find that textbook algorithms
differ from ``real programs'' in their assumptions about their environment of
operation. In textbook abstractions, the surrounding environment which
includes, for example, the data on which the algorithm operates is considered
static and well-defined. In the real world, this environment is dynamic and
noisy. 

\paragraph*{Sorting.} 
One might consider sorting to be a fairly simple ``solved'' problem. Indeed
textbook sorting algorithms have been formally specified, proved correct,
analyzed in the best case, worst case, average case, and are even taught in
many high school curricula. A beginner programmer can readily implement such
algorithms to sort ``real data'' --- or can he?

Under some assumptions, the answer is clearly yes. However, a moment's
reflection should convince us that the textbook algorithm is too idealized
and that sorting real data in the real world is still a challenge!  Consider
the following quote posted on the Google research blog in 2011:
\begin{quote}
  Almost three years ago we announced results of the first ever ``petasort''
  \ldots . It completed in just over six hours on 4000 computers. Recently we
  repeated the experiment using 8000 computers. The execution time was 33
  minutes, an order of magnitude improvement~\cite{petasort}.
\end{quote}
What happens at the petabyte scale that makes sorting such a
challenge? The answer is simply that ``the real world happens!'' In
the real world, many of the idealized mathematical assumptions
embodied in our computational models are invalidated. For example,
proper hardware functioning can no longer be taken for granted as
component failure becomes the norm, global synchronization becomes
impossible, determinism becomes expensive to maintain, speculative
execution becomes necessary, global consistency becomes impractical to
maintain, and eventual consistency is accepted; etc. There have been
many attempts to taken some of these issues into account in various
computational models but these attempts are arguably not going far
enough by not questioning the physical assumptions in our
computational models and re-thinking our very foundations.

\paragraph*{Binary Search.} 
In the idealized textbook setting, the binary search algorithm has the
precondition that the data is sorted. Even if we just have a few thousands
items, na\"\i vely checking sortedness (in linear time) defeats the purpose of
the efficient (logarithmic) binary search. If we have petabytes of
(changing!) data, is it even possible, in principle, to check sortedness? In
fact, in this dynamic situation, one is likely to get inconsistent
results. In other words, the very idea of pre-conditions and post-conditions
that can be checked statically appears to be at odds with computing in the
``real world.''

\paragraph*{Security.} 
Proving the security properties of an abstract mathematical model is great
(and necessary). This can be used to conclude that an isolated device is
secure but it is, however, not that relevant for applications based on RFID
devices which are targeted via DPA (Differential Power Analysis)~\cite{dpa}
and and DEMA (Differential Electromagnetic Analysis) attacks. There are
billions of such devices with applications like electronic passports. Power
consumption and electromagnetic properties are abstracted away in our
computational model.

\paragraph*{Privacy.} 
The Internet has created major concerns about privacy. Differential
privacy~\cite{dwork:differential} states that one can achieve privacy
by perturbing the output of a deterministic program by a suitable
amount of noise. The very idea of noise is abstracted away in our
computational and communication models and the amount of noise is a
non-existent abstraction in our typical computational models.

\paragraph*{A Crisis?}
Is this tension between theory and practice a crisis?  We argue:
yes. Computer science theory is founded on abstractions of the real
world. These abstractions are necessary to shield us from details of
technology, noise, errors, etc. and they have served us well until
recently.  It is, however, becoming more and more evident that
computer science theory is becoming less and less relevant to
practice, and hence it is important to revisit our abstractions from a
foundational point of view.

%%%
\subsection{Our Origins} 
 
Until Alan Turing's seminal work in 1936, computation was an activity
of the human mind; much like love and peace. While aspects of the idea
predated the twentieth century, it was Turing who established the idea
that computation has a formal interpretation and that all
computability can be captured within a formal system. Implicit in this
achievement however is the idea that abstract models of computation
are just that -- \emph{abstractions of computation realized in the
  physical world.}  In fact, one of the major achievements of Computer
Science has been the development of abstract models of computation
that shield the discipline from the underlying technology. As
effective as these models have been, one must note, however, that they
\emph{embody several implicit physical assumptions}.  As Tommaso
Toffoli explains in his influential 1980 paper:

{\begin{quote} Mathematical models of computation are abstract
  constructions, by their nature unfettered by physical laws. However,
  if these models are to give indications that are relevant to
  concrete computing, they must somehow capture, albeit in a selective
  and stylized way, certain general physical restrictions to which all
  concrete computing processes are subjected~\cite{Toffoli:1980}.
\end{quote}}

Our logic of programs and our hardware are based on Boolean Logic. We go back
to Boole's 1853 book entitled \emph{An Investigation of the Laws of Thought,
  on which are Founded the Mathematical Theories of Logic and
  Probabilities}. The opening sentence of Ch.~1 is:
\begin{quote}
  The design of the following treatise is to investigate the fundamental laws
  of those operations of the mind by which reasoning is performed; \ldots
\end{quote}
A few chapters later, we find:
\begin{quote}
  \textbf{Proposition IV.}  That axiom of metaphysicians which is termed the
  principle of contradiction, and which affirms that it is impossible for any
  being to possess a quality, and at the same time not to possess it, is a
  consequence of the fundamental law of thought, whose expression is $x^2 =
  x$.
\end{quote}
A detailed historical analysis of Boole's ideas are beyond the scope of this
book. The above quotes, however, should convey the idea that our hardware is
based on assumptions that Boole found as reasonable laws in 1853. 

Our theory of computation and complexity is based on the Turing Machine. We
go back to Turing's 1936 article entitled \emph{On Computable Numbers, with
  an Application to the Entscheidungsproblem,}. The opening sentence of
Sec. 1 is:
\begin{quote}
  We have said that the computable numbers are those whose decimals are
  calculable by finite means \ldots the justification lies in the fact that
  the human memory is necessarily limited.
\end{quote}
In Sec. 9, we find:
\begin{quote}
I think it is reasonable to suppose that they can only be squares
whose distance from the closest of the immediately previously observed
squares does not exceed a certain fixed amount.
\end{quote}
Again, a detailed historical analysis of Turing's ideas are beyond the scope
of this book. The above quotes should again convey the idea that our theories
of computation and complexity are based on some physical assumptions that
Turing found reasonable in 1936. When translated to physical language, the
assumptions embody what is known as the ``Bekenstein bound:'' which is an
upper limit on the amount of information that can be contained within a given
finite region of space.

One can examine more and more cases but the general story should be
clear. Our abstractions were made up based on a certain understanding of the
laws of physics. And since, as we argue, our abstractions are becoming at
odds with practice, it is time to revisit these abstractions. In the words of
Girard:
\begin{quote}
  In other terms, what is so good in logic that quantum physics should obey?
  Can't we imagine that our conceptions about logic are wrong, so wrong that
  they are unable to cope with the quantum miracle? 
  \\
  Instead of teaching logic to nature, it is more reasonable to learn
  from her. Instead of interpreting quantum into logic, we shall
  interpret logic into quantum (Girard 2007).
\end{quote}
Indeed one should take the physical principles underlying quantum mechanics,
the most successful physical theory known to us and adapt computation to
``learn'' from these principles. To illustrate the depth of our crisis, Scott
Aaronson, Umesh Vazirani, and others have proposed the following puzzle. 

One of these wild claims must be true!:
\begin{itemize}
\item the extended Church-Turing thesis is false, or
\item quantum physics is false, or
\item there is an efficient classical algorithm for factoring
\end{itemize} 
Indeed, if quantum physics is correct then there is an efficient quantum
algorithm for factoring (Shor). If there is no efficient classical algorithm
for factoring then the extended Church-Turing thesis is false. 

Indeed, conventional classical models of computation, including boolean
logic, the Turing machine, and the $\lambda$-calculus, are founded on
primitives which correspond to \emph{irreversible} physical processes which
gratuitously erase information.  For example, a \emph{nand} gate is an
irreversible logical operation in the sense that its inputs cannot generally
be recovered from observing its output, and so is the operation of overriding
a cell on a Turing machine tape with a new symbol, and so is a
$\beta$-reduction which typically erases or duplicates values in a way that
is destructive and irreversible.

The irreversible abstractions we choose as the basis of our most
primitive models of computing are at ``odds'' with the underlying
reversible physical reality. The information theoretic underpinnings
of physics is best understood through the remarkable \emph{Landauer
  principle}~\cite{Landauer:1961}. Landauer, through his analysis of
Maxwell's Demon, established the idea that \emph{information is a
  physical quantity} in much the sense that one would think of an
electron as a physical quantity, and that a certain minimum amount of
thermodynamic work has to be carried out to erase one bit of
information. Experimental verification of the Landauer principle
happened in recent years~\cite{berut2012experimental}. If information
has a physical significance, then it is subject to conservation laws
as mass and energy are. 

To make this idea concrete and to provide a taste of the applications
it opens, consider a tiny 2-bit password = \verb|"10"|. The password
checker looks like:

%subcode{code}
%! language = Haskell
% check-password (guess) = 
%   if guess == "10"
%   then True
%   else False

One can ask how much information is leaked by this program assuming
the attacker has no prior knowledge except that the password is 2
bits, i.e., the four possible 2-bits are equally likely. If the
attacker guesses \verb|"10"| (with probability $1/4$) the password (2
bits) is leaked. If the attacker guesses one of the other choices
(with probability $3/4$) the number of possibilities is reduced from 4
to 3, i.e., the attacker learns $\log{4} - \log{3}$ bits of
information. So in general the attacker learns\footnote{If the
  password is 8 restricted ASCII characters (6 bits), the attacker
  learns 0.00001 bits in the first probe.}:
\[\begin{array}{ll}
   &  1/4 * 2 + 3/4 (\log{4} - \log{3}) \\
  =&  1/4 \log{4} + 3/4 \log{4/3} \\
  =&  - 1/4 \log{1/4} - 3/4 \log{3/4} \\
  \sim& 0.8 \mbox{~bits~in~the~first~probe}
\end{array}\]

One can alternatively look at the situation by viewing the input as a
random variable with 4 possibilities and a uniform distribution (i.e.,
with 2 bits of information). The output is another random variable
with 4 possibilities but with the distribution
$\{ (True, 1/4), (False, 3/4) \}$ which contains 0.8 bits of
information. Thus 2 input bits of information were given to the
function and only 0.8 were produced. Where did the 1.2 bits of
information go? Once we accept the thesis that information is a
physical entity this question cannot be ignored.

The Landauer Principle states that erasing information requires
energy. In other words, the password checking function must have
erased 1.2 bits during its calculation which dissipate as heat in any
physical realization of the function. In conventional models of
computation, this erasure is \emph{implicit}, i.e., it occurs as a
side effect of calculation. A big part of our development will be
concerned with making that erasure explicit.

The discussion in this section suggests that our foundations are really in
need of revision. There are many principles we can embrace.  We will start
with one, ``\textbf{conservation of information}'' and follow its
consequences, which will turn out to be far reaching.

%%%
\subsection{Equalities as Isomorphisms}

\begin{quote}
  The laws of physics are essentially algorithms for calculation. These
  algorithms are significant only to the extent that they are executable in
  our real physical world. Our usual laws of physics depend on the
  mathematician's real number system. With that comes the presumption that
  given any accuracy requirement, there exist a number of calculations steps,
  which if executed, will satisfy that accuracy requirement. But the real
  world is unlikely to supply us with unlimited memory of unlimited Turing
  machine tapes, Therefore, continuum mathematics is not executable, and
  physical laws which invoke that can not really be satisfactory. They are
  references to illusionary procedures. Rolf Landauer, The physical nature of
  information, Physics Letters A 217 (1996) pp. 188-193:
\end{quote} 

When a physical law equates two quantities like force and the product of mass
and acceleration, a deep understanding of the law demands that we interpret
the equality as an isomorphism, i.e., as two processes, one in each
direction, that transform one side of the equation to the other. In fact, the
same insight applies to constructive mathematics and its homotopy
interpretation. There is a constructive character to equality that is at the
heart of the constructive approach to mathematics. 

Translated to the computational world, the essence of our thesis is this:
\emph{to speak of equality is to speak of computation}. This view is partly
familiar from the study of various systems of equalities, called
\emph{calculi}, as models of computation. The \lcal, for instance, is a
system of equalities over a set of syntactic $\lambda$-terms such that they
result in a model of computation. We distill this idea to its most primitive
form. We show that there is computational content not just in certain
specific calculi, such as the \lcal, but even in our most primitive notion of
equality -- isomorphisms of finite sets.

The equalities we speak of are not \emph{a priori} extensional
notions, but are intensional specifications. To talk about two things
being equal, one has to show \emph{how they are equal} wherein this
\emph{how} is an intensional definition of a translation from
witnesses of the first to witnesses of the second and vice versa,
i.e. they are programs. Equivalences that do not have operational
content do not exist for us.

One may ask, what is the point of such a rarefied spartan view of
computation? How does this compare with more conventional models of
computation?  \emph{What is the point?} 

\begin{enumerate}

\item The result is computation in its purest form: Programs written
  in this model are witnesses of equalities. In the process of
  computing these programs neither gain nor lose information. We get a
  crisp notion of quantitative information content of programs and
  show that \emph{computation preserves information}.

\item From the perspective of Quantum Physics, the physical world is
  one where every fundamental interaction is essentially reversible
  and various quantities such as energy, mass, angular momentum are
  conserved. Computation is no longer at odds with
  physics. Conservation of information and logical reversibility are
  intrinsic properties of computation as well. Further, various
  category theoretic models such as dagger symmetric traced monoidal
  categories studied for Quantum Physics are also the categorical
  models for computation based on isomorphisms.

\item Irreversibility is a computational effect: By embracing
  irreversible physical primitives, models of computation have also
  implicitly included a class of computational effects which we call
  \emph{information effects}. A consequence of our careful treatment
  of information, is that we effectively capture the gap between
  reversible computation and irreversible computation using a
  type-and-effect system.

\end{enumerate}

What do we mean when we talk about intentional specification of
equality leading to an information preserving model of computation?
The equalities we talk of are processes to transform one thing to the
other. For example, if one were to believe that apples and oranges
were equal then one would have to show how any given apple can be
transformed to an orange i.e. witness of apple-hood get transformed to
witnesses of orange-hood (and vice versa). This is inherently
different from an extensional notion of equality such as
\emph{knowledge is power} and have a more operational sense. Equality
in this sense acts more like the the more general notion of
\emph{isomorphism} as probably best explained in the paper
``\emph{When is one thing equal to some other
  thing?}''~\cite{mazur2008one}.

In our formalism, we are not concerned with fruit but with very simple
things, namely numbers and sets. When we talk of a set of five apples,
we are not concerned about the individuality of the apples that
constitute the set. Rather, each apple in the set has the same status
as any other apple and we care about the individual apples only to the
extent that we can tell them apart to count five distinct ones. We
care about the abstract idea of fiveness and its equality with any
other abstract idea of fiveness.  When used as types, sets take the
role of possibility spaces -- we are no longer talking about a set of
five apples, but talking of one apple out of a possible five. This is
also where the switch from sets to information happens: When one apple
out of five is identified, it corresponds to the information to
discriminate between five choices. Clearly a choice of one out of ten
presents more information in the sense that it is a more
discriminating choice. When a computation promises an output of a
certain type, the resultant value is a witness of one choice out of
several. This is the sense in which we are concerned with equalities
of sets.

Programs in our model are built from descriptions of equalities or
sets/numbers. These are the familiar laws of arithmetic or logic:
\emph{Commutativity.} The generalized notion of swapping,
i.e. $a*b=b*a$ and $a+b=b+a$.  
\emph{Associativity.} The
generalized notion of grouping, i.e. $a*(b*c)=(a*b)*c$ and
$a+(b+c)=(a+b)+c$.
\emph{Identity.} Multiplication by 1 and addition with 0,
  i.e. $a*1=a$ and $a+0=a$.
\emph{Distributivity.} Multiplication distributes over addition
  and cancels at zero, i.e. $a*(b+c)=a*b+a*c$ and $a*0=0$.
\emph{Cancellation.} Cancellation is the idea that equal things
  on both sides of a equality can be canceled out, i.e. $a+b=a+c$
  implies $b=c$, and is referred to as a \emph{trace} in category
  theory~\cite{joyal1996traced}.

How, one may ask, does this form a computational model? Even in very
simple computational models such as the SK combinatory basis, the
combinator K provides the basis of choice/conditional by deleting an
argument and the combinator S provides the basis for iteration by
duplicating an argument. As we will see, distributivity gives us
conditionals and trace gives us iteration, albeit in a logically
reversible manner.  In Chs.~\ref{ch:pi} and~\ref{ch:pi0} we work out
the details of the computational model, show that it is Turing
complete and develop programs in it including several simple numeric
operations, algebraic manipulations of trees and a meta-circular
interpreter (see Ch.~\ref{ch:interpreters}).

More formally, we develop an information preserving model of
computation, wherein the process of computing does not gain or lose
information. Our model arises from a computational interpretation of
type isomorphisms with iso-recursive types and trace. This can
equivalently be described as computing a commutative semi-ring with
cancellation or as a traced dagger symmetric bimonoidal category. The
category has a groupoid structure and all computations are logically
reversible. 

To summarize, the most primitive notion of equality that is available
to us -- that of isomorphisms -- already contains the notion of
computation. The resulting computational model exposes the fine
structure of how computation handles information. We believe that such
a model of computing is foundational and we are only beginning to
understand its applications and full potential. In conclusion, let us
quote Paul Blain Levy's fine advocacy slogan: Once the fine structure
has been exposed, why ignore it?

%%%
\subsection{Information}

We can now describe information preservation in a precise sense.  Let
`$b$' be a (not necessarily finite) type whose values are labeled
$b^1, b^2, \ldots$. Let $\xi$ be a random variable of type $b$ that is
equal to $b^i$ with probability $p_i$. The entropy of $\xi$ is defined
as $-\sum p_i \log{p_i}$~\cite{Shannon1948}.  Consider a function
$f : b_1 \rightarrow b_2$ where $b_2$ is a (not necessarily finite)
type whose values are labeled $b_2^1, b_2^2, \ldots$. The output
entropy of the function is given by $- \sum q_j \log{q_j}$ where $q_j$
indicates the probability of the output of the function to have value
${b_2}^j$. We say a function is \emph{information-preserving} if its
output entropy is equal to the entropy of its input. See
Ch.~\ref{ch-metalang} for details.

Operations that change the information content of programs have a
special status. We call them \emph{information effects} and they are
encapsulated using an arrow meta-language in much the same way that
one would encapsulate other computational effects in the \lcal\ using a
monadic metalanguage.  The treatment of information effects is
analogous to open and closed systems in physics. Closed physical
systems conserve mass and energy and are the basic unit of study in
physics. Pure computations that do not have information effects are
like closed physical systems. They describe equalities.  Open systems
interact with their environment, possibly exchanging mass or
energy. These interactions may be thought of as \emph{effects} that
modify the conservation properties of the system. Computations with
information effects are much like open systems and they can be
converted into pure computations by making explicit the surrounding
information environment that they interact with. 

Three things follow: (1) We capture the gap between reversible and
irreversible computation with a type-and-effect system. (2)
Categorically speaking, information effects turn monoidal categories
into ones that have a cartesian structure corresponding to
conventional computation.  (3) We show how conventional irreversible
computation such as the \lcal\ can be embedded into this model, such
that the embedding makes the implicit information effects of \lcal\
explicit.  As a consequence of this approach, many applications in
which information manipulation is computationally significant are put
within the reach of our conceptual model of computation. Such
applications include quantitative information-flow
security~\cite{myerssab}, differential
privacy~\cite{dwork:differential}, energy-aware
computing~\cite{1324180,605411}, VLSI
design~\cite{Macii:1996:ECE:874066.875828}, and biochemical models of
computation~\cite{bio}. 

Ch.~\ref{ch:piee} focuses on work in progress. It deals with the
duality of computation in the setting of information
preservation. Andrzej Filinski's work on the symmetric
\lcal\ established the fascinating connection that values are dual to
continuations, functions are dual to delimited continuations and
established the duality of call-by-name and call-by-value evaluation
strategies~\cite{Filinski:1989:DCI:648332.755574}. We explore
Filinski-style duality in the setting of our information preserving
model of computing. Here duality does not appear as one De
Morgan-style dualizing $\neg$ operator, but as two distinct dualizing
operations -- negation and division. This results in a crisp semantics
for negative and fractional types and the resulting system has a
equational formalism in zero-totalized fields or
meadows~\cite{DBLP:journals/tcs/BergstraHT09}. Computational analogues
of physical phenomena such as superposition and entanglement appear in
the resulting semantics.

Talk about conservation of information in biology, computer science,
evolutionary computing, and search. (See
\url{http://www.evolutionnews.org/2012/08/conservation_of063671.html}.

%%%
\subsection{Technical Overview}

The technical development derives from ideas which are implicit in the
works of \cite{Toffoli:1980, Zuliani:2001:LR,
  malacaria2007assessing,ClarkHM07,Ghica:2007:GSS:1190216.1190269}.
(1) Functions whose output entropy is equal to their input entropy are
information preserving functions.  (2) A function
$f : b_1 \rightarrow b_2$ is \emph{logically reversible} if there
exists an inverse function $f^{\dagger}$ such that for any input
$v_1:b_1$ if there exists $v_2:b_2$ such that $f(v_1)=v_2$ \emph{iff}
$f^{\dagger}(v_2)=v_1$.  (3) Logically reversible functions are
information preserving.

To summarize:
\begin{enumerate}
\item Based on isomorphisms of sum and product
  types~\cite{Fiore:2004}, we develop a strong normalizing model of
  computing, which we call $\Pi$, that is logically reversible and
  information preserving. Every computation expressible in $\Pi$
  is an isomorphism.

\item We extend $\Pi$ with isorecursive type and trace operators from
  category theory to obtain $\Pi$. Every computation expressible in
  $\Pi$ is a partial isomorphism --- i.e. the system admits
  non-termination. $\Pi$ is Turing complete while being information
  preserving and logically reversible.

\item We develop the categorical semantics of these
  models~\cite{rc2011}. \amr{Mention n-categories.} We also develop
  a graphical notation for programming in these models that
  reminiscent of Penrose diagrams for
  categories~\cite{selinger-graphical}, Geometry of Interaction (GoI)
  machines and Proof
  Nets~\cite{Mackie2011,DBLP:conf/popl/Mackie95}. In the graphical
  notation, computation is modeled by the flow of particles in a
  circuit.

\item Since these new models are substantially different from \lcal\
  or familiar high-level languages, a point of concern is how one can
  effectively develop programs in them.  We address this by presenting
  a straightforward technique for deriving $\Pi$ programs by
  systematically translating logically reversible small-step abstract
  machines~\cite{isoint}. Complex programs can be devised in this way;
  we demonstrate the derivation of a meta-circular interpreter for
  $\Pi$.

\item We develop an arrow meta-language to encapsulate information
  effects. This metalanguage serves to encapsulate effects, in much
  the same way that traditional arrows or monads serve to encapsulate
  effects over the \lcal.

\item We show how a conventional irreversible model can be expressed
  in our model. We compile the first-order fragment of typed \lcal\
  extended with sums, products and loops to $\Pi$. This compilation is
  interesting for two reasons: (1) it exposes the implicit information
  effects of \lcal\ in the compilation to $\Pi$ and (2) the
  compilation of $\Pi$ to $\Pi$ shows that information effects can be
  erased by treating them as interactions with an explicit information
  environment.

\item We develop the notion of `duality' in $\Pi$ which gives us not
  one duality (like in linear logic or Filinski's symmetric
  \lcal~\cite{Filinski:89}), but two notions of duality -- an additive
  duality and a multiplicative duality. This gives a crisp semantics
  for negative and fractional types in the context of $\Pi$.

\end{enumerate}

Several intriguing areas of exploration remain and the book concludes
with discussion of some of these, along with other possible
applications (see Ch.~\ref{ch:discussion}).  Of these, the field of
reversible and Quantum computing is closely related and we survey
several connections in depth.  The exact connection between $\Pi$ and
linear logic~\cite{Girard87tcs} is not fully understood though it is
clear that both systems capture related but different notions of
resource usage. A closely related area is the connection with duality
of computation \cite{Filinski:89, DBLP:conf/icfp/CurienH00,
  10.1109/LICS.2010.23,Wadler:2003} --- i.e., the quest for a unifying
framework between `computations', values, continuations and their
logical counterparts. The investigation of duality gives rise to
negative and fractional types which have an intuitive semantics and
are reminiscent of negative information flow, superposition and
entanglement from Quantum Physics~\cite{piee}.

%%%%%%%%%%%%%%%%%%%%%%%%%%%%%%%%%%%%%%%%%%%%%%%%%%%%%%%%%%%%%%%%%%%%%%%%%%%%%%
\section{Physics, Computation, and Logic}

Conflict between Church-Turing thesis, textbook quantum mechanics, and
RSA: Either Shor's algorithm is not ``natural''. (Textbook quantum
mechanics is wrong); or, Shor's algorithm is ``natural'' and there is
no classical counterpart. (There are ``natural'' computing models that
are exponentially faster than the Turing Machine); or, there is an
efficient classical factoring algorithm. (Possible but surprising); At
least one of these wild claims is true!!!

\noindent Also conflict between quantum mechanics and logic. Possibilities:

\paragraph*{Revise quantum mechanics.}

The mathematician's vision of an unlimited sequence of totally
reliable operations is unlikely to be implementable in this real
universe.  But the real world is unlikely to supply us with unlimited
memory or unlimited Turing machine tapes. Therefore, continuum
mathematics is not executable, and physical laws which invoke that can
not really be satisfactory. They are references to illusionary
procedures.  (Landauer 1996 and 1999)

% \jc{No access to infinite memory/tape just means that ``infinitary''
% arguments do not go through. The MoC with reversible computation
% assumes totally reliable operations (but not necessarily infinitely
% many). You leap to ``continuum'' mathematics without justification.
% While QC does use the continuum in non-trivial ways, and physical
% reality is ``discrete'' at the atomic level (but who knows at the
% quantum level?!?), I don't think that's where you want to go here.
% But see below.}

I want to talk about the possibility that there is to be an exact
simulation, that the computer will do exactly the same as nature. If
this is to be proved and the type of computer is as I've already
explained, then it's going to be necessary that everything that
happens in a finite volume of space and time would have to be exactly
analyzable with a finite number of logical operations. The present
theory of physics is not that way, apparently. It allows space to go
down into infinitesimal distances, wavelengths to get infinitely
great, terms to be summed in infinite order, and so forth; and
therefore, if this proposition is right, physical law is wrong.
(Feynman 1981)

% \jc{In a sense, physics already does this: statistical mechanics.
% Even though the systems under study are explicitly discrete, albeit
% will huge numbers of particules, the behaviour is approximated (!!)
% via a continuous model. Same in economics with Black-Scholes.
% The real problem are those conclusions drawn from such models which
% *require* arbitrarily small values to exist, and even worse, for
% the underlying values (reals) to be connected, i.e. form an actual
% continuum. What Feynman is asking here is 'finiteness of information',
% which is what things like Type-2 computability also ask for.}

\paragraph*{Revise Computer Science.}

Another thing that had been suggested early was that natural laws are
reversible, but that computer rules are not. But this turned out to be
false; the computer rules can be reversible, and it has been a very,
very useful thing to notice and to discover that. This is a place
where the relationship of physics and computation has turned itself
the other way and told us something about the possibilities of
computation. So this is an interesting subject because it tells us
something about computer rules.  (Feynman 1981)

Turing hoped that his abstracted-paper-tape model was so simple, so
transparent and well defined, that it would not depend on any
assumptions about physics that could conceivably be falsified, and
therefore that it could become the basis of an abstract theory of
computation that was independent of the underlying physics. ``He
thought, as Feynman once put it, that he understood paper. But he
was mistaken. Real, quantum-mechanical paper is wildly different from
the abstract stuff that the Turing machine uses. The Turing machine is
entirely classical, and does not allow for the possibility the paper
might have different symbols written on it in different universes, and
that those might interfere with one another.  (Deutsch 1985)

Ed Fredkin pursued the idea that information must be finite in
density. One day, he announced that things must be even more simple
than that. He said that he was going to assume that information itself
is conserved. ``You're out of you mind, Ed.'' I pronounced. ``That
completely ridiculous. Nothing could happen in such a world. There
couldn't even be logical gates. No decisions could ever be made.'' But
when Fredkin gets one of his ideas, he's quite immune to objections
like that; indeed, they fuel him with energy. Soon he went on to
assume that information processing must also be reversible and
invented what is now called the Fredkin gate. (Minsky 1999)

Even revise the laws of thought themselves: In other terms, what is so
good in logic that quantum physics should obey? Can't we imagine that
our conceptions about logic are wrong, so wrong that they are unable
to cope with the quantum miracle? [\ldots] Instead of teaching logic
to nature, it is more reasonable to learn from her. Instead of
interpreting quantum into logic, we shall interpret logic into
quantum. (Girard 2007)

One must be careful with this argument: we have a ``current
understanding'' of quantum physics, which could be quite flawed.  So
what we want to do is to learn from those parts of quantum physics
which seem to be extremely stable (like conservation of information!)
first. As researchers, we can certainly speculate on it all, but it
makes sense to vigorously pursue the conclusions one can draw from
those aspects of (quantum) physics that seem stable.

%%%%%%%%%%%%%%%%%%%%%%%%%%%%%%%%%%%%%%%%%%%%%%%%%%%%%%%%%%%%%%%%%%%%%%%%%%%%%%
\section{Conservation of Information}

A physical principle of computation

No creation of information
No duplication of information

All programs, proofs, deductions are equivalences, isomorphisms,
reversible

Several pages on reversible logic; reversible family of level 1 and
level 2 languages (Pi);

We should also have a quick survey of other works in reversible
computation? Most of the other work tries to start from well known
models of computation or well known programming languages, and then
adapts them to be reversible. Pi is different in that it starts from
the semantic implications of reversibility. Then, because it adds
types, rather than control-flow (or even data-flow) as its next layer
of ``understanding'', this leads to equivalences, isomorphisms, etc.
This then leads, quite naturally, to finding that the ``proof
language'' of semirings (and Rig Groupoids at level 2) is actually a
programming language. And it is Pi. This is a neat twist on
Curry-Howard because CH is about \textbf{inhabitation} only. But with
``conservation of information'' as the basis, a different kind of
correspondance arises; in fact, this one may well be an actual
isomorphism.

%%%%%%%%%%%%%%%%%%%%%%%%%%%%%%%%%%%%%%%%%%%%%%%%%%%%%%%%%%%%%%%%%%%%%%%%%%%%%%
\section{Applications}

Perhaps Information Flow Security

% \jc{This is where I am still least sure. I am wary of either doing
% new research with such a tight timeline, or of being too speculative.
% If there are papers on this already, perhaps references to them could
% be put here?}

%%%%%%%%%%%%%%%%%%%%%%%%%%%%%%%%%%%%%%%%%%%%%%%%%%%%%%%%%%%%%%%%%%%%%%%%%%%%%%
\section{Conclusions}

%%%%%%%%%%%%%%%%%%%%%%%%%%%%%%%%%%%%%%%%%%%%%%%%%%%%%%%%%%%%%%%%%%%%%%%%%%%%%%
\bibliographystyle{plainnat}
\pagestyle{headings}
\bibliography{cites}
\end{document}
