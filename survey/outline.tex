\documentclass{article}
\usepackage{fullpage}
\title{Embracing the Laws of Physics: \\ Reversible Models of Computation}
\author{Jacques Carette, Roshan P. James, Amr Sabry}

\newcommand{\amr}[1]{\fbox{Amr says:} \textbf{#1}}
\newcommand{\jc}[1]{\fbox{Jacques says:} \textbf{#1}}
\newcommand{\roshan}[1]{\fbox{Roshan says:} \textbf{#1}}

\begin{document}
\maketitle

%%%%%%%%%%%%%%%%%%%%%%%%%%%%%%%%%%%%%%%%%%%%%%%%%%%%%%%%%%%%%%%%%%%%%%%%%%%%%%
\section{Introduction : Reversibility, the Missing Principle}


To fully appreciate the missing principle of conventional computing,
one must go back to an old thought experiment by J. C. Maxwell,
codified in a letter that Maxwell wrote to P. G Tait in 1867 -- this
is the letter whose ideas are now well known as the `Maxwell's
Demon'. Maxwell's Demon was a thought experiment that seemed to
indicate that intelligent beings can somehow violate the second law of
thermodynamics, thereby violating physics itself.

Many resolutions were offered for this conundrum (for a compilation,
see Maxwell's Demon by Leff and Rex), but none withstood careful
scrutiny until the establishment of the so called `Landauer's
Principle' in 1961 -- a principle whose experimental validation
happened recently in 2012 by Eric Lutz and others.

Maxwell's demon appears to violate the second law of thermodynamics by
having a tiny `intelligence' observing the movement of individual
particles of a gas and separating fast moving particles from slow
moving ones, thereby reducing the total entropy of the
system. Landauer's resolution of the demon relied on two ideas that
took root only a few decades earlier -- the formal notion of
computation (through the work of Turing and Church, 1936) and the
formal notion of information (through the work of Claude Shannon,
1948). Landauer reasoned that the computation done by the finite brain
of the demon, involves getting information about the movement of
molecules, acting on that information and then overwriting it to make
room for the next computation.  Landauer reasoned, and this is the
important bit, that the computation that is manipulating information
in the demon's brain \textit{must be thermodynamic work}, thereby
bringing the demon back into the fold of physics.

In its essence this is a strange and wonderful idea. Information,
physics and computation are inextricably linked and that ideas in each
field have implications for the other. (Cite Bennett and the various
`thermodynamics of computation' papers here.)

When the early models computation, the Turing machine, and the
$\lambda$-calculus, were developed there was no real reason to the
take the information content of computations into serious
consideration -- in fact, at that time there was no quantifiable
notion of information. These models followed in the footsteps of logic
where conventional logic gates such as AND and OR happily erase their
inputs.

\begin{quote}
Toffoli 1980: Mathematical models of computation are abstract
constructions, by their nature unfettered by physical laws. However,
if these models are to give indications that are relevant to concrete
computing, they must somehow capture, albeit in a selective and
stylized way, certain general physical restrictions to which all
concrete computing processes are subjected.
\end{quote}

Now however, through the establishment of the Landauer's principle and
through recent investigations into computational models that are well
suited to quantum mechanics it has become increasingly evident that
our computation models need to be revisited. Information, not just
seems to have a physical manifestation, but also that it, like other
physical entities, is conserved.

Going back to the drawing board, can we have computation that closely
matches our underlying physics?  i.e. can we have computation that in
its essence preserves information? What is the missing principle that
we can use as a basis of a new model of computation to achieve this?
It turns out that we need to look at physics again for inspiration --
in physics all of the primitive rules, nature itself at its most
fundamental level, is reversible. Conventional computation is not. Can
we embrace this simple principle as the building block of a model of
computation? What would computation look like when viewed from this
vantage point? And, does it have applications?

The objective of this survey is to touch upon the many related areas
of research in this space.

%% Abstract models of computation: Turing machine, RAM, Von Neumann
%% architecture, $\lambda$-calculus, logic, discrete mathematics

%% \jc{why are logic and DM listed alongside MoC?}

%% Cyber-physical systems increasingly common. Are computing foundations
%% adequate?
%% \jc{your questioning of MoC below is not about CPS! Should either
%% foreshadow the next section 'properly' here, or add a leading
%% example based on CPS to MoC section}


%%%%%%%%%%%%%%%%%%%%%%%%%%%%%%%%%%%%%%%%%%%%%%%%%%%%%%%%%%%%%%%%%%%%%%%%%%%%%%
\section{Models of Computation}

Explain classical models of computation and their limitations in
real-world ``physical'' applications.

Computation in the ``real world'' vs. idealized computation in
abstract models.
\begin{itemize}
\item Mathematical abstraction of sorting is well-understood; sorting
  petabytes of data violates many of the assumptions: hardware
  component failure frequent, global synchronization impossible;
  consistency must be eventual; etc.
\item Mathematical abstraction of sortedness property: easy; in the
  real world with petabytes of evolving data; it is not clear that it
  is even possible to check sortedness; easy to get inconsistent
  results
\item Noise does not exist in our mathematical abstractions of
  computation; essential use in differential privacy
\item Mathematical models of secure computing typically ignore power
  consumption, electromagnetic signatures; actual attacks based on DPA
  and DEMA however
\end{itemize}

% \jc{While all the above are true (and not exhaustive), what I also read
% is that the first two are actually ``the same'' point mostly related
% to scale and distributivity, the 3rd is about inadequate models (but not
% related to physics), while the 4th is about the physics of devices.
% They are not about CPS. And not about Quantum either. So how are these
% motivating a deeper look at how quantum affects our MoC? It feels to me
% that we ought to have a quantum ``example'' here already.}

% \jc{What this seems to motivate is to look at our models with respect
% to what we're trying to achieve.  Models are, by definition, abstractions
% that ``abstract away'' irrelevant details. So the point would be that
% some details that we thought were irrelevant really are not.
% So I could see this section motivating a whole programme of
% targeted de-abstraction. This could lead into the focus of this paper
% following one strand, namely one aspect of quantum physics.}

%%%%%%%%%%%%%%%%%%%%%%%%%%%%%%%%%%%%%%%%%%%%%%%%%%%%%%%%%%%%%%%%%%%%%%%%%%%%%%
\section{Physics, Computation, and Logic}

Conflict between Church-Turing thesis, textbook quantum mechanics, and
RSA: Either Shor's algorithm is not ``natural''. (Textbook quantum
mechanics is wrong); or, Shor's algorithm is ``natural'' and there is
no classical counterpart. (There are ``natural'' computing models that
are exponentially faster than the Turing Machine); or, there is an
efficient classical factoring algorithm. (Possible but surprising); At
least one of these wild claims is true!!!

\noindent Also conflict between quantum mechanics and logic. Possibilities:

\paragraph*{Revise quantum mechanics.}

The mathematician's vision of an unlimited sequence of totally
reliable operations is unlikely to be implementable in this real
universe.  But the real world is unlikely to supply us with unlimited
memory or unlimited Turing machine tapes. Therefore, continuum
mathematics is not executable, and physical laws which invoke that can
not really be satisfactory. They are references to illusionary
procedures.  (Landauer 1996 and 1999)

% \jc{No access to infinite memory/tape just means that ``infinitary''
% arguments do not go through. The MoC with reversible computation
% assumes totally reliable operations (but not necessarily infinitely
% many). You leap to ``continuum'' mathematics without justification.
% While QC does use the continuum in non-trivial ways, and physical
% reality is ``discrete'' at the atomic level (but who knows at the
% quantum level?!?), I don't think that's where you want to go here.
% But see below.}

I want to talk about the possibility that there is to be an exact
simulation, that the computer will do exactly the same as nature. If
this is to be proved and the type of computer is as I've already
explained, then it's going to be necessary that everything that
happens in a finite volume of space and time would have to be exactly
analyzable with a finite number of logical operations. The present
theory of physics is not that way, apparently. It allows space to go
down into infinitesimal distances, wavelengths to get infinitely
great, terms to be summed in infinite order, and so forth; and
therefore, if this proposition is right, physical law is wrong.
(Feynman 1981)

% \jc{In a sense, physics already does this: statistical mechanics.
% Even though the systems under study are explicitly discrete, albeit
% will huge numbers of particules, the behaviour is approximated (!!)
% via a continuous model. Same in economics with Black-Scholes.
% The real problem are those conclusions drawn from such models which
% *require* arbitrarily small values to exist, and even worse, for
% the underlying values (reals) to be connected, i.e. form an actual
% continuum. What Feynman is asking here is 'finiteness of information',
% which is what things like Type-2 computability also ask for.}

\paragraph*{Revise Computer Science.}

Another thing that had been suggested early was that natural laws are
reversible, but that computer rules are not. But this turned out to be
false; the computer rules can be reversible, and it has been a very,
very useful thing to notice and to discover that. This is a place
where the relationship of physics and computation has turned itself
the other way and told us something about the possibilities of
computation. So this is an interesting subject because it tells us
something about computer rules.  (Feynman 1981)

Turing hoped that his abstracted-paper-tape model was so simple, so
transparent and well defined, that it would not depend on any
assumptions about physics that could conceivably be falsified, and
therefore that it could become the basis of an abstract theory of
computation that was independent of the underlying physics. ``He
thought, as Feynman once put it, that he understood paper. But he
was mistaken. Real, quantum-mechanical paper is wildly different from
the abstract stuff that the Turing machine uses. The Turing machine is
entirely classical, and does not allow for the possibility the paper
might have different symbols written on it in different universes, and
that those might interfere with one another.  (Deutsch 1985)

Ed Fredkin pursued the idea that information must be finite in
density. One day, he announced that things must be even more simple
than that. He said that he was going to assume that information itself
is conserved. ``You're out of you mind, Ed.'' I pronounced. ``That
completely ridiculous. Nothing could happen in such a world. There
couldn't even be logical gates. No decisions could ever be made.'' But
when Fredkin gets one of his ideas, he's quite immune to objections
like that; indeed, they fuel him with energy. Soon he went on to
assume that information processing must also be reversible and
invented what is now called the Fredkin gate. (Minsky 1999)

Even revise the laws of thought themselves: In other terms, what is so
good in logic that quantum physics should obey? Can't we imagine that
our conceptions about logic are wrong, so wrong that they are unable
to cope with the quantum miracle? [\ldots] Instead of teaching logic
to nature, it is more reasonable to learn from her. Instead of
interpreting quantum into logic, we shall interpret logic into
quantum. (Girard 2007)

One must be careful with this argument: we have a ``current
understanding'' of quantum physics, which could be quite flawed.  So
what we want to do is to learn from those parts of quantum physics
which seem to be extremely stable (like conservation of information!)
first. As researchers, we can certainly speculate on it all, but it
makes sense to vigorously pursue the conclusions one can draw from
those aspects of (quantum) physics that seem stable.

%%%%%%%%%%%%%%%%%%%%%%%%%%%%%%%%%%%%%%%%%%%%%%%%%%%%%%%%%%%%%%%%%%%%%%%%%%%%%%
\section{Conservation of Information}

A physical principle of computation

No creation of information
No duplication of information

All programs, proofs, deductions are equivalences, isomorphisms,
reversible

Several pages on reversible logic; reversible family of level 1 and
level 2 languages (Pi);

We should also have a quick survey of other works in reversible
computation? Most of the other work tries to start from well known
models of computation or well known programming languages, and then
adapts them to be reversible. Pi is different in that it starts from
the semantic implications of reversibility. Then, because it adds
types, rather than control-flow (or even data-flow) as its next layer
of ``understanding'', this leads to equivalences, isomorphisms, etc.
This then leads, quite naturally, to finding that the ``proof
language'' of semirings (and Rig Groupoids at level 2) is actually a
programming language. And it is Pi. This is a neat twist on
Curry-Howard because CH is about \textbf{inhabitation} only. But with
``conservation of information'' as the basis, a different kind of
correspondance arises; in fact, this one may well be an actual
isomorphism.

%%%%%%%%%%%%%%%%%%%%%%%%%%%%%%%%%%%%%%%%%%%%%%%%%%%%%%%%%%%%%%%%%%%%%%%%%%%%%%
\section{Applications}

Perhaps Information Flow Security

% \jc{This is where I am still least sure. I am wary of either doing
% new research with such a tight timeline, or of being too speculative.
% If there are papers on this already, perhaps references to them could
% be put here?}

%%%%%%%%%%%%%%%%%%%%%%%%%%%%%%%%%%%%%%%%%%%%%%%%%%%%%%%%%%%%%%%%%%%%%%%%%%%%%%
\section{Conclusions}

%%%%%%%%%%%%%%%%%%%%%%%%%%%%%%%%%%%%%%%%%%%%%%%%%%%%%%%%%%%%%%%%%%%%%%%%%%%%%%
\end{document}
