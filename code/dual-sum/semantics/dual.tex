% Compile this document using omake. There should be an OMakefile and
% OMakeroot in this directory. You can also run omake in `server mode'
% by typing `omake -P' where it will monitor for file changes and
% recompile when required.

\documentclass[preprint]{easychair}

%\usepackage{fullpage}
%\usepackage{hyperref}
\usepackage{graphicx}
\usepackage{longtable}
\usepackage{comment}
\usepackage{amsmath}
\usepackage{mdwlist}
\usepackage{txfonts}
\usepackage{xspace}
\usepackage{amstext}
\usepackage{amssymb}
\usepackage{stmaryrd}
\usepackage{proof}
\usepackage{multicol}
% \usepackage[colon]{natbib}
\usepackage[nodayofweek]{datetime}
\usepackage{etex}
\usepackage[all, cmtip]{xy}
\usepackage{xcolor}
\usepackage{listings}
\usepackage{multicol}
% \usepackage{inconsolata}

% %%%%%%%%%%%%%%%%%%%%%%%%%%%%%%%%%%%%%%%%%%%%%%%%%%%%%%%%%%%%%%%%%%%%%%%%%%%%%
\newtheorem{theorem}{Theorem}[section]
\newtheorem{lemma}[theorem]{Lemma}
\newtheorem{proposition}[theorem]{Proposition}
\newtheorem{corollary}[theorem]{Corollary}

%% \newenvironment{proof}[1][Proof]{\begin{trivlist}
%% \item[\hskip \labelsep {\bfseries #1}]}{\end{trivlist}}
%% \newenvironment{definition}[1][Definition]{\begin{trivlist}
%% \item[\hskip \labelsep {\bfseries #1}]}{\end{trivlist}}
%% \newenvironment{example}[1][Example]{\begin{trivlist}
%% \item[\hskip \labelsep {\bfseries #1}]}{\end{trivlist}}
%% \newenvironment{remark}[1][Remark]{\begin{trivlist}
%% \item[\hskip \labelsep {\bfseries #1}]}{\end{trivlist}}

% \newcommand{\qed}{\nobreak \ifvmode \relax \else
%       \ifdim\lastskip<1.5em \hskip-\lastskip
%       \hskip1.5em plus0em minus0.5em \fi \nobreak
%       \vrule height0.75em width0.5em depth0.25em\fi}

%% \newcommand{\qed}{\hfill \ensuremath{\Box}}


%%%%%%%%%%%%%%%%%%%%%%%%%%%%%%%%%%%%%%%%%%%%%%%%%%%%%%%%%%%%%%%%%%%%%%%%%%%%%
\newcommand{\xcomment}[2]{\textbf{#1:~\textsl{#2}}}
\newcommand{\amr}[1]{\xcomment{Amr}{#1}}
\newcommand{\roshan}[1]{\xcomment{Roshan}{#1}}

\newcommand{\lcal}{$\lambda$-calculus\xspace}
\newcommand{\yield}{\ensuremath{\mathit{yield}}\xspace}
\newcommand{\run}{\ensuremath{\mathit{run}}\xspace}
\newcommand{\shift}{\ensuremath{\mathit{shift}}\xspace}
\newcommand{\reset}{\ensuremath{\mathit{reset}}\xspace}
\newcommand{\iter}{\textit{iterator}\xspace}
\newcommand{\iters}{\textit{iterators}\xspace}
\newcommand{\Iter}{\textit{Iterator}\xspace}
\newcommand{\Iters}{\textit{Iterators}\xspace}
\newcommand{\ie}{\textit{i.e.}\xspace}
\newcommand{\eg}{\textit{e.g.}\xspace}

\newcommand{\code}[1]{\lstinline[basicstyle=\small]{#1}\xspace}
%\newcommand{\name}[1]{\texttt{\textbf{#1}}\xspace}
\newcommand{\name}[1]{\code{#1}}

\def\newblock{}

\newenvironment{floatrule}
    {\hrule width \hsize height .33pt \vspace{.5pc}}
    {\par\addvspace{.5pc}}

%%%%%%%%%%%%%%%%%%%%%%%%%%%%%%%%%%%%%%%%%%%%%%%%%%%%%%%%%%%%%%%%%%%%%%%%%%%%%

%subcode-inline{bnf-inline} name langRev
%! swap+ = \mathit{swap}^+
%! swap* = \mathit{swap}^*
%! dagger =  ^{\dagger}
%! assocl+ = \mathit{assocl}^+
%! assocr+ = \mathit{assocr}^+
%! assocl* = \mathit{assocl}^*
%! assocr* = \mathit{assocr}^*
%! identr* = \mathit{uniti}
%! identl* = \mathit{unite}
%! dist = \mathit{distrib}
%! factor = \mathit{factor}
%! (o) = \fatsemi
%! (;) = \fatsemi
%! (*) = \times
%! (+) = +



%subcode-inline{bnf-inline} regex \{\{(((\}[^\}])|[^\}])*)\}\} name main include langRev
%! Gx = \Gamma^{\times}
%! G = \Gamma
%! [] = \Box
%! |-->* = \mapsto^{*}
%! |-->> = \mapsto_{\ggg}
%! |--> = \mapsto
%! |- = \vdash
%! ==> = \Longrightarrow
%! <== = \Longleftarrow
%! <=> = \Longleftrightarrow
%! <-> = \leftrightarrow
%! ~> = \leadsto
%! ::= = &::=&
%! /= = \neq
%! forall = \forall
%! exists = \exists
%! empty = \epsilon
%! langRev = \Pi
%! theseus = Theseus

\begin{document}

%
\title{Additive Duality in {{langRev}} } 

% \titlerunning{} has to be set to either the main title or its shorter
% version for the running heads. Use {\sf} for highliting your system
% name, application, or a tool.
%
\titlerunning{Additive Duality} 

%% \author{Roshan P. James\\
%% Indiana University\\
%% Bloomington, Indiana, U.S.A.\\
%% \url{rpjames@indiana.edu}\\
%% \and
%% Amr Sabry\\
%% Indiana University\\
%% Bloomington, Indiana, U.S.A.\\
%% \url{sabry@indiana.edu}\\
%% }

% \authorrunning{} has to be set for the shorter version of the authors' names;
% otherwise a warning will be rendered in the running heads.
%
\authorrunning{R. P. James, and A. Sabry}

\maketitle

%% \begin{abstract}
%%   %% {{langRev}} is too low level, almost like an assembly
%%   %% langauge. {{theseus}} is a much more humane language.
%% \end{abstract}


% \category{D.1.1}{Applicative (Functional) Programming}{}
% \category{D.3.3}{Language Constructs and Features}{Control structures,
% Coroutines} \category{F.3.3}{Logics and Meanings of Programs}{Control
% primitives}
% \terms
% Languages
% \keywords
% continuation, control flow, Haskell, iterator, lazy evaluation, monad

%%%%%%%%%%%%%%%%%%%%%%%%%%%%%%%%%%%%%%%%%%%%%%%%%%%%%%%%%%%%%%%%%%%%%%%%%%%%%%%%
\section{Small Step Semantics}
\subsection{Grammar and other definitions}

%subcode{bnf} include main
% Values, v = () | (v, v) | L v | R v
% Combinators, c &=& iso | c (;) c | c(+)c | c (*) c 
%
% Sequential Contexts, P, F = [] | P:c
% Parallel Contexts, D = [] 
%                     &|& (P, [](+)c, F):D 
%                     &|&  (P, c(+)[], F):D
%                     &|& (P, c v, F):D 
%                     &|&  (P, v c, F):D
%
%
% Machine States  = <P, F, c v, D> | <P, F, v c, D>
% Start State  = <[], [], c v, []>
% Stop State  = <P, [], v c, []>

Combinator reconstruction, {{P[c]}}: 
%subcode{opsem} include main
% [][c] '= c
% (P:c')[c] '= P[c'(;)c]

\subsection{Operational Semantics}

The small step semantics present for {{langRev}} below work
symmetrically for forward and backward evaluation.

\begin{itemize}
\item 
Basic reduction of an isomorphism. Note that the evaluation leaves the
adjoint of the combinator behind. This will become important when we
reverse the direction of computation.
%subcode{opsem} include main
% <P; F; iso v; D>      &|-->& <P; F; v'~ iso{dagger}; D>

\item
Sequencing involves pushing and popping from the Future and Past
continuations:
%subcode{opsem} include main
% <P, F, c1(;)c2 v , D>  &|-->& <P, F:c2, c1 v, D>
% <P, F:c2, v c1, D> &|-->& <P:c1, F, c2 v, D>


\item
Parallel composition, captures the current Future and Past and extends
a Dump/meta-continuation.
%subcode{opsem} include main
% <P, F, c1 (+) c2 (L v), D> &|-->& <[], [], c1 v, (P,[](+)c2,F):D>
% <P, F, c1 (+) c2 (R v), D> &|-->& <[], [], c2 v, (P,c1(+)[],F):D> 
% <P', [], v c1, (P,[](+)c2,F):D> &|-->& < P, F, (L v) (P'[c1](+)c2{dagger}), D > 
% <P', [], v c2, (P,c1(+)[],F):D> &|-->& <P, F, (R v) (c1{dagger}(+)P'[c2]), D>


\item
Similarly for products:
%subcode{opsem} include main
% <P, F, c1 (*) c2 (v1, v2), D> &|-->& <[], [], c1 v1, (P, [] (*) c2 v2,F):D>
% <P', [], v1 c1, (P,[] (*) c2 v2,F):D> &|-->& <[], [], c2 v2, (P,v1 P'[c1] (*) [],F):D>
% <P', [], v2 c2, (P,v1 c1 (*) [],F):D> &|-->& <P, F, (v1, v2) (c1(*)P'[c2]):D>
and symmetrical rules for evaluation along the second branch. 
%subcode{opsem} include main
% <P', [], v2 c2, (P,c1 v1 (*) [],F):D> &|-->& <[], [], c1 v1, (P,[] (*) v2 P'[c2],F):D>
% <P', [], v1 c1, (P,[] (*) v2 c2,F):D> &|-->& <P, F, (v1, v2) (P'[c1] (*) c2):D>

The later two rules will be relevant only for reverse execution. 

\end{itemize}


\section{Rules for {{eta}} and {{eps}} }

The operation {{eps}} reverses the direction of a particle by reversing the world. 

\begin{itemize}
\item Grammar
%subcode{bnf} include main
% Values, v = () | (v, v) | L v | R v | -v
% Isomorphisms, iso &=& ... | eta | eps
% Combinators, c &=& iso | c (;) c | c(+)c | c (*) c 

\item
Type judgement.
%subcode{opsem} include main
% eta &: 0 <-> (-b) + b :& eps

%subcode{proof} include main
%@ |- v : b
%@@ |- -v : -b



\item
Operational Semantics.
%subcode{opsem} include main
% <P; F; eps (R v); D>      &|-->& <F; P; (L (-v))~ eps; D{dagger}>

Note: there is NO reduction rule for {{eta}}. 

\item
The adjoint of a Dump is defined to be:
%subcode{opsem} include main
% []{dagger} '= []
% ((P, [](+)c, F):D){dagger} '= (F, [](+)c{dagger}, P):D{dagger}
% ((P, c(+)[], F):D){dagger} '= (F, c{dagger}(+)[], P):D{dagger}
% ((P, [](*)c v, F):D){dagger} '= (F, [](*)v c, P):D{dagger}
% ((P, [](*)v c, F):D){dagger} '= (F, [](*)c v, P):D{dagger}
% ((P, c v(*)[], F):D){dagger} '= (F, v c(*)[], P):D{dagger}
% ((P, v c(*)[], F):D){dagger} '= (F, c v(*)[], P):D{dagger}

\end{itemize}


%% %subcode{bnf} include main
%% % Expressions, e &=& p -> e | let p = F p in e | let p = p in e | p
%% % Patterns, p = () | x | (p, p) | (p|p)
%% %
%% % Type Environments, G &=& empty | x : t | G * G | G + G
%% % Type Contexts, C &=& [] | C * G | G * C | G + C | C + G
%% %

%% \subsection{Type System}
%% %subcode{proof} include main
%% %@ p ==> G
%% %@@ C |- p : C[G] :: positive pattern
%% %
%% %@ p ==> G1 
%% %@ G <-> C[G1]
%% %@@ G |- p : C :: negative pattern
%% %
%% %---
%% %
%% %@ [] |- p : G 
%% %@ G |- e : []
%% %@ p ==> G1
%% %@ e ==> G2
%% %@@ |- p -> e : G1 <-> G2


%% %subcode{code}
%% %! language = Haskell
%% %(x, (y | z)) => 
%% %  let a = (x, y) in 
%% %  (a | (x, z))


%% %subcode{code}
%% %! language = Haskell
%% %G = [] 
%% %(x, (y | z)) => 
%% %  G = (x:t1, (y:t2 | z:t3))
%% %  G = ((x:t1, y:t2) | (x:t1, z:t3))
%% %  ([] | (x:t1, z:t3)) {
%% %    G = (x:t1, y:t2)
%% %    let a = (x, y) in 
%% %    G = a:(t1 *t2)
%% %  }
%% %  G = ((a:(t1 *t2) | (x:t1, z:t3))
%% %  (a | (x, z))
%% %  G = []


%% \bibliographystyle{alpha} 
%% \bibliography{p}

\end{document}
