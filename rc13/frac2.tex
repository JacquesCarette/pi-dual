\documentclass[preprint]{sigplanconf}
\usepackage{url}
\usepackage{proof}
\usepackage{amssymb}
\usepackage{stmaryrd}
\usepackage{listings}
\usepackage{graphicx}
\usepackage{comment}
\usepackage{multicol}

\newcommand{\dgm}[2][1.5]{
\begin{center}
\scalebox{#1}{
\includegraphics{diagrams/#2.pdf}
}
\end{center}
}

\newtheorem{definition}{Definition}
\newcommand{\bra}[1]{\langle #1 |}
\newcommand{\singleton}[1]{\textsc{Sing}~#1}

%subcode-inline{bnf-inline} name Pi
%! swap+ = \mathit{swap}^+
%! swap* = \mathit{swap}^*
%! dagger =  ^{\dagger}
%! assocl+ = \mathit{assocl}^+
%! assocr+ = \mathit{assocr}^+
%! assocl* = \mathit{assocl}^*
%! assocr* = \mathit{assocr}^*
%! identr* = \mathit{uniti}
%! identl* = \mathit{unite}
%! dist = \mathit{distrib}
%! factor = \mathit{factor}
%! (o) = \fatsemi
%! (;) = \fatsemi
%! (*) = \times
%! (+) = +
%! foldB = fold_B
%! unfoldB = unfold_B
%! foldN = fold_N
%! unfoldN = unfold_N
%! trace+ = \mathit{trace}^{+}
%! trace* = \mathit{trace}^{\times}
%! :-* = \multimap
%! :-+ = \multimap^{+}
%! emptyset = \emptyset

%subcode-inline{bnf-inline} regex \{\{(((\}[^\}])|[^\}])*)\}\} name main include Pi
%! [^ = \ulcorner
%! ^] = \urcorner
%! [v = \llcorner
%! v] = \lrcorner
%! [[ = \llbracket
%! ]] = \rrbracket
%! ^^^ = ^{\dagger}
%! eta* = \eta
%! eps* = \epsilon
%! Union = \bigcup
%! in = \in
%! |-->* = \mapsto^{*}
%! |-->> = \mapsto_{\ggg}
%! |-->let = \mapsto_{let}
%! |--> = \mapsto
%! <--| = \mapsfrom
%! |- = \vdash
%! bra = \bra
%! singleton = \singleton
%! <=> = \Longleftrightarrow
%! <-> = \leftrightarrow
%! -> = \rightarrow
%! ~> = \leadsto
%! ::= = ::=
%! /= = \neq
%! vi = v_i
%! di = d_i
%! si = s_i
%! sj = s_j
%! F = \texttt{F}
%! T = \texttt{T}
%! forall = \forall
%! exists = \exists
%! empty = \emptyset
%! Sigma = \Sigma
%! eta = \eta
%! where = \textbf{where}
%! epsilon = \varepsilon
%! least = \phi
%! loop+ = loop_{+}
%! loop* = loop_{\times}
%! CatC = {\mathcal C}
%! CatA = {\mathcal A}
%! gamma = \gamma
%! {[ = \{
%! ]} = \}
%! elem = \in
%! dagger = ^\dagger
%! alpha = \alpha
%! beta = \beta
%! rho = \rho
%! @@ = \mu
%! @ = \,@\,
%! Pow = \mathcal{P}
%! Pi = \Pi
%! PiT = \Pi^{o}
%! PiEE = \Pi^{\eta\epsilon}_{+}
%! PiT = \Pi^{o}
%! PiTF = \Pi^{/}
%! bullet = \bullet
%! * = \times

%%%%%%%%%%%%%%%%%%%%%%%%%%%%%%%%%%%%%%%%%%%%%%%%%%%%%%%%%%%%%%%%%%%%%%%%%%%%%

\begin{document}

\conferenceinfo{ICFP '13}{date, City.} 
\copyrightyear{2013}
\copyrightdata{[to be supplied]} 

\titlebanner{}
\preprintfooter{}

\title{Reversible Higher-Order Programs \\ (Functional Pearl)}
\authorinfo{Jacques Carette \and Amr Sabry}
           {McMaster University \and Indiana University}
           {carette@mcmaster.ca \and sabry@indiana.edu}

\maketitle

%%%%%%%%%%%%%%%%%%%%%%%%%%%%%%%%%%%%%%%%%%%%%%%%%%%%%%%%%%%%%%%%%%%%%%%%%%%%%
\begin{abstract}
  In previous work, we developed a \emph{first-order},
  information-preserving, and reversible programming language {{Pi}} founded
  on type isomorphisms. In subsequent work, we extended this language with
  \emph{fractional types}, the dual of conventional product types which, from
  an information-theoretic point of view, corresponds to the type of values
  with \emph{negative information}. The extended language was given an
  operational interpretation that hinted at two facts: fractional types can
  be used to define first-class pattern-matching clauses as well as
  higher-order reversible functions. Neither of these observations was backed
  by an appropriate denotational model, however. In this paper, we develop
  such a model by interpreting fractional types using functions mapping
  singleton types to the unit type. This interpretation requires that nearly
  all of the type constructs in {{Pi}} be re-interpreted as \emph{dependent
    types}. The framework, formalized in Agda, takes steps towards developing
  a computational interpretation of homotopy type theory using higher-order
  reversible languages.
\end{abstract}

%%%%%%%%%%%%%%%%%%%%%%%%%%%%%%%%%%%%%%%%%%%%%%%%%%%%%%%%%%%%%%%%%%%%%%%%%%%%%
\section{Introduction} 

%%%%%%%%%%%%%%%%%%%%%%%%%%%%%%%%%%%%%%%%%%%%%%%%%%%%%%%%%%%%%%%%%%%%%%%%%%%%%
\section{Background: {{Pi}} and {{PiTF}} }
\label{sec:pi}

We review our language {{Pi}} providing the necessary background and
context for our higher-order extension.\footnote{The presentation in this
  section focuses on the simplest version of {{Pi}}. Other versions
  include the empty type, recursive types, and trace operators but these
  extensions are orthogonal to the higher-order extension emphasized in this
  paper.} The terms of {{Pi}} are not classical values and functions;
rather, the terms are isomorphism witnesses.  In other words, the terms of
{{Pi}} are proofs that certain ``shapes of values'' are isomorphic.
And, in classical Curry-Howard fashion, our operational semantics shows how
these proofs can be directly interpreted as actions on ordinary values which
effect this shape transformation. Of course, ``shapes of values'' are very
familiar already: they are usually called \emph{types}.  But frequently one
designs a type system as a method of classifying terms, with the eventual
purpose to show that certain properties of well-typed terms hold, such as
safety.  Our approach is different: we start from a type system, and then
present a term language which naturally inhabits these types, along with an
appropriate operational semantics.

\paragraph*{Data.}
We view {{Pi}} as having two levels:  it has traditional values, given by:
%subcode{bnf} include main
% values, v ::= () | left v | right v | (v, v)

\noindent and these are classified by ordinary types:
%subcode{bnf} include main
% value types, b ::= 1 | b + b | b * b 

\noindent 
Types include the unit type {{1}}, sum types {{b1+b2}}, and product types
{{b1*b2}}.  Values include {{()}} which is the only value of type {{1}},
{{left v}} and {{right v}} which inject~{{v}} into a sum type, and
{{(v1,v2)}} which builds a value of product type.  But these values should be
regarded as largely ancillary: we do not treat them as first-class citizens,
and they only occur when observing the effect of an isomorphism.

\paragraph*{Isomorphisms.} The terms of {{Pi}} witness
type isomorphisms of the form {{b <-> b}}. They consist of base
isomorphisms, as defined below, and their composition.
%subcode{bnf} include main
%! columnStyle = r@{\hspace{-0.5pt}}r c l@{\hspace{-0.5pt}}l
%swap+ :&  b1 + b2 & <-> & b2 + b1 &: swap+
%assocl+ :&  b1 + (b2 + b3) & <-> & (b1 + b2) + b3 &: assocr+
%identl* :&  1 * b & <-> & b &: identr*
%swap* :&  b1 * b2 & <-> & b2 * b1 &: swap*
%assocl* :&  b1 * (b2 * b3) & <-> & (b1 * b2) * b3 &: assocr*
%dist :&~ (b1 + b2) * b3 & <-> & (b1 * b3) + (b2 * b3)~ &: factor

\noindent Each line of the above table introduces a pair of dual
constants\footnote{where {{swap*}} and {{swap+}} are self-dual.} that witness
the type isomorphism in the middle.  These are the base (non-reducible) terms
of the second, principal level of {{Pi}}. Note how the above has two
readings: first as a set of typing relations for a set of constants. Second,
if these axioms are seen as universally quantified, orientable statements,
they also induce transformations of the (traditional) values. The
(categorical) intuition here is that these axioms have computational content
because they witness isomorphisms rather than merely stating an extensional
equality.

The isomorphisms are extended to form a congruence relation by adding the
following constructors that witness equivalence and compatible closure:

%subcode{proof} include main
%@  ~
%@@ id : b <-> b 
%
%@ c : b1 <-> b2
%@@ sym c : b2 <-> b1
%
%@ c1 : b1 <-> b2
%@ c2 : b2 <-> b3
%@@ c1(;)c2 : b1 <-> b3
%---
%@ c1 : b1 <-> b3
%@ c2 : b2 <-> b4
%@@ c1 (+) c2 : b1 + b2 <-> b3 + b4
%
%@ c1 : b1 <-> b3
%@ c2 : b2 <-> b4
%@@ c1 (*) c2 : b1 * b2 <-> b3 * b4
\noindent The syntax is overloaded: we use the same symbol at the value-type level
and at the isomorphism-type level for denoting sums and products.  Hopefully
this will not cause undue confusion.

It is important to note that ``values'' and ``isomorphisms'' are completely
separate syntactic categories which do not intermix. The semantics of the
language come when these are made to interact at the ``top level'' via
\emph{application}: 
subcode{bnf} include main
top level term, l ::= c v

\noindent
To summarize, the syntax of {{Pi}} is given as follows. 

\begin{definition}{(Syntax of {{Pi}})}
\label{def:Pi}
%subcode{bnf} include main
% value types, b ::= 1 | b+b | b*b 
% values, v ::= () | left v | right v | (v,v) 
%
% iso.~types, t ::= b <-> b
% base iso ::= swap+ | assocl+ | assocr+ 
%     &|& unite | uniti | swap* | assocl* | assocr* 
%     &|& dist | factor 
% iso comb., c ::= iso | id | sym c | c (;) c | c (+) c | c (*) c 
% top level term, l ::= c v
\end{definition}

The language presented above, at the type level, models a commutative ringoid
where the multiplicative structure forms a commutative monoid, but the
additive structure is just a commutative semigroup.  Note that the version of
{{Pi}} that includes the empty type with its usual laws exactly captures, at
the type level, the notion of a \emph{semiring} (occasionally called a
\emph{rig}) where we replace equality by isomorphism.  Semantically, {{Pi}}
models a \emph{bimonoidal category} whose simplest example is the category of
finite sets and bijections. In that interpretation, each value type denotes a
finite set of a size calculated by viewing the types as natural numbers and
each combinator {{c : b1 <-> b2}} denotes a bijection between the sets
denoted by~{{b1}} and~{{b2}}. (We discuss the operational semantics in
conjuction with our extension in the next section.) 

Operationally, the semantics of {{Pi}} is given using two mutually recursive
interpreters: one going forward and one going backwards. The use of {{sym}}
switches control from one evaluator to the other.

%%%%%%%%%%%%%%%%%%%%%%%%%%%%%%%%%%%%%%%%%%%%%%%%%%%%%%%%%%%%%%%%%%%%%%%%%%%%%
\section{Fractional Types} 

For every value {{v : b}}, we introduce the \emph{singleton} type 
{{ singleton{v} }} whose only element is the value {{v}}. 


For every value {{v : b}}, we define the \emph{pattern} 
{{ bra{v} : singleton{v}  1}} which
matches exactly {{v}} and nothing else. The type of this pattern is a function
which

%%%%%%%%%%%%%%%%%%%%%%%%%%%%%%%%%%%%%%%%%%%%%%%%%%%%%%%%%%%%%%%%%%%%%%%%%%%%%
\section{Conclusion}
\label{sec:conc} 

%%%%%%%%%%%%%%%%%%%%%%%%%%%%%%%%%%%%%%%%%%%%%%%%%%%%%%%%%%%%%%%%%%%%%%%%%%%%%

\bibliographystyle{abbrvnat}
\bibliography{cites}

\end{document}

%%%%%%%%%%%%%%%%%%%%%%%%%%%%%%%%%%%%%%%%%%%%%%%%%%%%%%%%%%%%%%%%%%%%%%%%%%%%%
%%%%%%%%%%%%%%%%%%%%%%%%%%%%%%%%%%%%%%%%%%%%%%%%%%%%%%%%%%%%%%%%%%%%%%%%%%%%%

