\documentclass[preprint]{sigplanconf}
\usepackage{url}
\usepackage{proof}
\usepackage{amssymb}
\usepackage{stmaryrd}
\usepackage{listings}
\usepackage{graphicx}
\usepackage{comment}
\usepackage{multicol}

\newcommand{\dgm}[2][1.5]{
\begin{center}
\scalebox{#1}{
\includegraphics{diagrams/#2.pdf}
}
\end{center}
}

\newtheorem{definition}{Definition}
\newcommand{\bra}[1]{\langle #1 |}
\newcommand{\singleton}[1]{\textsc{Sing}~#1}

%subcode-inline{bnf-inline} name Pi
%! swap+ = \mathit{swap}^+
%! swap* = \mathit{swap}^*
%! dagger =  ^{\dagger}
%! assocl+ = \mathit{assocl}^+
%! assocr+ = \mathit{assocr}^+
%! assocl* = \mathit{assocl}^*
%! assocr* = \mathit{assocr}^*
%! identr* = \mathit{uniti}
%! identl* = \mathit{unite}
%! dist = \mathit{distrib}
%! factor = \mathit{factor}
%! (o) = \fatsemi
%! (;) = \fatsemi
%! (*) = \times
%! (+) = +
%! foldB = fold_B
%! unfoldB = unfold_B
%! foldN = fold_N
%! unfoldN = unfold_N
%! trace+ = \mathit{trace}^{+}
%! trace* = \mathit{trace}^{\times}
%! :-* = \multimap
%! :-+ = \multimap^{+}
%! emptyset = \emptyset

%subcode-inline{bnf-inline} regex \{\{(((\}[^\}])|[^\}])*)\}\} name main include Pi
%! [^ = \ulcorner
%! ^] = \urcorner
%! [v = \llcorner
%! v] = \lrcorner
%! [[ = \llbracket
%! ]] = \rrbracket
%! ^^^ = ^{\dagger}
%! eta* = \eta
%! eps* = \epsilon
%! Union = \bigcup
%! in = \in
%! |-->* = \mapsto^{*}
%! |-->> = \mapsto_{\ggg}
%! |-->let = \mapsto_{let}
%! |--> = \mapsto
%! <--| = \mapsfrom
%! |- = \vdash
%! bra = \bra
%! singleton = \singleton
%! <=> = \Longleftrightarrow
%! <-> = \leftrightarrow
%! -> = \rightarrow
%! ~> = \leadsto
%! ::= = ::=
%! /= = \neq
%! vi = v_i
%! di = d_i
%! si = s_i
%! sj = s_j
%! F = \texttt{F}
%! T = \texttt{T}
%! forall = \forall
%! exists = \exists
%! empty = \emptyset
%! Sigma = \Sigma
%! eta = \eta
%! where = \textbf{where}
%! epsilon = \varepsilon
%! least = \phi
%! loop+ = loop_{+}
%! loop* = loop_{\times}
%! CatC = {\mathcal C}
%! CatA = {\mathcal A}
%! gamma = \gamma
%! {[ = \{
%! ]} = \}
%! elem = \in
%! dagger = ^\dagger
%! alpha = \alpha
%! beta = \beta
%! rho = \rho
%! @@ = \mu
%! @ = \,@\,
%! Pow = \mathcal{P}
%! Pi = \Pi
%! PiT = \Pi^{o}
%! PiEE = \Pi^{\eta\epsilon}_{+}
%! PiT = \Pi^{o}
%! PiTF = \Pi^{/}
%! bullet = \bullet
%! * = \times

%%%%%%%%%%%%%%%%%%%%%%%%%%%%%%%%%%%%%%%%%%%%%%%%%%%%%%%%%%%%%%%%%%%%%%%%%%%%%

\begin{document}

\conferenceinfo{ICFP '13}{date, City.} 
\copyrightyear{2013}
\copyrightdata{[to be supplied]} 

\titlebanner{}
\preprintfooter{}

\title{Reversible Higher-Order Programs \\ (Functional Pearl)}
\authorinfo{Jacques Carette \and Amr Sabry}
           {McMaster University \and Indiana University}
           {carette@mcmaster.ca \and sabry@indiana.edu}

\maketitle

%%%%%%%%%%%%%%%%%%%%%%%%%%%%%%%%%%%%%%%%%%%%%%%%%%%%%%%%%%%%%%%%%%%%%%%%%%%%%
\begin{abstract}
  In previous work, we developed a \emph{first-order},
  information-preserving, and reversible programming language {{Pi}} founded
  on type isomorphisms. In subsequent work, we extended this language with
  \emph{fractional types}, the dual of conventional product types which, from
  an information-theoretic point of view, corresponds to the type of values
  with \emph{negative information}. The extended language was given an
  operational interpretation that hinted at two facts: fractional types can
  be used to define first-class pattern-matching clauses as well as
  higher-order reversible functions. Neither of these observations was backed
  by an appropriate denotational model, however. In this paper, we develop
  such a model by interpreting fractional types using functions mapping
  singleton types to the unit type. This interpretation requires that nearly
  all of the type constructs in {{Pi}} be re-interpreted as \emph{dependent
    types}. The framework, formalized in Agda, takes steps towards developing
  a computational interpretation of homotopy type theory using higher-order
  reversible languages.
\end{abstract}

%%%%%%%%%%%%%%%%%%%%%%%%%%%%%%%%%%%%%%%%%%%%%%%%%%%%%%%%%%%%%%%%%%%%%%%%%%%%%
\section{X}

For every value {{v : b}}, we introduce the \emph{singleton} type 
{{ singleton{v} }} whose only element is the value {{v}}. 

For every value {{v : b}}, we define the \emph{pattern} {{ bra{v} :
    singleton{v} 1}} which matches exactly {{v}} and nothing else. The type
of this pattern is a function which

%%%%%%%%%%%%%%%%%%%%%%%%%%%%%%%%%%%%%%%%%%%%%%%%%%%%%%%%%%%%%%%%%%%%%%%%%%%%%

\bibliographystyle{abbrvnat}
\bibliography{cites}

\end{document}

%%%%%%%%%%%%%%%%%%%%%%%%%%%%%%%%%%%%%%%%%%%%%%%%%%%%%%%%%%%%%%%%%%%%%%%%%%%%%
%%%%%%%%%%%%%%%%%%%%%%%%%%%%%%%%%%%%%%%%%%%%%%%%%%%%%%%%%%%%%%%%%%%%%%%%%%%%%

