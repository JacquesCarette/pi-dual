% Created 2010-09-18 Sat 17:30

\documentclass[svgnames,11pt]{beamer}

%\usepackage{fullpage}
\usepackage{hyperref}
\usepackage{graphicx}
\usepackage{longtable}
\usepackage{amsmath}
\usepackage{mdwlist}
\usepackage{txfonts}
\usepackage{xspace}
\usepackage{amstext}
\usepackage{amssymb}
\usepackage{stmaryrd}
\usepackage{proof}
\usepackage{multicol}
\usepackage[colon]{natbib}
\usepackage[nodayofweek]{datetime}
\usepackage{etex}
\usepackage[all, cmtip]{xy}



%%%%%%%%%%%%%%%%%%%%%%%%%%%%%%%%%%%%%%%%%%%%%%%%%%%%%%%%%%%%%%%%%%%%%%%%%%%%%
\newcommand{\xcomment}[2]{\textbf{#1:~\textsl{#2}}}
\newcommand{\amr}[1]{\xcomment{Amr}{#1}}
\newcommand{\roshan}[1]{\xcomment{Roshan}{#1}}

\newcommand{\arrow}[1]{{\color{blue}{#1}}}
\newcommand{\red}[1]{{\color{red}{#1}}}
\newcommand{\timesarr}{\otimes}
\newcommand{\plusarr}{\oplus}
\newcommand{\lcal}{\ensuremath{\lambda}-calculus\xspace}

\def\newblock{}


\newenvironment{floatrule}
    {\hrule width \hsize height .33pt \vspace{.5pc}}
    {\par\addvspace{.5pc}}



%%%%%%%%%%%%%%%%%%%%%%%%%%%%%%%%%%%%%%%%%%%%%%%%%%%%%%%%%%%%%%%%%%%%%%%%%%%%%

%subcode-inline{bnf-inline} name langRev
%! swap+ = \mathit{swap}^+
%! swap* = \mathit{swap}^*
%! dagger =  ^{\dagger}
%! assocl+ = \mathit{assocl}^+
%! assocr+ = \mathit{assocr}^+
%! assocl* = \mathit{assocl}^*
%! assocr* = \mathit{assocr}^*
%! identr* = \mathit{uniti}
%! identl* = \mathit{unite}
%! identr+ = \mathit{zeroi}
%! identl+ = \mathit{zeroe}
%! dist = \mathit{distrib}
%! factor = \mathit{factor}
%! eta = \eta
%! eps = \epsilon
%! eta+ = \eta^+
%! eps+ = \epsilon^+
%! eta* = \eta^{\times}
%! eps* = \epsilon^{\times}
%! (o) = \fatsemi
%! (;) = \fatsemi
%! (*) = \times
%! (+) = +


%subcode-inline{bnf-inline} name langArr
%! arr = \arrow{arr}
%! >>> = ~\arrow{\ggg\xspace}~
%! *** = ~\arrow{\otimes}~
%! +++ = ~\arrow{\oplus}~
%! create = \arrow{create}
%! erase = \arrow{erase}
%! afirst = \arrow{first}
%! asecond = \arrow{second}
%! aleft = \arrow{left}
%! atrace = \arrow{traceA}
%! fstA = \arrow{fstA}
%! sndA = \arrow{sndA}
%! leftA = \arrow{leftA}
%! rightA = \arrow{rightA}
%! |-->>* = \mapsto_{\mathsf{ML}}^{\ast}
%! |-->> = \mapsto_{\mathsf{ML}}


%subcode-inline{bnf-inline} regex \{\{(((\}[^\}])|[^\}])*)\}\} name main include langRev, langArr
%! Gx = \Gamma^{\times}
%! G = \Gamma
%! |-->* = \mapsto^{\ast}
%! |-->> = \mapsto_{\ggg}
%! |-->let = \mapsto_{let}
%! |--> = \mapsto
%! |- = \vdash
%! ==> = \Longrightarrow
%! <=> = \Longleftrightarrow
%! <--> = \rightleftharpoons
%! <-> = \leftrightarrow
%! ~~> = \rightharpoonup
%! ~> = \leadsto
%! ::= = &::=&
%! /= = \neq
%! trans = \mathcal{T}
%! trans1 = \mathcal{T}_1
%! trans2 = \mathcal{T}_2
%! forall = \forall
%! exists = \exists
%! empty = \epsilon
%! least = \phi
%! {[ = \{
%! ]} = \}
%! bottom = \bot
%! alpha = \alpha
%! beta = \beta
%! rho = \rho
%! dagger = ^\dagger
%! @@ = \mu
%! STLC = \lambda^{\rightarrow}
%! STLClet = \textsf{LET}
%! STLCfor = \textsf{LET}^{o}
%! langArr = \textsf{ML}_{\Pi}
%! langArrT = \textsf{ML}_{\Pi^o}
%! Pi = \Pi
%! Pi0 = \Pi^{o}
%! PiEE = \Pi^{\eta\epsilon}
%! if = \mathbf{if}
%! do = \mathbf{do}
%! then = \mathbf{then}
%! else = \mathbf{else}
%! elem = \in
%! * = \times

\newcommand{\dgm}[2][1.5]{
\begin{center}
\scalebox{#1}{
\includegraphics{../diagrams/thesis/#2.pdf}
}
\end{center}
}

\newcommand{\iso}[2][1.5]{
\begin{center}
\scalebox{#1}{
\includegraphics{diagrams/#2.pdf}
}
\end{center}
}

\newcommand{\rat}[2][1.0]{
\begin{center}
\scalebox{#1}{
\includegraphics{../diagrams/#2.pdf}
}
\end{center}
}


\usetheme{CambridgeUS}
%\usetheme{Boadilla}
%\usecolortheme{crane}

\title{Isomorphic Interpreters from Logically Reversible Abstract Machines}
% \subtitle{Work done with Amr Sabry}
\author{Roshan P. James and Amr Sabry}
\institute[IU]{
  School of Informatics and Computing \\
  Indiana University.\\
  \texttt{rpjames@indiana.edu},
  \texttt{sabry@indiana.edu}
}

\date{\today}

\begin{document}



%%%%%%%%%%%%%%%%%%%%%%%%%%%%%%%%%%%%%%%%%%%%%%%%%%%%%%%%%%%%%%%%%%%%%%%%%
\maketitle


%%%%%%%%%%%%%%%%%%%%%%%%%%%%%%%%%%%%%%%%%%%%%%%%%%%%%%%%%%%%%%%%%%%%%%%%%
\begin{frame}{Context}

  \begin{enumerate}
  \item Dagger Traced Symmetric Monoidal Categories and Reversible Computation (Wip RC, 2011)
  \item Information Effects (POPL, 2012)
  \end{enumerate}

\vfill
  \begin{itemize}
  \item The computational model {{Pi0}}.
  \item The idea that computation can preserve information/entropy.
  \item Information effects can be encapsulated by  a type-and-effect system. 
  \item Simple semantics, categorical connections, Turing complete.. etc.
  \end{itemize}
\vfill
\end{frame}

%%%%%%%%%%%%%%%%%%%%%%%%%%%%%%%%%%%%%%%%%%%%%%%%%%%%%%%%%%%%%%%%%%%%%%%%%
\begin{frame}{Context}

  \begin{itemize}
  \item {{Pi0}} is logically reversible by construction.  
    \begin{itemize}
    \item Reversibility is not added to an irreversible model.
    \item It does not arise from compilation.
    \item .. nor from tracking history/undo information.
    \item .. nor from application specific predicates on branch and join points. 
    \end{itemize}
  \end{itemize}

  \vfill
  \pause
  But there is a problem:

  \vfill
  \pause 

  \begin{center}
    It's not obvious how to write any interesting programs in {{Pi0}}. 
  \end{center}
  \vfill

\end{frame}

%%%%%%%%%%%%%%%%%%%%%%%%%%%%%%%%%%%%%%%%%%%%%%%%%%%%%%%%%%%%%%%%%%%%%%%%%
\begin{frame}{A New Model of Computation}
Here is a simple program.
\vfill  

  \dgm{b-wire}

\vfill
\end{frame}

%%%%%%%%%%%%%%%%%%%%%%%%%%%%%%%%%%%%%%%%%%%%%%%%%%%%%%%%%%%%%%%%%%%%%%%%%
\begin{frame}{A New Model of Computation}
It is well-typed. Its type is {{b <-> b}}.
\vfill  

  \dgm{b-wire}

\vfill
\end{frame}

%%%%%%%%%%%%%%%%%%%%%%%%%%%%%%%%%%%%%%%%%%%%%%%%%%%%%%%%%%%%%%%%%%%%%%%%%
\begin{frame}{A New Model of Computation}
Here is how you run it. 
\vfill  

  \dgm{b-wire-value}

\vfill
\end{frame}

%%%%%%%%%%%%%%%%%%%%%%%%%%%%%%%%%%%%%%%%%%%%%%%%%%%%%%%%%%%%%%%%%%%%%%%%%
\begin{frame}{A New Model of Computation}
Here is another simple program of type {{b1*b2<->b1*b2}}.
\vfill  

  \dgm{product-one-wire}

\vfill
\end{frame}

%%%%%%%%%%%%%%%%%%%%%%%%%%%%%%%%%%%%%%%%%%%%%%%%%%%%%%%%%%%%%%%%%%%%%%%%%
\begin{frame}{A New Model of Computation}
Here is how you run it. 
\vfill  

  \dgm{product-one-wire-value}

\vfill
\end{frame}

%%%%%%%%%%%%%%%%%%%%%%%%%%%%%%%%%%%%%%%%%%%%%%%%%%%%%%%%%%%%%%%%%%%%%%%%%
\begin{frame}{A New Model of Computation}
Here is the same program {{b1*b2<->b1*b2}}.
\vfill  

  \dgm{product-two-wires}

\vfill
\end{frame}

%%%%%%%%%%%%%%%%%%%%%%%%%%%%%%%%%%%%%%%%%%%%%%%%%%%%%%%%%%%%%%%%%%%%%%%%%
\begin{frame}{A New Model of Computation}
Here is how you run it. 
\vfill  

  \dgm{product-two-wires-value}

\vfill
\end{frame}


%%%%%%%%%%%%%%%%%%%%%%%%%%%%%%%%%%%%%%%%%%%%%%%%%%%%%%%%%%%%%%%%%%%%%%%%%
\begin{frame}{A New Model of Computation}
\vfill  

The programs so far were rather boring. 

\vfill
\end{frame}


%%%%%%%%%%%%%%%%%%%%%%%%%%%%%%%%%%%%%%%%%%%%%%%%%%%%%%%%%%%%%%%%%%%%%%%%%
\begin{frame}{A New Model of Computation}
Here is a more interesting program. ({{b1*b2<->b2*b1}})
\vfill  

  \dgm{swap_times}

\vfill
\end{frame}

%%%%%%%%%%%%%%%%%%%%%%%%%%%%%%%%%%%%%%%%%%%%%%%%%%%%%%%%%%%%%%%%%%%%%%%%%
\begin{frame}{A New Model of Computation}
Here is how you run it. ({{b1*b2<->b2*b1}})
\vfill  

  \dgm{swap_times_value}

\vfill
\end{frame}

%%%%%%%%%%%%%%%%%%%%%%%%%%%%%%%%%%%%%%%%%%%%%%%%%%%%%%%%%%%%%%%%%%%%%%%%%
\begin{frame}{A New Model of Computation}
Shuffle some booleans. ({{bool*bool*bool<->bool*bool*bool}})
\vfill  

  \dgm{reverse-3-bit}

\vfill
\end{frame}


%%%%%%%%%%%%%%%%%%%%%%%%%%%%%%%%%%%%%%%%%%%%%%%%%%%%%%%%%%%%%%%%%%%%%%%%%
\begin{frame}{A New Model of Computation}
Something different. ({{b1+b2 <-> b1+b2 }})
\vfill  

  \dgm{sum-two-wires}

\vfill
\end{frame}


%%%%%%%%%%%%%%%%%%%%%%%%%%%%%%%%%%%%%%%%%%%%%%%%%%%%%%%%%%%%%%%%%%%%%%%%%
\begin{frame}{A New Model of Computation}
Let's run it. ({{b1+b2 <-> b1+b2 }})
\vfill  

  \dgm{sum-two-wires-left-value}

\vfill
\end{frame}


%%%%%%%%%%%%%%%%%%%%%%%%%%%%%%%%%%%%%%%%%%%%%%%%%%%%%%%%%%%%%%%%%%%%%%%%%
\begin{frame}{A New Model of Computation}
We can have a value on only one of the two wires. ({{b1+b2 <-> b1+b2 }})
\vfill  

  \dgm{sum-two-wires-left-value}

\vfill
\end{frame}


%%%%%%%%%%%%%%%%%%%%%%%%%%%%%%%%%%%%%%%%%%%%%%%%%%%%%%%%%%%%%%%%%%%%%%%%%
\begin{frame}{A New Model of Computation}
Run it with a value on the other wire. ({{b1+b2 <-> b1+b2 }})
\vfill  

  \dgm{sum-two-wires-right-value}

\vfill
\end{frame}

%%%%%%%%%%%%%%%%%%%%%%%%%%%%%%%%%%%%%%%%%%%%%%%%%%%%%%%%%%%%%%%%%%%%%%%%%
\begin{frame}{A New Model of Computation}
It can be drawn as just one wire. ({{b1+b2 <-> b1+b2 }})
\vfill  

  \dgm{sum-one-wire}

\vfill
\end{frame}


%%%%%%%%%%%%%%%%%%%%%%%%%%%%%%%%%%%%%%%%%%%%%%%%%%%%%%%%%%%%%%%%%%%%%%%%%
\begin{frame}{A New Model of Computation}
Another swapping program. ({{b1+b2 <-> b2+b1 }})
\vfill  

  \dgm{swap_plus}

\vfill
\end{frame}


%%%%%%%%%%%%%%%%%%%%%%%%%%%%%%%%%%%%%%%%%%%%%%%%%%%%%%%%%%%%%%%%%%%%%%%%%
\begin{frame}{A New Model of Computation}
Another swapping program. ({{b1+b2 <-> b2+b1 }})
\vfill  

  \dgm{swap_plus_value}

\vfill
\end{frame}


%%%%%%%%%%%%%%%%%%%%%%%%%%%%%%%%%%%%%%%%%%%%%%%%%%%%%%%%%%%%%%%%%%%%%%%%%
\begin{frame}{A New Model of Computation}
One special triangular box. ({{(b1+b2)*b3<->(b1*b3)+(b2*b3)}})
\vfill  

  \dgm{dist}

\vfill
\end{frame}

%%%%%%%%%%%%%%%%%%%%%%%%%%%%%%%%%%%%%%%%%%%%%%%%%%%%%%%%%%%%%%%%%%%%%%%%%
\begin{frame}{A New Model of Computation}
One possible execution. {{(b1+b2)*b3<->(b1*b3)+(b2*b3)}}
\vfill  

  \dgm{dist-wire-value1}

\vfill
\end{frame}


%%%%%%%%%%%%%%%%%%%%%%%%%%%%%%%%%%%%%%%%%%%%%%%%%%%%%%%%%%%%%%%%%%%%%%%%%
\begin{frame}{A New Model of Computation}
Another possible execution. {{(b1+b2)*b3<->(b1*b3)+(b2+b3)}}
\vfill  

  \dgm{dist-wire-value2}

\vfill
\end{frame}

%%%%%%%%%%%%%%%%%%%%%%%%%%%%%%%%%%%%%%%%%%%%%%%%%%%%%%%%%%%%%%%%%%%%%%%%%
\begin{frame}[fragile]{Values and their Types}
Take {{bool}}, {{true}} and {{false}} as {{1+1}}, {{left()}}, {{right ()}}
\vfill

\dgm{if-c1-c2}

\vfill
\pause
This is a conditional statement, like {{if}}.
\end{frame}

%% %%%%%%%%%%%%%%%%%%%%%%%%%%%%%%%%%%%%%%%%%%%%%%%%%%%%%%%%%%%%%%%%%%%%%%%%%
%% \begin{frame}[fragile]{Values and their Types}
%% The universal reversible Toffoli gate. 
%% \vfill

%% \dgm{toffoli}

%% \vfill
%% Complete for boolean circuits.
%% \end{frame}


%%%%%%%%%%%%%%%%%%%%%%%%%%%%%%%%%%%%%%%%%%%%%%%%%%%%%%%%%%%%%%%%%%%%%%%%%%
\begin{frame}
\frametitle{ Trace }

Loops, Cancellation Law, Categorical Trace 

\dgm{trace_plus}

%subcode{proof} include main
%@ c : b1 + b2 <--> b1 + b3
%@@ trace c : b2 <--> b3  


\end{frame}

%%%%%%%%%%%%%%%%%%%%%%%%%%%%%%%%%%%%%%%%%%%%%%%%%%%%%%%%%%%%%%%%%%%%%%%%%
\begin{frame}
\frametitle{ Trace Example }

\dgm{trace-example0}

\end{frame}

\begin{frame}
\frametitle{ Trace Example }

\dgm{trace-example1}

\end{frame}

\begin{frame}
\frametitle{ Trace Example }

\dgm{trace-example2}

\end{frame}

\begin{frame}
\frametitle{ Trace Example }

\dgm{trace-example3}

\end{frame}

\begin{frame}
\frametitle{ Trace Example }

\dgm{trace-example4}

\end{frame}

\begin{frame}
\frametitle{ Trace Example }

\dgm{trace-example5}

\end{frame}

\begin{frame}
\frametitle{ Trace Example }

\dgm{trace-example6}

\end{frame}

\begin{frame}
\frametitle{ Trace Example }

\dgm{trace-example7}

\end{frame}


%%%%%%%%%%%%%%%%%%%%%%%%%%%%%%%%%%%%%%%%%%%%%%%%%%%%%%%%%%%%%%%%%%%%%%%%%
\begin{frame}{Values and their Types}
So lets talk about the types. Here is one model. 

\vfill
\begin{block}

types = 0, 1, 2, 3, 4, ...   
\end{block}


\vfill
Values that inhabit types:

\begin{block}
  
0 = \{\}

1 = \{o\}

2 = \{o, o\}

3 = \{o, o, o\}
\end{block}

\vfill
\pause
Now it would be convenient to give values names so that we know which
`o' we are talking about.

\end{frame}

%%%%%%%%%%%%%%%%%%%%%%%%%%%%%%%%%%%%%%%%%%%%%%%%%%%%%%%%%%%%%%%%%%%%%%%%%
\begin{frame}[fragile]{Values and their Types}
Hijack some good notation:

\vfill
\begin{block}

  {{ types, b = 0 ~|~ 1 ~|~ b + b ~|~ b * b }}

\end{block}

\vfill
This gives us all the types 0, 1, 2, 3 ... 

\vfill
\pause

\begin{enumerate}
\item 
{{() elem 1}}

\vfill
\pause
\item 
{{(v1, v2) elem b1*b2}} \pause if {{v1 elem b1}} and {{v2 elem b2}}

\vfill
\pause
\item 
{{left v elem b1+b2}} if {{v elem b1}}

{{right v elem b1+b2}} if {{v elem b2}}

\end{enumerate}

\pause

%subcode{proof} include main
%@ ~
%@@ |- () : 1
%
%@ |- v1 : b1
%@ |- v2 : b2
%@@ |- (v1, v2) : b1 * b2
%
%@ |- v : b1
%@@ |- left v  : b1 + b2
%
%@ |- v : b2
%@@ |- right v : b1 + b2  



\end{frame}



%%%%%%%%%%%%%%%%%%%%%%%%%%%%%%%%%%%%%%%%%%%%%%%%%%%%%%%%%%%%%%%%%%%%%%%%%%
\begin{frame}
\frametitle{ Recursive types : Nat }

%subcode{bnf} include main 
%!language = Haskell
% data Nat = Zero | Succ Nat

\begin{block}

  \begin{center}
    {{nat = @@x.(1+x)}}

  \end{center}

\end{block}

%subcode{opsem} include main
%! columnStyle = ||c|c|c||
% {@1@ Nat @} & {@1@ nat @} & {@1@ numeric @}
% Zero & <left ()> & 0 
% Succ Zero & <right <left ()> > & 1
% Succ (Succ Zero) & <right <right <left ()> > > & 2
% ... & ... & ...

\end{frame}

%%%%%%%%%%%%%%%%%%%%%%%%%%%%%%%%%%%%%%%%%%%%%%%%%%%%%%%%%%%%%%%%%%%%%%%%%%
\begin{frame}
\frametitle{ Recursive types : Nat }

%subcode{bnf} include main 
%!language = Haskell
% data Nat = Zero | Succ Nat


  \begin{center}
    {{nat = @@x.(1+x)}}
    
    \vfill

    {{unfold : nat <--> 1 + nat : fold }}
  \end{center}


\vfill

\iso{unfold_nat}

\vfill

\end{frame}

%%%%%%%%%%%%%%%%%%%%%%%%%%%%%%%%%%%%%%%%%%%%%%%%%%%%%%%%%%%%%%%%%%%%%%%%%%
\begin{frame}
\frametitle{ Recursive types : Binary Trees }

%subcode{bnf} include main 
%!language = Haskell
% data Tree = Leaf | Node Tree Tree


  \begin{center}
    {{tree, t = @@x.(1+x*x)}}
    
    \vfill

    {{unfold : t <--> 1 + t*t : fold }}
  \end{center}


\vfill

\iso{unfold_tree}

\vfill

\end{frame}


%%%%%%%%%%%%%%%%%%%%%%%%%%%%%%%%%%%%%%%%%%%%%%%%%%%%%%%%%%%%%%%%%%%%%%%%%
\begin{frame}[fragile]{Values and their Types}

\vfill
\begin{block}

  {{ types, b = 0 ~|~ 1 ~|~ b + b ~|~ b * b ~|~ x ~|~ @@x.b }}

  {{ values, v = () ~|~ (v, v) ~|~ left v ~|~ right v ~|~ <v> }}

\end{block}

\vfill
%subcode{proof} include main
%@ ~
%@@ |- () : 1
%
%@ |- v1 : b1
%@ |- v2 : b2
%@@ |- (v1, v2) : b1 * b2
%----
%@ |- v : b1
%@@ |- left v  : b1 + b2
%
%@ |- v : b2
%@@ |- right v : b1 + b2  
%----
%@ |- v : b[@@x.b/x]
%@@ |- <v>  : @@x.b

\vfill

\end{frame}



%% %%%%%%%%%%%%%%%%%%%%%%%%%%%%%%%%%%%%%%%%%%%%%%%%%%%%%%%%%%%%%%%%%%%%%%%%%
%% \begin{frame}[fragile]{Values and their Types: Isorecursive types}

%% nat = {0, 1, 2, 3... }

%% \begin{multicols}{2}
%% {{data Nat = Zero ~|~ Succ Nat}}

%% {{0 = Zero}}

%% {{1 = Succ Zero}}

%% {{2 = Succ (Succ Zero)}}

%% \columnbreak
%% {{nat = @@x.(1+x)}}

%% {{0 = <left ()>:nat}}

%% {{1 = <right <left ()>>:nat}}

%% {{2 = <right <right <left ()>> > :nat}}

%% \end{multicols}

%% \vfill

%% \iso{unfold_nat}

%% \vfill

%% \end{frame}


%%%%%%%%%%%%%%%%%%%%%%%%%%%%%%%%%%%%%%%%%%%%%%%%%%%%%%%%%%%%%%%%%%%%%%%%%
\begin{frame}[fragile]{Wires}

What are the rules that determine how wires can be drawn? 

\end{frame}

%%%%%%%%%%%%%%%%%%%%%%%%%%%%%%%%%%%%%%%%%%%%%%%%%%%%%%%%%%%%%%%%%%%%%%%%%
\begin{frame}[fragile]{Wires : Monoids}

  \begin{block}{Commutative Monoid: {{(1, *)}} }
    
%subcode{bnf} include main
% 1 * b &<->& b // identity for~ *
% b1 * b2 &<->& b2 * b1 // commutativity for~ *
% b1 * (b2 * b3) &<->& (b1 * b2) * b3 //associativity for~ *

  \end{block}

\pause

\begin{multicols}{2}
\dgm[0.7]{unite}
\dgm[0.7]{uniti}  
\end{multicols}


\begin{multicols}{2}
\dgm[0.7]{swap_times}

\dgm[0.7]{assoc}  
\end{multicols}

\end{frame}

%%%%%%%%%%%%%%%%%%%%%%%%%%%%%%%%%%%%%%%%%%%%%%%%%%%%%%%%%%%%%%%%%%%%%%%%%
\begin{frame}[fragile]{Wires : Two Monoids}


  \begin{block}{Commutative Monoid: {{(1, *)}} }
    
%subcode{bnf} include main
% 1 * b &<->& b // identity for~ *
% b1 * b2 &<->& b2 * b1 // commutativity for~ *
% b1 * (b2 * b3) &<->& (b1 * b2) * b3 //associativity for~ *

  \end{block}

  \begin{block}{Commutative Monoid: {{(0, +)}} }

%subcode{bnf} include main
% 0 + b&<->& b // identity for~ +
% b1 + b2 &<->& b2 + b1 // commutativity for~ +
% b1 + (b2 + b3) &<->& (b1 + b2) + b3 // associativity for~ +

  \end{block}


\end{frame}


%%%%%%%%%%%%%%%%%%%%%%%%%%%%%%%%%%%%%%%%%%%%%%%%%%%%%%%%%%%%%%%%%%%%%%%%%
\begin{frame}[fragile]{ Isomorphisms of Finite Types }


  \begin{block}{ Primitive Combinators }
    
%subcode{bnf} include main
% 1 * b &<->& b // identity for~ *
% b1 * b2 &<->& b2 * b1 // commutativity for~ *
% b1 * (b2 * b3) &<->& (b1 * b2) * b3 //associativity for~ *


%subcode{bnf} include main
% 0 + b&<->& b // identity for~ +
% b1 + b2 &<->& b2 + b1 // commutativity for~ +
% b1 + (b2 + b3) &<->& (b1 + b2) + b3 // associativity for~ +

%subcode{bnf} include main
% 0 *b &<->& 0  // distribute over~ 0
% (b1 + b2) * b3 &<->& (b1 * b3) + (b2 * b3) // distribute over~ +

  \end{block}


\end{frame}


%%%%%%%%%%%%%%%%%%%%%%%%%%%%%%%%%%%%%%%%%%%%%%%%%%%%%%%%%%%%%%%%%%%%%%%%%
\begin{frame}[fragile]{ Rules for Composing Wires }

%% \begin{block}

%%   {{ types, b = 0 ~|~ 1 ~|~ b + b ~|~ b * b }}

%% \end{block}


\dgm{c1c2_seq}

\begin{multicols}{2}
\dgm{c1c2_par_times}

\dgm{c1c2_par_sum}  
\end{multicols}


\begin{block}{ Closure }

%subcode{proof} include main
%@ b1 <-> b2
%@ b2 <-> b3
%@@ b1 <-> b3
%
%@ b1 <-> b3
%@ b2 <-> b4
%@@ (b1 * b2) <-> (b3 * b4)
%
%@ b1 <-> b3
%@ b2 <-> b4
%@@ (b1 + b2) <-> (b3 + b4) 

\end{block}


\end{frame}



%%%%%%%%%%%%%%%%%%%%%%%%%%%%%%%%%%%%%%%%%%%%%%%%%%%%%%%%%%%%%%%%%%%%%%%%%%%%
\begin{frame}
\frametitle{ Type Isomorphisms and Trace : {{Pi0}} }

%\begin{block}{Primitive operators and their composition}

{\scriptsize
%Primitive operators and their composition:


%subcode{bnf} include main
% base types, b ::= 0 | 1 | b+b | b*b | x | @@x.b
% values, v ::= () | left v | right v | (v, v) | <v>



%subcode{opsem} include main
%! columnStyle = rrcll
%
% {@1@ unfold @}:& @@x.b  &<->& b[@@x.b/x] &: {@1@fold @}
%
% {@1@ identl+ @}:&  0 + b &<->& b &: {@1@identr+ @}
% {@1@ swap+ @}:&  b1 + b2 &<->& b2 + b1 &: {@1@swap+ @}
% {@1@ assocl+ @}:&  b1 + (b2 + b3) &<->& (b1 + b2) + b3 &: {@1@assocr+ @}
%
% {@1@ identl* @}:&  1 * b &<->& b &: {@1@identr* @}
% {@1@ swap* @}:&  b1 * b2 &<->& b2 * b1 &: {@1@swap* @}
% {@1@ assocl* @}:&  b1 * (b2 * b3) &<->& (b1 * b2) * b3 &: {@1@assocr* @}
%
% {@1@ dist0 @}:&  0 *b &<->& 0  &: {@1@factor0 @}
% {@1@ dist @}:& (b1 + b2) * b3 &<->& (b1 * b3) + (b2 * b3) &: {@1@factor @}

%subcode{proof} include main
%@  ~
%@@ {@1@ id @}: b <-> b 
%
%@ {@1@ c @}: b1 <-> b2
%@@ {@1@ sym c @}: b2 <-> b1
%
%@ {@1@ c1 @}: b1 <-> b2
%@ {@1@ c2 @}: b2 <-> b3
%@@ {@1@ (c1(;)c2) @}: b1 <-> b3
%---
%@ {@1@ c1 @}: b1 <-> b3
%@ {@1@ c2 @}: b2 <-> b4
%@@ {@1@ (c1 (+) c2) @}: (b1 + b2) <-> (b3 + b4)
%
%@ {@1@ c1 @}: b1 <-> b3
%@ {@1@ c2 @}: b2 <-> b4
%@@ {@1@ (c1 (*) c2) @}: (b1 * b2) <-> (b3 * b4)
%
%@ {@1@ c @}: b1 + b2 <-> b1 + b3
%@@ {@1@ trace c @}: b2 <-> b3
}
%\end{block}


\end{frame}



% %%%%%%%%%%%%%%%%%%%%%%%%%%%%%%%%%%%%%%%%%%%%%%%%%%%%%%%%%%%%%%%%%%%%%%%%%
\begin{frame}
\frametitle{ An Example }

% A programming language emerges:

%% \begin{block}

%% %subcode{bnf} include main 
%% % base types, b ::= 0 | 1 | b+b | b*b | x | @@x.b
%% % combinator types, t ::= b <-> b
%% % values, v ::= () | left v | right v | (v,v) | <v>
%% %
%% % isomorphisms, iso ::= swap+ | assocl+ | assocr+ | zeroe | zeroi
%% %                   &|& swap* | assocl* | assocr* | identl* | identr* 
%% %                   &|& dist | factor | id | fold | unfold
%% % combinators, c ::= iso | sym c | c (;) c | c (*) c | c (+) c | trace c

%% \end{block}

\vfill
{{if = dist(;)((id(*)c1)(+)(id(*)c2))(;)factor}}
\vfill

\dgm{if-c1-c2}


\vfill
\end{frame}

% %%%%%%%%%%%%%%%%%%%%%%%%%%%%%%%%%%%%%%%%%%%%%%%%%%%%%%%%%%%%%%%%%%%%%%%%%
\begin{frame}
\frametitle{ Simple Abstract Machine }

\begin{center}

\vfill

{{n,m = 0 ~|~ n + 1 }}

\vfill


\vfill
\begin{enumerate}
\item Machine states 
\item Start state 
\item Stop state 
\end{enumerate}

\end{center}

\vfill


{{|--> <2, 1>}}
{{|--> <1, 2>}}
{{|--> <0, 3>}}

\vfill
\end{frame}

% %%%%%%%%%%%%%%%%%%%%%%%%%%%%%%%%%%%%%%%%%%%%%%%%%%%%%%%%%%%%%%%%%%%%%%%%%
\begin{frame}{Goal}

Translate the abstract machine to a {{Pi0}} circuit. 

\end{frame}

%%%%%%%%%%%%%%%%%%%%%%%%%%%%%%%%%%%%%%%%%%%%%%%%%%%%%%%%%%%%%%%%%%%%%%%%%%
\begin{frame}{Start by examining the machine states}

\vfill

\begin{center}
  
\end{center}

\vfill
\pause

\iso[0.9]{nat-nat2}

\end{frame}


%%%%%%%%%%%%%%%%%%%%%%%%%%%%%%%%%%%%%%%%%%%%%%%%%%%%%%%%%%%%%%%%%%%%%%%%%%
\begin{frame}{Start by examining the machine states}

\vfill

\begin{center}
  
\end{center}

\vfill

\iso[0.8]{nat-nat3}

\end{frame}


%%%%%%%%%%%%%%%%%%%%%%%%%%%%%%%%%%%%%%%%%%%%%%%%%%%%%%%%%%%%%%%%%%%%%%%%%%
\begin{frame}{Start by examining the machine states}

\vfill

\begin{center}
  
\end{center}

\vfill

\iso[0.8]{nat-nat4}

\end{frame}


%%%%%%%%%%%%%%%%%%%%%%%%%%%%%%%%%%%%%%%%%%%%%%%%%%%%%%%%%%%%%%%%%%%%%%%%%%
\begin{frame}{Reductions maps input states to output states}

\vfill

\begin{center}
  
\end{center}

\vfill
 becomes 

Equations:

{{n' = n-1}}

{{m' = m + 1}}

\vfill

\pause

\iso[0.8]{nat-nat4}

\end{frame}

%%%%%%%%%%%%%%%%%%%%%%%%%%%%%%%%%%%%%%%%%%%%%%%%%%%%%%%%%%%%%%%%%%%%%%%%%%
\begin{frame}{Reductions maps input states to output states}

\vfill

\begin{center}
  
\end{center}

\vfill
 becomes 

Equations:

{{n' = n-1}}

{{m' = m + 1}}

\vfill


\iso[0.8]{nat-nat5}

\end{frame}


%%%%%%%%%%%%%%%%%%%%%%%%%%%%%%%%%%%%%%%%%%%%%%%%%%%%%%%%%%%%%%%%%%%%%%%%%%

\begin{frame}{Start and stop states}

\iso[0.8]{nat-nat5}

\vfill
\begin{enumerate}
\item Start state 
\item Stop state 
\end{enumerate}

\pause

\begin{center}
  We need to expose the start and stop states. 
\end{center}

\end{frame}

%%%%%%%%%%%%%%%%%%%%%%%%%%%%%%%%%%%%%%%%%%%%%%%%%%%%%%%%%%%%%%%%%%%%%%%%%%
\begin{frame}{Start and stop states}

\iso[0.8]{nat-nat5}


\vfill
  \begin{itemize}
  \item Given a start state 
\vfill
  \item Produce an output state 
\vfill
  \item All the other  states are internal, intermediate... 
  \end{itemize}
\vfill

\end{frame}

%%%%%%%%%%%%%%%%%%%%%%%%%%%%%%%%%%%%%%%%%%%%%%%%%%%%%%%%%%%%%%%%%%%%%%%%%%
\begin{frame}{Exposing start and stop states}

\iso[0.75]{nat-nat5}

\pause
\iso[0.75]{nat-nat6}

\end{frame}

%%%%%%%%%%%%%%%%%%%%%%%%%%%%%%%%%%%%%%%%%%%%%%%%%%%%%%%%%%%%%%%%%%%%%%%%%%
\begin{frame}{Exposing start and stop states}

\iso[0.70]{nat-nat6}

\pause
\iso[0.70]{nat-nat7}

\end{frame}

%%%%%%%%%%%%%%%%%%%%%%%%%%%%%%%%%%%%%%%%%%%%%%%%%%%%%%%%%%%%%%%%%%%%%%%%%%
\begin{frame}{Exposing start and stop states}

\iso[0.75]{nat-nat7}

\pause
\iso[0.75]{nat-nat8}

\end{frame}

%%%%%%%%%%%%%%%%%%%%%%%%%%%%%%%%%%%%%%%%%%%%%%%%%%%%%%%%%%%%%%%%%%%%%%%%%%
\begin{frame}{Exposing start and stop states}

\iso[0.9]{nat-nat8}

\end{frame}



%%%%%%%%%%%%%%%%%%%%%%%%%%%%%%%%%%%%%%%%%%%%%%%%%%%%%%%%%%%%%%%%%%%%%%%%%%%%
\begin{frame}
\frametitle{Compile Abstract Machines to {{Pi0}} }
For example, a meta-circular interpreter for {{Pi0}}

  \begin{scriptsize}
    
%subcode{bnf} include main
% Combinators, c = iso | c (;) c | c (*) c | c (+) c | trace c
% Combinator Contexts, cc = [] | Fst cc c | Snd c cc 
%                  &|& LeftTimes cc c v | RightTimes c v cc 
%                  &|& LeftPlus cc c | RightPlus c cc | Trace cc
% Values, v = () | (v, v) | L v | R v

%
% Machine states = <c, v, cc> | {[c, v, cc]}
% Start state = <c, v, []> 
% Stop State = {[c, v, []]}


%subcode{opsem} include main
%! columnStyle = rclr
% <iso, v, cc> &|-->& {[iso, iso(v), cc]} &~~~~~~~~~~~~ rule 1
% <c1(;)c2, v, cc> &|-->& <c1, v, Fst cc c2> &~~~~~~~~~~~~ rule 2
% {[c1, v, Fst cc c2]} &|-->& <c2, v, Snd c1 cc> &~~~~~~~~~~~~ rule 3
% {[c2, v, Snd c1 cc]} &|-->& {[ c1(;)c2, v, cc ]} &~~~~~~~~~~~~ rule 4
% <c1(+)c2, L v, cc> &|-->& <c1, v, LeftPlus cc c2> &~~~~~~~~~~~~ rule 5
% {[ c1, v, LeftPlus cc c2 ]} &|-->& {[c1 (+) c2, L v, cc ]} &~~~~~~~~~~~~ rule 6
% <c1(+)c2, R v, cc> &|-->& <c2, v, RightPlus c1 cc> &~~~~~~~~~~~~ rule 7
% {[ c2, v, RightPlus c1 cc ]} &|-->& {[c1 (+) c2, R v, cc ]} &~~~~~~~~~~~~ rule 8
% <c1(*)c2, (v1, v2), cc> &|-->& <c1, v1, LeftTimes cc c2 v2> &~~~~~~~~~~~~ rule 9
% {[ c1, v1, LeftTimes cc c2 v2 ]} &|-->& <c2, v2, RightTimes c1 v1 cc> &~~~~~~~~~~~~ rule 10
% {[ c2, v2, RightTimes c1 v1 cc ]} &|-->& {[ c1 (*) c2, (v1, v2), cc ]} &~~~~~~~~~~~~ rule 11
% < trace c, v, cc> &|-->& <c, R v, Trace cc> &~~~~~~~~~~~~ rule 12
% {[c, L v, Trace cc]} &|-->& <c, L v, Trace cc> &~~~~~~~~~~~~ rule 13
% {[c, R v, Trace cc]} &|-->& {[trace c, R v, cc]} &~~~~~~~~~~~~ rule 14

  \end{scriptsize}


\end{frame}


%%%%%%%%%%%%%%%%%%%%%%%%%%%%%%%%%%%%%%%%%%%%%%%%%%%%%%%%%%%%%%%%%%%%%%%%%%
\begin{frame}{ {{Pi0}} examples : Metacircular Evaluator}

\iso[0.7]{pi0-2}

\begin{scriptsize}
  (See ``Isomorphic Interpreters'', RC 2012)
\end{scriptsize}
\end{frame}




\end{document}
