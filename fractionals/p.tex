
Reference cells have fractional types

1 -> A * 1/A
From no information, create a ref cell with a getter (A) and a setter (1/A)
Usually type of getter is 1 -> A and setter A -> 1
Using that A -> B is [not A + B] and hence [1/A * B] in this case then:
1 -> A is [1/1 * A] which is A, and
A -> 1 is [1/A * 1] which is 1/A

Inspiration from lenses
Particle/Anti-particle
Entanglement (side channel between setter and getter)

Forget 1/0 for now; we will fix later with meadow trick

Also we will index the types with a monad or something to make the effect of creating the cell explicit

Denotational semantics already kind of known: groupoid with fractional cardinality

If A and 1/A are left loose, we get errors (reading uninitialized cells, race conditions, etc.). But if information is conserved and we do meadow trick, no errors! That should be the main theorem.

Can get functions A -> B as [1/A * B] , which has a hidden cell for parameter passing, a setter to pass the parameter and a value of type B that is blocked until the cell is full.

Now do interesting programs. Start with all the constructions for closed monoidal categories.

Check 1/(1/A) = A
1 -> 1/A * 1/(1/A)
Se here we are creating a cell that holds the setter for another cell; ho-functions!!!


===============================================================

Comments on the above:

I am going to use a mixture of 'reversible', 'functional' and 'imperative' thinking below,
and may not always signal it explicitly -- you've been warned!

So a getter is 1 -> A only in an imperative setting. In a functional setting, the 'better' type
is World -> A; and a setter is then World -> A -> World. Which is great, because THAT is
actually exactly the type of a lens! [We could generalize to World' as the result of set,
which would push us into generalized optics]

The 'reasoning' to get to A and 1/A is dubious at best; it uses 'Boolean' Curry-Howard,
which is not reversible. I don't believe in that aspect at all.

However, I do see "Lens S A" as being a kind of 'relative complement', i.e. S // A.
And it does make complete sense to see World as being chopped-up, through allocation, into
A and its 'complement' World // A. Let's just call this 'Rest'.

Here World is some kind of 'potentially infinite' universe where a Lens does not
*necessarily* allocate new memory, but it certainly does reserve memory for our use.
Observationally, these are probably the same.

What do we obtain here? What seems to drop out quite naturally is a flavour of
separation logic!  The axioms of Lens, equivalently of isomorphisms and product, give
us exactly that!

Downside? World is _neither_ 1 or 0.  So 1 / A still doesn't pop out.
Nevertheless, I think this is very interesting.
