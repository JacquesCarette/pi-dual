There are two ways to read types and Curry-Howard: 
\begin{enumerate}
\item One first concentrates on the \emph{types}, as propositions.  We can come up with a ``type system'' which 
reflects a logic.  Then, as a second step, choose the proofs which inhabit these types/propositions, and call these
proofs \emph{terms}.
\item Alternatively, one first concentrates on the \emph{terms}, focusing on their operational semantics as primary.
Then, as a second step, one designs a ``type system'' to classify the terms, so as to be able to prove progress and
preservations theorems, to ensure that well-typed terms ``don't go wrong''.
\end{enumerate}

Under the first interpretation, coming up with an operational interpretation of the terms can be challenging:
witness the ongoing effort to understand aspects of classical logic as a type system for a (programming) calculus.
Similarly, various modal logics as well as linear logic are still being given different interpretations as type
systems for calculi, to different degrees of effectiveness.

The second interpretation is not straightforward either.  The definition of ``doing wrong'' frequently gets 
expanded to covering effects, permissions, memory access, and so forth, which can seriously affect the decidability
properties of the type system (not to mention efficiency when decidability is not an issue).

Even in the first interpretation, the understanding is that terms represented \emph{values}, which are irreducible
terms (proof terms reduce principally via cut elimination, but there are other reduction rules for projections, etc).

We wish to shift the focus and consider \emph{equivalences} as our types.  This will mean, under the first 
interpretation, that our terms will be witnesses for these equivalences.  The underlying reason is that we wish
to examine \emph{reversible computation}.  From an information-theoretic point of view, this means that information
\emph{must be preserved}, although it can certainly be rearranged in many different ways.

In other words, we wish to regard types with shape $T_1\equiv T_2$, where $T_1$ and $T_2$ are usual value-types.
In fact, we want more: we wish to regard these as \emph{oriented}: the eventual induced operational interpretation
will regard equivalence witnesses as inducing a transformation from values of types $T_1$ to values of type $T_2$.
Of course, as these are equivalences, there will always be a term which inhabits $T_2\equiv T_1$ whenever there
is one inhabiting $T_1 \equiv T_2$.  This will be clearly visible when we give rules for terms, as these will always
by symmetric (and will provably induce invertible transformations).

- equational theories
- equations of equational theories
- axioms as constants
- various operators to create ``larger'' equiations from smaller
- (speak about lens as imperfect equivalences)
- algebraic structures as classifiers
- Curry-Howard up a level too -- sums and products of equivalences
