\documentclass{llncs}

%%%%%%%%%%%%%%%%%%%%%%%%%%%%%%%%%%%%%%%%%%%%%%%%%%%%%%%%%%%%%%%%%%%%%%%%%%%%%%
\begin{document}

\title{Permutations}
\titlerunning{Permutations}
\author{Jacque Carette \and Amr Sabry}
\authorrunning{Carette and Sabry}
\institute{? \\
\email{...@...}
\and
...}
\maketitle

\begin{abstract}

\keywords{}
\end{abstract}

%%%%%%%%%%%%%%%%%%%%%%%%%%%%%%%%%%%%%%%%%%%%%%%%%%%%%%%%%%%%%%%%%%%%%%%%%%%%%%
\section{Introduction}

%%%%%%%%%%%%%%%%%%%%%%%%%%%%%%%%%%%%%%%%%%%%%%%%%%%%%%%%%%%%%%%%%%%%%%%%%%%%%%
\section{Common Representations}

Say we want to write the permutation between \verb|Fin 3 x Fin 2| and
\verb|Fin 2 x Fin 3|. First let's enumerate the two sets in some
canonical order:
\begin{verbatim}
Fin 3 x Fin 2 = { (0,0), (0,1), (1,0), (1,1), (2,0), (2,1) } 
Fin 2 x Fin 3 = { (0,0), (0,1), (0,2), (1,0), (1,1), (1,2) }
\end{verbatim}

Then the permutation can be visualized as:

\begin{verbatim} 
Fin 3 x Fin 2 = { (0,0), (0,1), (1,0), (1,1), (2,0), (2,1) }
                    |      |      |      |      |      |
                    |      |      |      +---------+   |
                    |      |      |             |  |   |
                    |      +-------------+      |  |   |
                    |             |      |      |  |   |
                    |      +------+      |      |  |   |
                    |      |             |      |  |   |
                    |      |      +-------------+  |   |
                    |      |      |      |         |   |
                    |      |      |      |      +--+   |
                    |      |      |      |      |      |
Fin 2 x Fin 3 = { (0,0), (0,1), (0,2), (1,0), (1,1), (1,2) }
\end{verbatim}

How to represent this in a way that amenable to efficient and
convenient manipulation. There are several common representations:

\begin{itemize}
\item Truth table
\item Matrix
\item Reed-Muller expansion
\item Cycle expansion
\item Decision diagram
\item Vectors
\item Factorial representation
\item Terms of a programming language
\end{itemize}

As vectors we would get:
\begin{verbatim}
p = [ 0, 3, 1, 4, 2, 5 ]
\end{verbatim}
which says that the item at position \verb|i| goes to the item at
position \verb|p(i)|. 

As a sequence of transpositions, we would get several possible
representations... etc.

Very difficult to work with such representations. Imagine trying to
write big permutations for a reversible adder. Imagine trying to prove
properties of such permutations (bimonoidal structure). Etc.

%%%%%%%%%%%%%%%%%%%%%%%%%%%%%%%%%%%%%%%%%%%%%%%%%%%%%%%%%%%%%%%%%%%%%%%%%%%%%%
\section{Essence}

Our point is that the essence of the permutation above is a type
equivalence that is trivial to prove. Indeed the equivalence between
\verb|A x B| and \verb|B x A| is trivial to prove. We can once and for
all derive |Fin m x Fin n ~= Fin (m * n)| from such a type
equivalence. We can derive the two monoidal structures and their
interactions. Our main theorem is that:
\begin{verbatim}
Fin, ~=, ~~~ is a weak rig groupoid
\end{verbatim}

%%%%%%%%%%%%%%%%%%%%%%%%%%%%%%%%%%%%%%%%%%%%%%%%%%%%%%%%%%%%%%%%%%%%%%%%%%%%%%

\bibliographystyle{acm}
\bibliography{cites}
\end{document}



