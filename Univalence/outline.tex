\documentclass{article}
\usepackage{fullpage}
\usepackage{amssymb}
\begin{document}

\newcommand{\PiL}{\textsc{\textbf{Pi}}}
\newcommand{\PiCalc}{\PiL${}^\equiv$}
\newcommand{\PiCat}{\ensuremath{\Pi\mathbb{C}}}

%%%%%%%%%%%%%%%%%%%%%%%%%%%%%%%%%%%%%%%%%%%%%%%%%%%%%%%%%%%%%%%%%%%%%%%
\section{Goal} 

Our first major milestone is (i) to develop \emph{a sound and complete
  calculus of permutations over finite sets}, (ii) to define, in that
setting, a variant of \emph{univalence} that has a computational
content, and to \emph{prove} this univalence property. In the longer
term, we will want to generalize this framework to sets with negative
and fractional cardinalities which would effectively give us (at
least) higher-order functions. One can in fact go further and consider
sets with more general cardinalities: imaginary numbers, irrational
numbers, all the way to algebraic numbers.

To reach the first major milestone, our detailed steps are:
\begin{itemize}

\item to develop a \emph{syntactic notion of permutations over finite
    sets}. We achieve this by defining a typed programming language
  \PiL\ whose types denote finite sets and whose expressions denote
  permutations over finite sets. This language has been completely
  developed: its syntax, type system, and operational semantics are
  well-understood;

\item to develop a \emph{calculus \PiCalc\ for reasoning about \PiL\
    permutations}. We eventually want to prove that this calculus is
  sound and complete with respect to a semantic notion of permutation
  equivalence;

\item to develop a \emph{semantic notion of equivalence of
    permutations over finite sets}.  To make the connections to HoTT
  more evident, we formalize this equivalence using a category
  (groupoid actually) of finite sets and permutations \PiCat\ which we
  establish is \emph{rig category}, i.e., it has two symmetric
  monoidal structures with the multiplicative one distributing over
  the additive one. The objects of this category are discrete
  groupoids representing finite sets; the morphisms of \PiCat\ are
  permutations over the finite sets; the complete definition of the
  category also needs an equivalence relation that specifies when two
  morphisms should be considered equal. This equivalence relation is
  our notion of semantic equivalence of permutations over finite
  sets. The advantage of using the categorical setting to define the
  semantic notion of equivalence is that the coherence laws for rig
  categories provide us with an axiomatization of semantic equivalence
  of permutations that is rich enough to reason about sequential
  compositions of permutations (given by the plain categorical
  structure), parallel compositions of permutations (given by the
  additive monoidal structure), and tensor compositions of
  permutations (given by the multiplicative monoidal structure) while
  taking care of all the induced equivalences that result from the
  interactions among these structures. 

\item to derive the complete definition of \PiCalc\ by reifying each
  of the primitives used in defining semantic equivalence as syntactic
  objects; a proof of soundness and completeness is then immediate as
  the syntactic and semantic structures are isomorphic by construction.

\item to define a variant of univalence in the current setting we
  reason as follows. In the HoTT setting, which is not limited to
  finite sets and permutations, we have a larger category of arbitrary
  sets and with equivalences as morphisms. Univalence in that context
  postulates that each equivalence between sets induces a path between
  these sets. Translated to our setting, univalence postulates that
  each semantic equivalence between sets induces a syntactic
  equivalence between these sets. In the richer setting of HoTT,
  univalence is postulated because there is no obvious way to reason
  about extensional equivalence of the functions used to define
  semantic equivalences. In our setting, these semantic equivalences
  are defined using permutations which are amenable to equational
  reasoning via the coherence laws. Therefore to make univalence
  constructive, all we need is a computational mechanism to calculate
  a syntactic equivalence from a semantic one. This mechanism is
  embedded in the proof of completeness above. 

\item Our complete development is formalized in Agda. 
\end{itemize}

Which still leaves open one big question: is our main representation
of permutations _fin~_ or CPerm ?  Or do we need both, as one is
'syntactic' (through the use of vectors) while the other is more
abstract (as it uses functions)?

I am quite sure that all the work needed to bring FinEquivCat all the
way to RigCategory is feasible, and not too hard [but will be
enlightening].

To do the same with CPermCat (and SkFinSetCategory) will be a fair bit
harder.  So it is very important for us to decide what the exact path
we want to take.  [Which is indeed what your outline is about!]




%%%%%%%%%%%%%%%%%%%%%%%%%%%%%%%%%%%%%%%%%%%%%%%%%%%%%%%%%%%%%%%%%%%%%%%
\section{Basic Utilities}

\paragraph*{LeqLemmas.} A few lemmas about natural numbers. 

\paragraph*{FinNatLemmas.} A few lemmas about $\texttt{Fin}~n$ which
are the numbers used to index into vectors.

\paragraph*{SubstLemmas.} A few lemmas about compositions of
propositional equalities..

\paragraph*{VectorLemmas.} A few lemmas about vectors, lookups,
mapping functions over vectors, etc.

\paragraph*{FiniteFunctions.} Proves extensionality for finite functions. 

\paragraph*{Proofs.} Collects all the above and re-exports them along
with a couple of other general utilities for managing Agda proofs.

\paragraph*{DivModUtils.} External library for reasoning about
uniqueness of \texttt{divMod}. This is used extensively to mediate
between $\texttt{Fin}~(m*n)$ and
$\texttt{Fin}~m \times \texttt{Fin}~n$.

%%%%%%%%%%%%%%%%%%%%%%%%%%%%%%%%%%%%%%%%%%%%%%%%%%%%%%%%%%%%%%%%%%%%%%%
\section{Structures}

\paragraph*{SetoidUtils.} Any discrete type $A$ can be viewed as a
setoid with propositional equality $\equiv$ as the equivalence
relation.

\paragraph*{Groupoid.} A definition of 1-groupoids and operations on 
them. 

\paragraph*{Categories.Everything.} Basic and some advanced category theory.

\paragraph*{SymmetricMonoidalCategory.} A definition of symmetric
monoidal categories.

%%%%%%%%%%%%%%%%%%%%%%%%%%%%%%%%%%%%%%%%%%%%%%%%%%%%%%%%%%%%%%%%%%%%%%%
\section{Equivalences} 

\paragraph*{Equiv.} Defines extensional equivalence of functions
$\sim$ and shows that it is an equivalence relation. Defines
equivalence between sets $\simeq$ using two functions that go back and
forth and whose compositions are extensionally equivalent to the
identity, and shows that this equivalence is indeed an equivalence
relation. Finally shows that equivalences are injective and form a
congruence that respects $\uplus$ and $\times$. 

\paragraph*{TypeEquiv.} Establishes that the Agda types $\bot$,
$\top$, $\uplus$, and $\times$ form a commutative semiring using
$\simeq$ as the underlying equivalence relation.

\paragraph*{FinEquiv.} Establishes that $\texttt{Fin}~n$ also forms a
commutative semiring with $\simeq$ as the underlying equivalence
relation. In particular, we have:
\[\begin{array}{rcll}
\texttt{Fin}~0 &\simeq& \bot \\
\texttt{Fin}~1 &\simeq& \top \\
\texttt{Fin}~(m+n) &\simeq& \texttt{Fin}~m \uplus \texttt{Fin}~n \\
\texttt{Fin}~(m*n) &\simeq& \texttt{Fin}~m \times \texttt{Fin}~n
\end{array}\]
and then we have all the commutative semiring axioms, e.g.,
$\texttt{Fin}~(0+m) \simeq \texttt{Fin}~m$. The actual proof says that
0, 1, $+$, and $*$ for a commutative semiring structure under the
equivalence that equates $m$ and $n$ if
$\texttt{Fin}~m \simeq \texttt{Fin}~n$.

%%%%%%%%%%%%%%%%%%%%%%%%%%%%%%%%%%%%%%%%%%%%%%%%%%%%%%%%%%%%%%%%%%%%%%%
\section{Bimonoidal Categories} 

\paragraph*{SkFinSetCategory.} Defines the skeletal category of finite
sets and all functions between them. There is one object for each
natural number $n$ (including $n=0$), and a morphism from $m$ to $n$
is an $m$-tuple $(f_0,\ldots,f_{m−1})$ of numbers satisfying
$0 \leq f_i < n$. This structure will be a building block for
permutations (defined in \texttt{ConcretePermutation}).

%%%%%%%%%%%%%%%%%%%%%%%%%%%%%%%%%%%%%%%%%%%%%%%%%%%%%%%%%%%%%%%%%%%%%%%
\section{To do}

PiLevel0
ConcretePermutation
VecOps
PiPerm

Enumeration ??
SEequivSCPermEquiv ??

\paragraph*{Cauchy representation.} A permutation on $n$ elements is
represented by a vector \texttt{v : Vec (Fin n) n}. If the $n$
elements are indexed by positions, the element at position $i$ is
mapped to position \texttt{v !! i} by the permutation. There is always
a trivial permutation called \texttt{1C} that maps each position to
itself. The Cauchy representation does not enforce that the vector
entries are disjoint. This is enforced by the definition of ``concrete
permutations'' below.

\paragraph*{Concrete permutation.} A concrete permutation consists of
two Cauchy vectors and two proofs that their compositions is the
identity permutation \texttt{1C}. Concrete permutations are an
equivalence relation. Concrete permutations actually have more
structure: a sum, a unit for the sum, etc. We can also build setoids
whose carriers are concrete permutations under the standard $\equiv$
propositional equality.

%%%%%%%%%%%%%%%%%%%%%%%%%%%%%%%%%%%%%%%%%%%%%%%%%%%%%%%%%%%%%%%%%%%%%%%
\end{document}
