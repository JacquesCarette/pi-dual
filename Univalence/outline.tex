\documentclass{article}
\usepackage{fullpage}
\begin{document}
%%%%%%%%%%%%%%%%%%%%%%%%%%%%%%%%%%%%%%%%%%%%%%%%%%%%%%%%%%%%%%%%%%%%%%%
\section{Permutations}

\paragraph*{Cauchy representation.} A permutation on $n$ elements is represented by a vector \texttt{v : Vec (Fin n) n}. If the $n$ elements are indexed by positions, the element at position $i$ is mapped to position \texttt{v !! i} by the permutation. There is always a trivial permutation called \texttt{1C} that maps each position to itself. The Cauchy representation does not enforce that the vector entries are disjoint. This is enforced by the definition of ``concrete permutations'' below. 

\paragraph*{Concrete permutation.} A concrete permutation consists of two Cauchy vectors and two proofs that their compositions is the identity permutation \texttt{1C}. Concrete permutations are an equivalence relation. Concrete permutations actually have more structure: a sum, a unit for the sum, etc. We can also build setoids whose carriers are concrete permutations under the standard $\equiv$ propositional equality.

%%%%%%%%%%%%%%%%%%%%%%%%%%%%%%%%%%%%%%%%%%%%%%%%%%%%%%%%%%%%%%%%%%%%%%%
\section{?}


%%%%%%%%%%%%%%%%%%%%%%%%%%%%%%%%%%%%%%%%%%%%%%%%%%%%%%%%%%%%%%%%%%%%%%%
\end{document}