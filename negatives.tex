\documentclass[authoryear,preprint]{sigplanconf}
\usepackage{amsmath}
\usepackage{listings} 
\usepackage{stmaryrd}
\usepackage{latexsym}
\usepackage{amssymb}
\usepackage{xcolor}
\usepackage{courier}
\usepackage{thmtools}
\usepackage{bbold}

%%

\newcommand{\game}[2]{\langle #1 ~|~ #2 \rangle}
\newcommand{\zerog}{$\mathbb{0}$}
\newcommand{\oneg}{$\mathbb{1}$}
\newcommand{\twog}{$\mathbb{2}$}
\newcommand{\threeg}{$\mathbb{3}$}
\newcommand{\fourg}{$\mathbb{4}$}

%%

\begin{document}
\special{papersize=8.5in,11in}
\setlength{\pdfpageheight}{\paperheight}
\setlength{\pdfpagewidth}{\paperwidth}

\newcommand{\alt}{~|~}
\lstnewenvironment{code}{\lstset{basicstyle={\sffamily\footnotesize}}}{}

\lstset{frame=none,
         language=Haskell,
         basicstyle=\sffamily, 
         numberstyle=\tiny,
         numbersep=5pt,
         tabsize=2,    
         extendedchars=true,
         breaklines=true,   
         breakautoindent=true,
         keywordstyle=\color{black},
         captionpos=b,
         stringstyle=\color{black}\ttfamily,
         showspaces=false,  
         showtabs=false,    
         framexleftmargin=2em,
         framexbottommargin=1ex,
         showstringspaces=false
         basicstyle=\sffamily,
         columns=[l]flexible,
         flexiblecolumns=true,
         aboveskip=\smallskipamount,
         belowskip=\smallskipamount,
         lineskip=-1pt,
         xleftmargin=1em,
         escapeinside={/+}{+/},
         keywords=[1]{Monad,Just,Nothing,type,data,right,left,id,where,do,
                     if,then,else,let,in},
         literate={+*}{{$+*$}}1 {+}{{$\;+\;$}}1 {/}{{$/$}}1 {*}{{$\;*\;$}}1
           {=}{{$=\ $}}1 {/=}{{$\not=$}}2
           {[]}{$[\;]$}2
           {<}{{$<$}}1 {>}{{$\rangle$}}1 
           {++}{{$+\!\!\!+\;$}}1 {::}{{$:\mkern -2.5mu:\;$}}1
           {:+:}{{$:\mkern -5mu+\mkern -5mu:\;$}}2
           {$}{{\texttt{\$}\hspace{0.5em}}}1
           {`}{$^\backprime$}1
           {==}{{$=\!=\;$}}2
           {===}{{$\equiv\;$}}2
           {->}{{$\rightarrow\;$}}2 {>=}{{$\geq$}}2 {<-}{{$\leftarrow$}}2
           {<=}{{$\leq$}}2 {=>}{{$\Rightarrow\;$}}2
           {<<}{{$\ll$}}2 {>>}{{$\gg\;$}}2
           {>>>}{{$\ggg\;$}}3 {<<<}{{$\lll\;$}}3
           {>>=}{{$\gg\mkern -2.5mu=\;$}}3
           {=<<}{{$=\mkern -2.5mu\ll\;$}}3
           {<|}{$\lhd\;$}2
           {<||}{$\unlhd\;$}2
           {\ ||\ }{$\|$}1
           {\\}{$\lambda$}1
           {:>}{{$\rhd$}}2
           {||>}{{$\unrhd$}}2
           {_}{{$\_$}}1
           {_B}{{$_b$}}2
           {forall}{{$\forall$}}1
}

\lstset{postbreak=\raisebox{0ex}[0ex][0ex]
        {\ensuremath{\hookrightarrow}}}
\lstset{breaklines=true, breakatwhitespace=true}
\lstset{numbers=none, numbersep=5pt, stepnumber=2, numberstyle=\scriptsize}
\lstset{rangeprefix=/*!\ , rangesuffix=\ !*\/, includerangemarker=false}

%% double-blind reviewing...
\title{Negative Types}
\authorinfo{}{}{}
\maketitle

\begin{abstract}
\ldots
\end{abstract}

%%%%%%%%%%%%%%%%%%%%%%%%%%%%%%%%%%%%%%%%%%%%%%%%%%%%%%%%%%%%%%%%%%%%%%%%%%%%%%
\section{Introduction}

%%%%%%%%%%%%%%%%%%%%%%%%%%%%%%%%%%%%%%%%%%%%%%%%%%%%%%%%%%%%%%%%%%%%%%%%%%%%%%
\section{Conway Numbers}

A Conway number is a game consisting of left and right options where each
option leads to another game. In Haskell, one might define the datatype of
Conway numbers as:
\begin{code}
data ConwayNumber = CN [ConwayNumber] [ConwayNumber]
\end{code}
The simplest Conway number is \lstinline|CN [] []| with empty left and right
options. We call this number \zerog. Once we have defined \zerog, we can also
define the following:
\begin{code}
/+ \oneg +/  = CN  [ /+ \zerog +/ ]  []
/+ \twog +/  = CN  [ /+ \oneg +/ ]  []
/+ \threeg +/  = CN  [ /+ \twog +/ ]  []
\end{code}
and so on. Intuitively, the number $\mathbb{n}$ represents a game in which
the left player has an $n$-move advantage over the right player. Dually, we
can define the following numbers where the left player has $n$-move
\emph{disadvantage} over the right player:
\begin{code}
/+ -\oneg +/  = CN  [] [ /+ \zerog +/ ]
/+ -\twog +/  = CN  [] [ /+ \oneg +/ ]
/+ -\threeg +/  = CN  [] [ /+ \twog +/ ]
\end{code}

%%%%%%%%%%%%%%%%%%%%%%%%%%%%%%%%%%%%%%%%%%%%%%%%%%%%%%%%%%%%%%%%%%%%%%%%%%%%%%
\bibliographystyle{abbrvnat}
\softraggedright
\bibliography{cites}

\end{document}

