\documentclass[authoryear,preprint]{sigplanconf}
\usepackage{amsmath}
\usepackage{listings} 
\usepackage{stmaryrd}
\usepackage{latexsym}
\usepackage{amssymb}
\usepackage{xcolor}
\usepackage{courier}
\usepackage{thmtools}
\usepackage{bbold}

%%

\newcommand{\game}[2]{\langle #1 ~|~ #2 \rangle}
\newcommand{\zerog}{$\mathbb{0}$}
\newcommand{\oneg}{$\mathbb{1}$}
\newcommand{\twog}{$\mathbb{2}$}
\newcommand{\threeg}{$\mathbb{3}$}
\newcommand{\fourg}{$\mathbb{4}$}
\newcommand{\fiveg}{$\mathbb{5}$}

%%

\begin{document}
\special{papersize=8.5in,11in}
\setlength{\pdfpageheight}{\paperheight}
\setlength{\pdfpagewidth}{\paperwidth}

\newcommand{\alt}{~|~}
\lstnewenvironment{code}{\lstset{basicstyle={\sffamily\footnotesize}}}{}

\lstset{frame=none,
         language=Haskell,
         basicstyle=\sffamily, 
         numberstyle=\tiny,
         numbersep=5pt,
         tabsize=2,    
         extendedchars=true,
         breaklines=true,   
         breakautoindent=true,
         keywordstyle=\color{black},
         captionpos=b,
         stringstyle=\color{black}\ttfamily,
         showspaces=false,  
         showtabs=false,    
         framexleftmargin=2em,
         framexbottommargin=1ex,
         showstringspaces=false
         basicstyle=\sffamily,
         columns=[l]flexible,
         flexiblecolumns=true,
         aboveskip=\smallskipamount,
         belowskip=\smallskipamount,
         lineskip=-1pt,
         xleftmargin=1em,
         escapeinside={/+}{+/},
         keywords=[1]{Monad,Just,Nothing,type,data,right,left,id,where,do,
                     if,then,else,let,in},
         literate={+*}{{$+*$}}1 {+}{{$\;+\;$}}1 {/}{{$/$}}1 {*}{{$\;*\;$}}1
           {=}{{$=\ $}}1 {/=}{{$\not=$}}2
           {[]}{$[\;]$}2
           {<}{{$<$}}1 {>}{{$\rangle$}}1 
           {++}{{$+\!\!\!+\;$}}1 {::}{{$:\mkern -2.5mu:\;$}}1
           {&&}{{$\&\!\!\!\&$}}2
           {:=:}{{$:\mkern -2mu=\mkern -2mu:\;$}}2
           {:+:}{{$:\mkern -5mu+\mkern -5mu:\;$}}2
           {:-:}{{$:\mkern -5mu-\mkern -5mu:\;$}}2
           {:*:}{{$:\mkern -5mu*\mkern -5mu:\;$}}2
           {$}{{\texttt{\$}\hspace{0.5em}}}1
           {`}{$^\backprime$}1
           {==}{{$=\!=\;$}}2
           {===}{{$\equiv\;$}}2
           {->}{{$\rightarrow\;$}}2 {>=}{{$\geq$}}2 {<-}{{$\leftarrow$}}2
           {<=}{{$\leq$}}2 {=>}{{$\Rightarrow\;$}}2
           {<<}{{$\ll$}}2 {>>}{{$\gg\;$}}2
           {>>>}{{$\ggg\;$}}3 {<<<}{{$\lll\;$}}3
           {>>=}{{$\gg\mkern -2.5mu=\;$}}3
           {=<<}{{$=\mkern -2.5mu\ll\;$}}3
           {<|}{$\lhd\;$}2
           {<||}{$\unlhd\;$}2
           {\ ||\ }{$\|$}1
           {\\}{$\lambda$}1
           {:>}{{$\rhd$}}2
           {||>}{{$\unrhd$}}2
           {_}{{$\_$}}1
           {_B}{{$_b$}}2
           {forall}{{$\forall$}}1
}

\lstset{postbreak=\raisebox{0ex}[0ex][0ex]
        {\ensuremath{\hookrightarrow}}}
\lstset{breaklines=true, breakatwhitespace=true}
\lstset{numbers=none, numbersep=5pt, stepnumber=2, numberstyle=\scriptsize}
\lstset{rangeprefix=/*!\ , rangesuffix=\ !*\/, includerangemarker=false}

%% double-blind reviewing...
\title{Negative Types}
\authorinfo{}{}{}
\maketitle

\begin{abstract}
\ldots
\end{abstract}

%%%%%%%%%%%%%%%%%%%%%%%%%%%%%%%%%%%%%%%%%%%%%%%%%%%%%%%%%%%%%%%%%%%%%%%%%%%%%%
\section{Introduction}

%%%%%%%%%%%%%%%%%%%%%%%%%%%%%%%%%%%%%%%%%%%%%%%%%%%%%%%%%%%%%%%%%%%%%%%%%%%%%%
\section{Conway Numbers}

A Conway number is a game consisting of left and right options where each
option models a move to another game. Players alternate taking options and
the player with no available options loses. In Haskell, one might define the
datatype of Conway numbers as:
\begin{code}
data ConwayNumber = CN [ConwayNumber] [ConwayNumber]
\end{code}
The simplest Conway number is \lstinline|CN [] []| with empty left and right
options. We call this number \zerog: 
\begin{code}
/+ \zerog +/  = CN  [] []
\end{code}
Once we have defined \zerog, we can also define the following:
\begin{code}
/+ \oneg +/  = CN  [ /+ \zerog +/ ]  []
/+ \twog +/  = CN  [ /+ \oneg +/ ]  []
/+ \threeg +/  = CN  [ /+ \twog +/ ]  []
\end{code}
and so on. Intuitively, the number $\mathbb{n}$ represents a game in which
the left player has an $n$-move advantage over the right player. Dually, we
can define the following numbers where the left player has $n$-move
\emph{disadvantage} over the right player:
\begin{code}
/+ -\oneg +/  = CN  [] [ /+ \zerog +/ ]
/+ -\twog +/  = CN  [] [ /+ -\oneg +/ ]
/+ -\threeg +/  = CN  [] [ /+ -\twog +/ ]
\end{code}
More generally, the \emph{unary negation} of a number is defined as follows:
\begin{code}
neg :: ConwayNumber -> ConwayNumber
neg (CN xls xrs) = 
  CN [ neg xr | xr <- xrs ] [ neg xl | xl <- xls ]
\end{code}
Conway numbers also come equipped with addition and multiplication operations
defined as follows. 

\noindent\textbf{Addition and subtraction.} Addition of two games intuitively
gives the player the choice of selecting an option from either
game. Substraction is simply the addition of the dual of a game. Formally:
\begin{code}
(:+:) :: ConwayNumber -> ConwayNumber -> ConwayNumber
x@(CN xls xrs) :+: y@(CN yls yrs) = 
  CN 
    ([ xl :+: y | xl <- xls ] `union`
     [ x :+: yl | yl <- yls ])
    ([ xr :+: y | xr <- xrs ] `union`
     [ x :+: yr | yr <- yrs ])

(:-:) :: ConwayNumber -> ConwayNumber -> ConwayNumber
x :-: y = x :+: (neg y) 
\end{code}
It is easy to check that adding two positive numbers like \twog\ and
\threeg\ gives \fiveg\ as desired, and similarly for two negative
numbers. When mixing positive and negative numbers, e.g., adding \threeg\ and
$-$\twog, the result appears much more complicated:
\begin{code}
CN [CN  [CN  [-/+\twog+/] 
             [CN [-/+\oneg+/] [/+\oneg+/]]] 
        [CN  [CN [-/+\oneg+/] [/+\oneg+/]] 
             [/+\twog+/]]] 
   [CN  [CN  [CN [-/+\oneg+/] [/+\oneg+/]] 
             [/+\twog+/]] 
        [/+\threeg+/]]
\end{code}
There is however an intuitive way to understand why the game above is
equivalent to the game \oneg\ (which is formalized below). Consider the game
\lstinline{CN [-1] [1]}: it is equivalent to the game \zerog.

\noindent\textbf{Multiplication.} 

\begin{code}
(:*:) :: ConwayNumber -> ConwayNumber -> ConwayNumber
x@(CN xls xrs) :*: y@(CN yls yrs) = 
  CN 
    ([ (xl :*: y) :+: (x :*: yl) :-: (xl :*: yl)
     | xl <- xls, yl <- yls] `union`
     [ (xr :*: y) :+: (x :*: yr) :-: (xr :*: yr)
     | xr <- xrs, yr <- yrs])
    ([ (xl :*: y) :+: (x :*: yr) :-: (xl :*: yr)
     | xl <- xls, yr <- yrs] `union`
     [ (xr :*: y) :+: (x :*: yl) :-: (xr :*: yl)
     | xr <- xrs, yl <- yls])
\end{code}

Conway numbers form a \emph{ring} under the semantic notion of equality
\lstinline$:=:$ defined below:
\begin{code}
geq/+\zerog+/, leq/+\zerog+/, eq/+\zerog+/ :: ConwayNumber -> Bool
geq/+\zerog+/ (CN _ xrs)  = not $ or (map leq/+\zerog+/ xrs) 
leq/+\zerog+/ (CN xls _)  = not $ or (map geq/+\zerog+/ xls) 
eq/+\zerog+/ x            = geq/+\zerog+/ x && leq/+\zerog+/ x 

(:=:) :: ConwayNumber -> ConwayNumber -> ConwayNumber
x :=: y = eq/+\zerog+/ (x :-: y)
\end{code}


%%%%%%%%%%%%%%%%%%%%%%%%%%%%%%%%%%%%%%%%%%%%%%%%%%%%%%%%%%%%%%%%%%%%%%%%%%%%%%
\bibliographystyle{abbrvnat}
\softraggedright
\bibliography{cites}

\end{document}

